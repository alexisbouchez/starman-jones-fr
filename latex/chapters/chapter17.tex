\chapter{Charité}

\og Charitéville \fg{} était une affaire qui marchait en moins d'une semaine. Elle avait un maire, M. Daigler, une rue principale, l'Avenue Hendrix, même son premier mariage, célébré par le maire en présence des villageois -- M. Arthur et la petite Becky Weberbauer. La première maison, en cours de construction, était réservée aux jeunes mariés. C'était une cabane en rondins et un travail très bâclé, car, bien qu'il y eût parmi eux ceux qui avaient vu des photos ou même vu des cabanes en rondins, il n'y avait personne qui en eût jamais construit une auparavant.

Il y avait un air d'espoir, de courage commun, même de gaieté dans la nouvelle communauté. L'endroit était parfumé de nouveaux départs, de pensées tournées vers l'avenir. Ils dormaient encore dans le vaisseau et y prenaient le petit-déjeuner, puis emportaient leurs déjeuners et travaillaient puissamment, hommes et femmes pareillement, pendant la courte journée -- Charité tournait sur son axe en vingt et une heures et quelques. Ils rentraient à la tombée de la nuit, dînaient dans le vaisseau, et certains trouvaient l'énergie de danser un peu avant d'aller se coucher.

Charité semblait être tout ce que son nom impliquait. Les jours étaient doux, les nuits étaient clémentes -- et belles au-delà de tout ce qu'on avait encore trouvé dans la Galaxie. Son étoile (ils l'appelaient simplement \og le Soleil \fg{}) était accompagnée de plus de comètes qu'on n'en avait encore vu autour d'aucune étoile. Une géante avec une large queue s'étirait du zénith à l'horizon occidental, plongeant vers leur Soleil. Une autre, pas encore aussi grandiose mais assez impressionnante pour avoir causé des veillées pour la fin du monde sur les collines terrestres, approchait du nord, et deux autres décoraient le ciel austral d'une dentelle de feu glacé. Corollaire des comètes était, nécessairement, une abondance égale de météores. Chaque nuit était une pluie d'étoiles filantes, chaque jour finissait comme le Jour de l'Union Solaire avec un feu d'artifice.

Ils n'avaient vu aucun animal dangereux. Certains des colons rapportaient avoir vu des créatures semblables à des centaures de la taille de poneys Shetland, mais elles semblaient timides et avaient filé quand elles avaient été découvertes. La forme de vie prédominante semblait être des mammifères marsupiaux de diverses tailles et formes. Il n'y avait pas d'oiseaux, mais il y avait un autre type de vie volante qu'on ne trouvait nulle part ailleurs -- des créatures semblables à des méduses de un mètre vingt à un mètre cinquante de haut avec des tentacules pendants, des ballons animés. Elles semblaient avoir un contrôle musculaire sur leurs vessies gonflées car elles pouvaient monter et descendre, et pouvaient même, par quelque moyen non évident, remonter le vent contre une brise légère -- par vents plus forts elles s'ancraient aux cimes des arbres, ou flottaient librement et se laissaient porter par le vent. Elles semblaient curieuses de Charitéville et restaient au-dessus d'un chantier, tournant lentement comme pour tout voir. Mais elles ne venaient jamais à portée de main.

Certains des colons voulaient en abattre une et l'examiner~; le Maire Daigler l'interdit.

Il y avait un autre animal aussi -- ou peut-être. On les appelait les \og zieuters \fg{} parce que tout ce que quiconque avait vu était quelque chose qui se cachait rapidement derrière un rocher ou un arbre quand quelqu'un essayait de regarder. Entre le zieuter possiblement mythique et les ballons omniprésents, les colons avaient le sentiment que leurs nouveaux voisins prenaient un intérêt profond mais pas inamical à ce qu'ils faisaient.

Maggie Daigler -- elle était \og Maggie \fg{} pour tout le monde maintenant -- avait rangé ses bijoux, tiré une salopette du magasin du vaisseau, et s'était coupé les cheveux. Ses ongles étaient courts et généralement noirs de crasse. Mais elle avait l'air plus jeune de plusieurs années et tout à fait heureuse. En fait, tout le monde semblait heureux sauf Max.

Ellie l'évitait. Il se maudissait lui et sa grande bouche trois fois par jour et quatre fois la nuit. Bien sûr, Daigler avait parlé à tort et à travers -- mais était-ce une raison pour lui d'ouvrir la bouche et d'y mettre les pieds~? Bien sûr, il n'avait jamais envisagé d'épouser Ellie -- mais bon sang, peut-être qu'ils étaient coincés ici pour toujours. \og Probablement \fg{}, pas \og peut-être \fg{}, se corrigea-t-il. L'interdiction de rejoindre la colonie serait levée avec le temps -- auquel cas, quel sens y avait-il à se mettre mal avec la seule fille éligible dans les parages~? Un astrogateur devrait être célibataire mais un fermier avait besoin d'une femme. Drôlement agréable d'avoir quelqu'un pour cuisiner les fanes de navet et découper un poulet pendant qu'un homme était aux champs. Il devait le savoir -- M'man avait laissé couler assez souvent.

Ellie ne serait pas comme M'man. Elle était forte et pratique et avec juste un peu d'enseignement s'en sortirait très bien. En plus elle était à peu près la plus jolie chose qu'il ait jamais vue, si on la regardait bien.

Quand M. et Mme Dumont, par dispense spéciale, rejoignirent la colonie, cela le fit agir. Puisque le steward et la stewardess n'auraient pas de fonctions dans un vaisseau sans passagers, personne ne pouvait raisonnablement s'y opposer -- mais cela donnait à Max une approche. Il alla voir le Premier Officier.

\og Apprenti Probatoire Jones, monsieur. \fg{}

Walther leva les yeux. \og Je pense que je dirais "Assistant Astrogateur Jones" si j'étais vous. Plus proche des faits. Entrez. \fg{}

\og Euh, c'est de ça que je voulais vous parler, monsieur. \fg{}

\og Ah~? Comment~? \fg{}

\og Je veux revenir à mon poste. \fg{}

\og Hein~? Pourquoi préféreriez-vous être cartographe qu'astrogateur~? Et quelle différence cela fait-il -- maintenant~? \fg{}

\og Non, monsieur. Je choisis de reprendre ma nomination permanente, aide-steward de troisième classe. \fg{}

Walther parut stupéfait. \og Il doit y avoir plus que ça. Expliquez-vous. \fg{}

Avec beaucoup de bégaiements Max expliqua ses ennuis avec Simes. Il essaya d'être juste et finit avec le sentiment lamentable qu'il avait sonné puéril. Walther dit~: \og Vous êtes sûr de cela~? M. Simes ne m'a rien dit à votre sujet. \fg{}

\og Il ne le ferait pas, monsieur. Mais c'est vrai. Vous pouvez demander à Kelly. \fg{}

Walther réfléchit un moment. \og M. Jones, je n'attacherais pas trop d'importance à cela. À votre âge ces conflits de personnalité semblent souvent plus graves qu'ils ne le sont. Mon conseil est d'oublier cela et de faire votre travail. Je parlerai à M. Simes de ce qu'il vous tient à l'écart de la salle de contrôle. Ce n'est pas convenable et je suis surpris de l'entendre. \fg{}

\og Non, monsieur. \fg{}

\og "Non, monsieur" quoi~? \fg{}

\og Je veux retourner aide-steward. \fg{}

\og Hein~? Je ne vous comprends pas. \fg{}

\og Parce que, monsieur, je veux rejoindre la colonie. Comme le Chef Steward Dumont. \fg{}

\og Oh... Une lumière commence à poindre. \fg{} Walther frappa le bureau avec emphase. \og Absolument pas~! En aucune circonstance. \fg{}

\og Monsieur~? \fg{}

\og Comprenez-moi bien, s'il vous plaît. Ce n'est pas de la discrimination. Si vous étiez un aide-steward et rien d'autre, j'examinerais votre demande -- dans les circonstances spéciales que je crois pertinentes. Mais vous êtes un astrogateur. Vous connaissez notre situation. Le Dr Hendrix est mort. Le Capitaine Blaine -- eh bien, vous l'avez vu. Il peut se rétablir, je ne peux pas compter dessus. M. Jones, tant qu'il y a le moindre faible espoir que ce vaisseau redécolle jamais, tant que nous avons un équipage pour le manœuvrer, aucun astrogateur, aucun cartographe, aucun calculateur ne sera relevé de ses fonctions pour quelque raison que ce soit. Vous voyez cela, n'est-ce pas~? \fg{}

\og Je suppose, monsieur. Euh, bien, monsieur. \fg{}

\og Bien. Au fait, gardez cela pour vous, mais dès que la colonie pourra se passer de nous temporairement, je veux que le vaisseau soit placé en orbite de stationnement pour que vous, spécialistes, puissiez maintenir une recherche. Vous ne pouvez pas très bien travailler à travers cette atmosphère, n'est-ce pas~? \fg{}

\og Non, monsieur. Nos instruments ont été conçus pour l'espace ouvert. \fg{}

\og Alors nous devons faire en sorte que vous l'ayez. \fg{} Le Premier Officier resta silencieux, puis ajouta~: \og M. Jones -- Max, n'est-ce pas~? Puis-je vous parler d'homme à homme~? \fg{}

\og Hein~? Certainement, monsieur. \fg{}

\og Mmm... Max, ce ne sont pas mes affaires, mais considérez cela comme un conseil paternel. Si vous avez l'occasion de vous marier -- et le voulez -- vous n'avez pas à rejoindre la colonie pour le faire. Si nous restons, cela n'aura pas d'importance à long terme que vous soyez de l'équipage ou un membre fondateur du village. Si nous partons, votre femme part avec vous. \fg{}

Les oreilles de Max brûlaient. Il ne trouvait rien à dire.

\og Question hypothétique, bien sûr. Mais c'est la bonne solution. \fg{} Walther se leva. \og Pourquoi ne pas prendre la journée libre~? Allez faire une promenade ou quelque chose. L'air frais vous fera du bien. Je parlerai à M. Simes. \fg{}

Au lieu de cela, Max alla chercher Sam, ne le trouva pas dans le vaisseau, découvrit qu'il était descendu à terre. Il le suivit et marcha les huit cents mètres jusqu'à Charitéville. Avant d'atteindre le bâtiment sur lequel on travaillait, il vit une silhouette se détacher du groupe. Il vit bientôt que c'était Eldreth.

Elle s'arrêta devant lui, une petite silhouette robuste en salopette sale. Elle planta ses pieds et mit les poings sur les hanches. \og Euh, salut, Ellie. \fg{}

\og Toujours tes vieux tours~! Tu m'évites. Explique-toi. \fg{}

L'injustice de cela le laissa bégayant. \og Mais... Mais enfin, Ellie, ce n'est pas du tout comme ça. Tu as... \fg{}

\og Une histoire vraisemblable. On dirait Chipsie prise la main dans le bonbon. Je voulais juste te dire, Don Juan réticent, que tu n'as rien à craindre. Je n'épouse personne cette saison. Alors tu peux reprendre le cours inégal de ta vie. \fg{}

\og Mais, Ellie... \fg{} commença-t-il désespérément.

\og Tu veux que je le mette par écrit~? Que je donne une caution~? \fg{} Elle le regarda férocement, puis commença à rire, plissant le nez. \og Oh, Max, grand lourdaud, tu éveilles l'éternel maternel en moi. Quand tu es contrarié ton visage s'allonge comme celui d'une mule. Écoute, oublie ça. \fg{}

\og Mais, Ellie... Bon, d'accord. \fg{}

\og Copains~? \fg{}

\og Copains. \fg{}

Elle soupira. \og Je me sens mieux. Je ne sais pas pourquoi, mais je n'aime pas être fâchée avec toi. Où allais-tu~? \fg{}

\og Euh, nulle part. Je me promenais. \fg{}

\og Parfait. Je viens aussi. Une demi-seconde que je récupère Chipsie. \fg{} Elle se retourna et appela~: \og Monsieur Chips~! Chipsie~! \fg{}

\og Je ne la vois pas. \fg{}

\og Je vais la chercher. \fg{} Elle partit en courant, pour revenir rapidement avec le chiot-araignée sur son épaule et un paquet dans la main. \og J'ai pris mon déjeuner. On peut le partager. \fg{}

\og Oh, on ne sera pas partis si longtemps. Salut, bébé Chipsie. \fg{}

\og Salut, Max. Bonbon~? \fg{}

Il fouilla dans une poche, trouva un morceau de sucre qu'il avait gardé plusieurs jours auparavant dans ce but~; le chiot-araignée l'accepta gravement et dit~: \og Merci. \fg{}

\og Si, on sera partis si longtemps, \fg{} contredit Ellie, \og parce que certains des hommes ont vu un troupeau de ces poneys centaures de l'autre côté de cette crête. C'est une sacrée marche. \fg{}

\og Je ne pense pas qu'on devrait aller si loin, \fg{} dit-il avec doute. \og Tu ne vas pas leur manquer~? \fg{}

\og J'ai fait ma part. Tu vois mes callosités~? \fg{} Elle tendit une patte crasseuse. \og J'ai dit à M. Hornsby que j'avais soudainement attrapé la flemme aiguë et qu'il devrait trouver quelqu'un d'autre pour tenir pendant qu'il martelait. \fg{}

Il fut content de céder. Ils montèrent un terrain en pente et entrèrent dans un arroyo et furent bientôt dans un bosquet de conifères primitifs. Mademoiselle Chips sauta des épaules d'Ellie et grimpa en vitesse sur un arbre. Max s'arrêta. \og On ne devrait pas la rattraper~? \fg{}

\og Tu t'inquiètes trop. Chipsie ne s'enfuirait pas. Elle mourrait de peur. Chipsie~! Ici, ma chérie~! \fg{}

Le chiot-araignée se faufila à travers les branches, se plaça directement au-dessus d'eux, laissa tomber une pomme de pin sur Max. Puis elle rit, un gloussement aigu. \og Tu vois~? Elle veut juste jouer. \fg{}

La crête était haute et Max découvrit que son souffle de montagnard avait été perdu quelque part parmi les étoiles. L'arroyo serpentait lentement vers le haut. Il était encore assez homme des bois pour garder un œil vif sur les repères et les directions. Enfin, épuisé, ils atteignirent le sommet. Ellie fit une pause. \og Je suppose qu'ils sont partis, \fg{} dit-elle avec déception, regardant le pays plus plat en dessous d'eux.

\og Non~! Regarde là-bas. Tu les vois~! Une vingtaine de petits points noirs. \fg{}

\og Oui. Ouais. \fg{}

\og Approchons-nous. Je veux bien les voir. \fg{}

\og Je me demande si c'est malin~? On est loin du vaisseau et je ne suis pas armé. \fg{}

\og Oh, ils sont inoffensifs. \fg{}

\og Je pensais à ce qu'il pourrait y avoir d'autre dans ces bois. \fg{}

\og Mais on est déjà dans les bois, et tout ce qu'on a vu ce sont les farfadets. \fg{} Elle faisait référence aux créatures ballons, dont deux les avaient suivis dans l'arroyo. Les humains s'étaient tellement habitués à leur présence qu'ils ne leur prêtaient plus aucune attention. \og Ellie, c'est l'heure de rentrer. \fg{}

\og Non. \fg{}

\og Si. Je suis responsable de toi. Tu as vu tes centaures. \fg{}

\og Max Jones, je suis une citoyenne libre. Tu peux rentrer~; moi je vais aller voir de près ces vaches centaures. \fg{} Elle commença à descendre.

\og Eh bien -- Attends un moment. Je veux prendre mes repères. \fg{} Il prit une bonne vue d'ensemble, fixa la scène pour toujours dans son esprit, et la suivit. Il n'était pas vraiment désireux de la contrarier de toute façon~; il avait ruminé l'idée que c'était le bon moment pour expliquer pourquoi il avait dit ce qu'il avait dit à M. Daigler -- et peut-être amener le sujet général de l'avenir. Il n'irait pas jusqu'à parler de mariage -- bien qu'il puisse l'aborder dans l'abstrait s'il pouvait trouver une approche. Comment abordait-on un tel sujet~? On ne disait pas juste~: \og Tiens, voilà les farfadets, marions-nous~! \fg{}

Ellie fit une pause. \og Tiens, voilà les farfadets. On dirait qu'ils se dirigent droit vers le troupeau. \fg{}

Max fronça les sourcils. \og Peut-être. Peut-être qu'ils leur parlent~? \fg{}

Elle rit. \og Ces trucs~? \fg{} Elle l'examina soigneusement. \og Maxie, je viens de comprendre pourquoi je me donne la peine de m'occuper de toi. \fg{}

Hein~? Peut-être qu'elle allait amener le sujet pour lui. \og Pourquoi~? \fg{}

\og Parce que tu me rappelles Putzie. Tu as le même air perplexe que lui. \fg{}

\og "Putzie"~? Qui est Putzie~? \fg{}

\og Putzie est l'homme dont mon père m'a expédiée sur Terre pour m'éloigner -- et la raison pour laquelle je me suis évadée de trois écoles pour retourner sur Hespéra. Seulement Papa l'aura probablement expédié aussi. Papa est rusé. Viens ici, Chipsie. Ne va pas si loin. \fg{} Elle continua~: \og Tu vas adorer Putzie. Il est gentil. Arrête, Chipsie. \fg{}

Max méprisait déjà l'homme. \og Je ne veux pas t'ennuyer, \fg{} dit-il, \og mais c'est loin jusqu'à Hespéra. \fg{}

\og Je sais. N'empruntons pas les ennuis. \fg{} Elle l'examina de nouveau. \og Je pourrais te garder en réserve, si tu n'étais pas si nerveux. \fg{}

Avant qu'il ne puisse penser à la bonne réponse, elle avait recommencé à descendre. Les centaures -- c'était le meilleur nom, bien que les parties inférieures ne ressemblaient pas beaucoup à des chevaux et les parties qui dépassaient n'étaient que vaguement humanoïdes -- se regroupaient près du pied de la colline, pas loin des arbres. Ils ne broutaient pas, il était difficile de dire ce qu'ils faisaient. Les deux farfadets étaient au-dessus du groupe, planant comme avec intérêt, exactement comme ils le faisaient avec les humains.

Ellie insista pour aller au bord de la clairière pour mieux les voir. Ils rappelaient à Max des clowns maquillés pour ressembler à des chevaux. Ils avaient des expressions idiotes et simples et apparemment pas de place pour une boîte crânienne. Ils semblaient être des marsupiaux, avec des poches presque comme des bavoirs. Soit ils étaient tous des femelles, soit chez cette espèce le mâle avait une poche aussi. Plusieurs petits centaures gambadaient, entrant et sortant des jambes de leurs aînés.

Un des bébés les repéra, vint trottant vers eux, reniflant et bêlant. Derrière lui le plus grand adulte sortit du troupeau pour surveiller le petit. Le poulain accourut et s'arrêta à environ six mètres. \og Oh, le chéri~! \fg{} dit Ellie et courut quelques mètres, tomba sur un genou. \og Viens ici, mon mignon. Viens voir maman. \fg{}

Max se précipita vers elle. \og Ellie~! Reviens ici~! \fg{}

Le grand centaure plongea dans sa poche, en sortit quelque chose, le fit tournoyer autour de sa tête comme la corde de lancer d'un gaucho. \og Ellie~! \fg{} Il la rejoignit juste au moment où la créature lâcha prise. La chose les frappa, s'enroula autour et les maintint. Ellie cria et Max lutta pour s'en défaire -- mais ils étaient tenus comme Laocoon. Une autre ligne vint voler dans les airs, s'accrocha à eux. Et une autre. Mademoiselle Chips avait suivi Ellie. Maintenant elle détala en pleurant. Elle s'arrêta au bord de la clairière et cria~: \og Max~! Ellie~! \emph{Revenez}. \emph{S'il vous plaît} revenez~! \fg{}
