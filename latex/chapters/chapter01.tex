\chapter{Le \textit{Tomahawk}}

Max aimait ce moment de la journée, cette période de l'année. Les récoltes rentrées, il pouvait terminer ses corvées du soir de bonne heure et paresser un peu. Après avoir donné leur pâtée aux cochons et nourri les poules, au lieu de préparer le dîner, il avait suivi un sentier menant à une butte à l'ouest de la grange et s'était étendu dans l'herbe, sans se soucier des aoûtats. Il avait emporté un livre emprunté samedi dernier à la bibliothèque du comté, \textit{Bêtes du Ciel~: Guide de zoologie exotique} de Bonforte, mais il l'avait glissé sous sa tête en guise d'oreiller. Un geai bleu fit des remarques sur son honnêteté, puis se tut quand Max resta immobile. Un écureuil roux, perché sur une souche, l'observa fixement avant de retourner à ses provisions de noix. Max gardait les yeux rivés vers le nord-ouest. Il affectionnait cet endroit car de là, il pouvait voir les pylônes d'acier et les anneaux de guidage de la Ligne Circulaire Chicago, Springfield \& Earthport émerger d'une entaille dans la crête sur sa droite. Un anneau de guidage se dressait à la sortie de la tranchée, un grand cerceau d'acier de six mètres de haut. Une paire de trépieds élancés comme des échasses soutenait un autre anneau à trente mètres de la tranchée. Un troisième et dernier anneau, ses pylônes hauts de plus de trente mètres pour le maintenir au niveau des autres, se trouvait à l'ouest de lui, là où le terrain s'abaissait encore plus abruptement vers la vallée en contrebas. À mi-hauteur, il pouvait distinguer l'antenne de liaison énergétique pointée vers l'autre versant. Sur sa gauche, les guides de la C.S.\&E. reprenaient de l'autre côté du gouffre. L'anneau d'entrée était plus large pour compenser la déviation maximale due au vent~; sur ses pylônes était fixée l'antenne réceptrice de la liaison énergétique. Cette crête-là était plus escarpée~; il n'y avait plus qu'un seul anneau avant que la voie ne disparaisse dans un tunnel. Il avait lu que sur la Lune, les anneaux d'entrée n'étaient pas plus grands que les anneaux de passage, puisqu'il n'y avait jamais de vent pour causer des variations balistiques. Quand il était enfant, cet anneau d'entrée avait été légèrement plus petit et, lors d'une tempête sans précédent, un train avait percuté l'anneau, provoquant un déraillement inimaginable qui avait fait plus de quatre cents morts. Il n'avait pas vu l'accident et son père ne l'avait pas laissé fouiner dans les parages à cause du carnage, mais la cicatrice en restait visible sur la crête de gauche, d'un vert plus sombre que le reste. Il regardait passer les trains dès qu'il le pouvait, sans souhaiter de malheur aux passagers --- mais tout de même, s'il devait y avoir une catastrophe, il ne voulait pas la manquer.

Max gardait les yeux fixés sur la tranchée~; le \textit{Tomahawk} devait arriver d'un instant à l'autre. Soudain, il y eut un éclair argenté~; un cylindre étincelant au nez effilé jaillit de la tranchée, traversa le dernier anneau en un éclair et, pendant un instant à couper le souffle, fut en trajectoire libre entre les deux crêtes. Avant qu'il ait pu tourner la tête, le projectile pénétra dans l'anneau de l'autre côté du gouffre et disparut dans le flanc de la colline --- juste au moment où le son le frappait. Ce fut un coup de tonnerre qui rebondit entre les collines.

Max eut le souffle coupé. \og Ça alors~! \fg{} murmura-t-il. \og Ça alors, ça alors~! \fg{}

Ce spectacle incroyable et l'impact sur ses oreilles lui faisaient toujours le même effet. Il avait entendu dire que pour les passagers le train était silencieux, le son restant derrière eux, mais il n'en savait rien~; il n'avait jamais pris le train et il semblait peu probable, avec M'man et la ferme dont il fallait s'occuper, qu'il le fasse un jour.

Il se redressa en position assise et ouvrit son livre, le tenant de façon à rester conscient du ciel au sud-ouest. Sept minutes après le passage du \textit{Tomahawk}, il devrait pouvoir voir, par une soirée claire, l'orbite de lancement de la navette lunaire quotidienne. Bien que beaucoup plus lointain et beaucoup moins spectaculaire que le saut du train circulaire tout proche, c'était cela qu'il était venu voir. Les trains circulaires, c'était bien, mais les vaisseaux spatiaux étaient sa passion --- même un petit engin comme la navette lunaire.

Mais il venait à peine de retrouver sa page, une description des crustacés intelligents mais flegmatiques d'Epsilon Ceti~IV, quand il fut interrompu par un appel derrière lui.

\og Maxie~! Maximilian~! Max\ldots mi\ldots \textit{lian}~! \fg{}

Il resta immobile et ne dit rien.

\og \textit{Max}~! Je te vois, Max --- viens tout de suite, tu m'entends~? \fg{}

Il marmonna dans sa barbe et se leva. Il descendit lentement le sentier, regardant le ciel par-dessus son épaule jusqu'à ce que la grange lui bouche la vue. M'man était de retour et c'était comme ça --- elle lui rendrait la vie impossible s'il ne rentrait pas l'aider. Quand elle était partie ce matin, il avait eu l'impression qu'elle serait absente pour la nuit --- non qu'elle l'eût dit~; elle ne le faisait jamais --- mais il avait appris à décoder les signes. Maintenant, il allait devoir écouter ses plaintes et ses ragots insignifiants alors qu'il voulait lire, ou tout aussi pénible, être dérangé par les feuilletons larmoyants de la stéréovision qu'elle affectionnait. Il avait souvent été tenté de saboter ce fichu appareil de SV --- à la hache, tant qu'à faire~! Il n'arrivait presque jamais à regarder les programmes qu'il aimait.

Quand il aperçut la maison, il s'arrêta net. Il avait supposé que M'man avait pris le bus depuis le Carrefour et remonté le ravin à pied comme d'habitude. Mais il y avait un petit monocycle sportif garé près du perron --- et quelqu'un l'accompagnait. Il avait d'abord cru que c'était un \og étranger \fg{} --- mais en s'approchant, il reconnut l'homme.

Max aurait préféré voir un étranger, n'importe quel étranger.

Biff Montgomery était un gars du coin mais il n'exploitait pas de ferme~; Max ne se souvenait pas de l'avoir vu faire un travail honnête. Il avait entendu dire que Montgomery se faisait parfois embaucher comme garde quand un des alambics clandestins dans les collines était en activité, et cela pouvait être vrai --- Montgomery était un homme grand et corpulent, et le rôle pouvait lui convenir. Max connaissait Montgomery depuis toujours, l'avait vu traîner au Carrefour de Clyde. Mais d'ordinaire il l'avait évité et n'avait rien eu à faire avec lui --- jusqu'à récemment~: M'man avait commencé à être vue avec lui, était même allée à des bals et des corvées de battage avec lui. Max avait essayé de lui dire que Papa n'aurait pas aimé ça. Mais on ne pouvait pas discuter avec M'man --- ce qu'elle ne voulait pas entendre, elle ne l'entendait tout simplement pas.

Mais c'était la première fois qu'elle l'amenait à la maison. Max sentit une colère sourde monter en lui.

\og Dépêche-toi, Maxie~! \fg{} cria M'man. \og Ne reste pas planté là comme un idiot. \fg{}

Max avança à contrecœur et les rejoignit. M'man dit~: \og Maxie, serre la main de ton nouveau père \fg{}, puis prit un air espiègle, comme si elle avait dit quelque chose de spirituel.

Max la fixa, bouche bée. Montgomery sourit largement et tendit la main. \og Eh oui, Max, tu t'appelles Max Montgomery maintenant --- je suis ton nouveau papa. Mais tu peux m'appeler Monty. \fg{}

Max regarda la main, la prit brièvement. \og Mon nom est Jones \fg{}, dit-il d'un ton neutre.

\og Maxie~! \fg{} protesta M'man.

Montgomery rit jovialemen.t \og Ne le brusque pas, Nellie mon amour. Laisse Max s'y habituer. Vivre et laisser vivre, c'est ma devise. \fg{} Il se tourna vers sa femme. \og Une seconde, je vais chercher les bagages. \fg{}

D'une des sacoches du monocycle, il sortit un tas de vêtements froissés~; de l'autre, deux bouteilles plates d'un demi-litre. Voyant Max l'observer, il fit un clin d'œil et dit~: \og Un toast pour la mariée. \fg{}

Sa mariée se tenait près de la porte~; il s'apprêtait à passer devant elle quand elle protesta~: \og Mais Monty chéri, tu ne vas pas\ldots \fg{}

Montgomery s'arrêta. \og Oh. Je n'ai pas beaucoup d'expérience dans ces choses-là. Bien sûr. \fg{} Il se tourna vers Max --- \og Tiens, prends les bagages \fg{} --- et lui fourra les bouteilles et les vêtements dans les bras. Puis il la souleva dans ses bras, grognant un peu, lui fit franchir le seuil, la posa et l'embrassa tandis qu'elle poussait des petits cris et rougissait.

Max les suivit en silence, posa les affaires sur la table et se tourna vers le poêle. Il était froid~; il ne l'avait pas utilisé depuis le petit-déjeuner. Il y avait une cuisinière électrique mais elle avait grillé avant la mort de son père et il n'y avait jamais eu d'argent pour la réparer. Il sortit son couteau de poche, fit des copeaux, ajouta du petit bois et l'alluma avec une Flamme-Perpétuelle. Quand le feu prit, il sortit chercher un seau d'eau.

Quand il revint, Montgomery demanda~: \og Je me demandais où tu étais passé. Ce taudis n'a même pas l'eau courante~? \fg{}

\og Non. \fg{} Max posa le seau, puis ajouta quelques bûches dans le feu.

Sa M'man dit~: \og Maxie, tu aurais dû préparer le dîner. \fg{}

Montgomery intervint aimablement~: \og Voyons, ma chère, il ne savait pas que nous arrivions. Et ça laisse le temps pour un toast. \fg{}

Max leur tournait le dos, concentrant toute son attention sur les tranches de lard. Le changement était si bouleversant qu'il n'avait pas eu le temps de l'assimiler. Montgomery l'appela~: \og Hé, fiston~! Bois à la santé de la mariée. \fg{}

\og Je dois préparer le dîner. \fg{}

\og N'importe quoi~! Voilà ton verre. Dépêche-toi. \fg{}

Montgomery avait versé un doigt de liquide ambré dans le verre~; le sien était à moitié plein et celui de sa mariée au moins au tiers. Max l'accepta et alla au seau, le dilua avec une louche d'eau.

\og Tu vas le gâcher. \fg{}

\og Je n'ai pas l'habitude. \fg{}

\og Bon, tant pis. À la mariée rougissante --- et à notre famille heureuse~! Cul sec~! \fg{}

Max prit une gorgée prudente et reposa le verre. Cela avait pour lui le goût du tonique amer que l'infirmière du district lui avait donné un printemps. Il retourna à son travail, pour être de nouveau interrompu.

\og Hé, tu ne l'as pas fini. \fg{}

\og Écoute, je dois cuisiner. Tu ne veux pas que je fasse brûler le dîner, non~? \fg{}

Montgomery haussa les épaules. \og Bah --- d'autant plus pour nous. On se servira du tien pour faire passer. Fiston, à ton âge, je pouvais vider un verre cul sec et ensuite faire le poirier. \fg{}

Max avait prévu de dîner de lard et de biscuits réchauffés, mais il ne restait qu'une demi-fournée de biscuits. Il fit des œufs brouillés dans la graisse du lard, prépara du café, et s'en tint là. Quand ils s'assirent, Montgomery regarda le repas et déclara~: \og Ma chère, à partir de demain, j'attends de toi que tu sois à la hauteur de ce que tu m'as dit sur ta cuisine. Ton garçon n'est pas très doué. \fg{}

Néanmoins, il mangea copieusement. Max décida de ne pas lui dire qu'il cuisinait mieux que M'man --- il le découvrirait bien assez tôt.

Bientôt Montgomery se renversa en arrière et s'essuya la bouche, puis se versa du café et alluma un cigare. M'man demanda~: \og Maxie, chéri, qu'y a-t-il pour le dessert~? \fg{}

\og Le dessert~? Eh bien --- il y a cette glace dans le congélateur, qui reste du Jour de l'Union Solaire. \fg{}

Elle eut l'air contrarié. \og Oh là là~! J'ai bien peur qu'elle n'y soit plus. \fg{}

\og Hein~? \fg{}

\og Eh bien, je crois que je l'ai un peu mangée un après-midi quand tu étais dans le champ sud. Il faisait terriblement chaud. \fg{}

Max ne dit rien, il n'était pas surpris. Mais elle ne pouvait pas s'en tenir là. \og Tu n'as pas préparé de dessert, Max~? Mais c'est une occasion \textit{spéciale}. \fg{}

Montgomery ôta le cigare de sa bouche. \og Laisse tomber, ma chère \fg{}, dit-il gentiment. \og Je ne suis pas très sucré, je suis un homme à viande et pommes de terre --- ça cale. Parlons de choses plus agréables. \fg{} Il se tourna vers Max. \og Max, qu'est-ce que tu sais faire à part fermier~? \fg{}

Max fut surpris. \og Hein~? Je n'ai jamais rien fait d'autre. Pourquoi~? \fg{}

Montgomery tapota la cendre de son cigare dans son assiette. \og Parce que tu as fini de fermier. \fg{}

Pour la deuxième fois en deux heures, Max eut plus de changements qu'il ne pouvait en absorber. \og Pourquoi~? Qu'est-ce que tu veux dire~? \fg{}

\og Parce qu'on a vendu la ferme. \fg{}

Max se sentit comme si on lui avait retiré le tapis sous les pieds. Mais il pouvait voir sur le visage de M'man que c'était vrai. Elle avait l'air qu'elle avait toujours quand elle lui en avait fait voir --- triomphante et légèrement inquiète.

\og Papa n'aurait pas aimé ça \fg{}, lui dit-il durement. \og Cette terre appartient à notre famille depuis quatre cents ans. \fg{}

\og Voyons, Maxie~! Je t'ai dit je ne sais combien de fois que je n'étais pas faite pour la ferme. J'ai grandi en ville. \fg{}

\og Le Carrefour de Clyde~! Belle ville~! \fg{}

\og Ce n'était pas une ferme. Et j'étais juste une jeune fille quand ton père m'a amenée ici --- tu étais déjà un grand garçon. J'ai encore ma vie devant moi. Je ne peux pas la vivre enterrée dans une ferme. \fg{}

Max éleva la voix. \og Mais tu avais promis à Papa que tu\ldots \fg{}

\og Ça suffit \fg{}, dit fermement Montgomery. \og Et reste poli quand tu parles à ta mère --- et à moi. \fg{}

Max se tut.

\og La terre est vendue, un point c'est tout. Combien tu estimes que vaut ce lopin~? \fg{}

\og Je n'y ai jamais réfléchi. \fg{}

\og Quoi que tu aies pensé, j'en ai tiré plus. \fg{} Il fit un clin d'œil à Max. \og Oh que oui~! C'était un jour de chance pour ta mère et toi quand elle a jeté son dévolu sur moi. Je suis un homme qui a l'oreille collée au sol. Je savais pourquoi un agent passait dans le coin racheter ces propriétés épuisées et sans valeur. Je\ldots \fg{}

\og J'utilise les engrais du gouvernement. \fg{}

\og Sans valeur j'ai dit et sans valeur je maintiens. Pour l'agriculture, du moins. \fg{} Il se passa le doigt le long du nez, prit un air rusé, et expliqua.

Il semblait qu'un grand projet énergétique gouvernemental était en cours pour lequel cette zone avait été sélectionnée --- Montgomery restait mystérieux à ce sujet, d'où Max conclut qu'il n'en savait pas beaucoup. Un syndicat achetait discrètement des terrains en anticipation de l'achat gouvernemental.

\og Alors on leur a fait cracher cinq fois ce qu'ils comptaient payer. Pas mal, hein~? \fg{}

M'man ajouta~: \og Tu vois, Maxie~? Si ton père avait su qu'un jour on obtiendrait\ldots \fg{}

\og Silence, Nellie~! \fg{}

\og Mais j'allais juste lui dire combien\ldots \fg{}

\og J'ai dit \og silence \fg{}~! \fg{}

Elle se tut. Montgomery repoussa sa chaise, mit son cigare dans sa bouche et se leva.

Max mit de l'eau à chauffer pour la vaisselle, racla les assiettes et porta les restes aux poules. Il resta dehors un bon moment, regardant les étoiles et essayant de réfléchir. L'idée d'avoir Biff Montgomery dans la famille le secouait jusqu'aux os. Il se demandait quels droits avait exactement un beau-père, ou plutôt un beau-beau-père, un homme qui avait épousé sa belle-mère. Il n'en savait rien.

Bientôt il décida qu'il devait rentrer, aussi peu qu'il en eût envie. Il trouva Montgomery debout devant l'étagère qu'il avait construite au-dessus du récepteur stéréo~; l'homme fouillait dans les livres et en avait empilé plusieurs sur le récepteur. Il se retourna.

\og Te revoilà~? Reste là, je veux que tu me parles du bétail. \fg{}

M'man apparut dans l'embrasure de la porte. \og Chéri \fg{}, dit-elle à Montgomery, \og ça ne peut pas attendre demain matin~? \fg{}

\og Ne sois pas pressée, ma chère \fg{}, répondit-il. \og Ce type du commissaire-priseur sera là de bonne heure. Je dois avoir l'inventaire prêt. \fg{} Il continua à descendre des livres. \og Dis donc, ce sont de jolies choses. \fg{} Il tenait dans ses mains une demi-douzaine de volumes, imprimés sur le plus fin papier et reliés en plastique souple. \og Je me demande combien ils valent~? Nellie, passe-moi mes lunettes. \fg{}

Max s'avança précipitamment, tendit la main vers les livres. \og Ils sont à moi~! \fg{}

\og Hein~? \fg{} Montgomery lui jeta un coup d'œil, puis leva les livres en l'air. \og Tu es trop jeune pour posséder quoi que ce soit. Non, tout y passe. Table rase et nouveau départ. \fg{}

\og Ils sont à moi~! Mon oncle me les a donnés. \fg{} Il fit appel à sa mère. \og Dis-lui, M'man. \fg{}

Montgomery dit calmement~: \og Oui, Nellie, mets ce jeune au parfum --- avant que je doive le corriger. \fg{}

Nellie avait l'air inquiet. \og Eh bien, je ne sais pas vraiment. Ils appartenaient à Chet. \fg{}

\og Et Chet était ton frère~? Alors tu es l'héritière de Chet, pas ce morveux. \fg{}

\og Ce n'était pas son frère, c'était son beau-frère~! \fg{}

\og Ah bon~? Peu importe. Ton père était l'héritier de ton oncle, donc, et ta mère est l'héritière de ton père. Pas toi, tu es mineur. C'est la loi, fiston. Désolé. \fg{} Il reposa les livres sur l'étagère mais resta planté devant.

Max sentit sa lèvre supérieure droite commencer à tressaillir de façon incontrôlable~; il savait qu'il ne pourrait pas parler de manière cohérente. Ses yeux se remplirent de larmes de rage au point qu'il n'y voyait presque plus.

\og Tu\ldots tu n'es qu'un \textit{voleur}~! \fg{}

Nellie poussa un cri. \og Max~! \fg{}

Le visage de Montgomery devint froidement malveillant. \og Là, tu es allé trop loin. J'ai bien peur que tu aies mérité un coup de ceinture. \fg{} Ses doigts commencèrent à déboucler sa lourde ceinture.

Max recula d'un pas. Montgomery détacha la ceinture et avança d'un pas.

Nellie couina~: \og Monty~! \textit{S'il te plaît}~! \fg{}

\og Ne te mêle pas de ça, Nellie. \fg{} À Max, il dit~: \og Autant régler une bonne fois pour toutes qui est le chef ici. Excuse-toi~! \fg{}

Max ne répondit pas. Montgomery répéta~: \og Excuse-toi, et on n'en parlera plus. \fg{} Il fit claquer la ceinture comme un chat fouettant l'air de sa queue.

Max recula encore d'un pas~; Montgomery s'avança et l'attrapa. Max esquiva et s'enfuit par la porte ouverte dans l'obscurité.

Il ne s'arrêta que lorsqu'il fut certain que Montgomery ne le suivait pas. Puis il reprit son souffle, toujours furieux. Il regrettait presque que Montgomery ne l'ait pas poursuivi~; il ne pensait pas que quiconque puisse rivaliser avec lui sur son propre terrain dans le noir. Il savait où était le tas de bois~; Montgomery non. Il savait où était la bauge aux cochons. Oui, il savait où était le puits --- même \textit{ça}.

Il lui fallut longtemps avant de se calmer suffisamment pour penser rationnellement. Quand il y parvint, il fut content que cela se soit terminé si facilement. Montgomery pesait bien plus lourd que lui et avait la réputation d'être un sale type dans une bagarre.

Si c'était \textit{bien} terminé, se corrigea-t-il. Il se demanda si Montgomery déciderait d'oublier d'ici le matin. La lumière était encore allumée dans le salon~; il se réfugia dans la grange et attendit, assis par terre, le dos contre les planches.

Au bout d'un moment, il se sentit terriblement fatigué. Il envisagea de dormir dans la grange mais il n'y avait pas d'endroit convenable pour s'allonger, même si la vieille mule était morte. Au lieu de cela, il se leva et regarda la maison. La lumière était éteinte dans le salon, mais il pouvait voir de la lumière dans la chambre~; ils étaient sûrement encore éveillés.

Quelqu'un avait fermé la porte d'entrée après sa fuite~; elle ne se verrouillait pas, donc il n'y avait pas de difficulté à entrer, mais il avait peur que Montgomery l'entende. Sa propre chambre était un appentis ajouté du côté cuisine de la pièce principale, en face de la chambre, mais elle n'avait pas de porte extérieure. Peu importe --- il avait résolu ce problème quand il avait grandi assez pour vouloir entrer et sortir la nuit sans consulter ses aînés.

Il fit le tour de la maison, trouva le tréteau, le plaça sous sa fenêtre, monta dessus et dégagea le clou qui retenait la fenêtre. Un instant plus tard, il descendait silencieusement dans sa propre chambre.

La porte donnant sur le reste de la maison était fermée mais il décida de ne pas risquer d'allumer la lumière~; Montgomery pourrait décider de venir dans le salon et voir un rai de lumière sous sa porte. Il se déshabilla silencieusement et se glissa dans son lit de camp.

Le sommeil ne venait pas. À un moment, il commença à ressentir cette douce somnolence, puis un petit bruit le fit se réveiller en sursaut, raide comme un piquet. Probablement juste une souris --- mais pendant un instant, il avait cru que Montgomery se tenait au-dessus de son lit.

Le cœur battant, il s'assit au bord de son lit, toujours nu. Bientôt il fit face au problème de ce qu'il allait faire --- pas seulement pour l'heure suivante, pas seulement demain matin, mais le surlendemain matin et tous les matins après cela.

Montgomery seul ne présentait pas de problème~; il ne resterait pas volontairement dans le même comté que cet homme. Mais qu'en était-il de M'man~? Son père lui avait dit, quand il avait su qu'il allait mourir~: \og Prends soin de ta mère, fiston. \fg{}

Eh bien, il l'avait fait. Il avait fait une récolte chaque année --- de la nourriture dans la maison et un peu d'argent, même si les temps avaient été durs. Quand la mule était morte, il s'était débrouillé, empruntant l'attelage de McAllister et le remboursant en travail. Mais Papa avait-il voulu dire qu'il devait prendre soin de sa belle-mère même si elle se remariait~? Il ne lui était jamais venu à l'esprit de considérer la question. Papa lui avait dit de veiller sur elle et il l'avait fait, même si cela avait mis fin à l'école et ne semblait pas avoir de fin. Mais elle n'était plus Mme Jones mais Mme Montgomery. Papa avait-il voulu qu'il subvienne aux besoins de Mme Montgomery~?

Bien sûr que non~! Quand une femme se mariait, son mari subvenait à ses besoins. Tout le monde savait ça. Et Papa ne s'attendrait pas à ce qu'il supporte Montgomery.

Il se leva, sa décision soudainement prise. La seule question était ce qu'il allait emporter. Il y avait peu à prendre. Tâtonnant dans le noir, il trouva le sac à dos qu'il utilisait pour ses randonnées de chasse et y fourra son autre chemise et ses chaussettes. Il ajouta la règle à calcul d'astrogation circulaire de l'oncle Chet et le morceau de verre volcanique que son oncle lui avait rapporté de la Lune. Sa carte d'identité de citoyen, sa brosse à dents, et le rasoir de son père --- non qu'il en eût besoin très \textit{souvent} --- complétaient à peu près le butin.

Il y avait une planche branlante derrière son lit. Il la chercha à tâtons, la retira et fouilla entre les montants --- ne trouva rien. Il avait mis de côté un peu d'argent de temps en temps pour les mauvais jours, car M'man ne pouvait ou ne voulait pas économiser. Mais apparemment elle l'avait trouvé lors d'une de ses fouilles. Eh bien, il devait quand même partir~; cela rendait juste les choses un peu plus difficiles.

Il prit une grande inspiration. Il y avait quelque chose qu'il \textit{devait} récupérer\ldots Les livres de l'oncle Chet\ldots et ils étaient encore (vraisemblablement) sur l'étagère contre le mur commun avec la chambre. Mais il \textit{devait} les récupérer, même au risque de rencontrer Montgomery.

Avec précaution, très lentement, il ouvrit la porte donnant sur le salon, resta là, la sueur ruisselant sur lui. Il y avait encore un rai de lumière sous la porte de la chambre et il hésita, presque incapable de se forcer à continuer. Il entendit Montgomery marmonner quelque chose et M'man glousser.

Alors que ses yeux s'adaptaient, il put distinguer, grâce à la faible lumière filtrant sous la porte de la chambre, quelque chose d'empilé devant la porte d'entrée. C'était un piège de casseroles et de marmites, sûr de faire un vacarme épouvantable si la porte était ouverte. Apparemment Montgomery avait compté sur son retour et s'attendait à être prêt à s'occuper de lui. Il fut très content d'être entré par la fenêtre.

Inutile de remettre à plus tard --- il traversa le sol à pas feutrés, attentif à la planche qui grinçait près de la table. Il ne pouvait pas voir mais il pouvait sentir, et les volumes étaient connus de ses doigts. Avec précaution, il les fit glisser, veillant à ne pas faire tomber les autres.

Il était revenu jusqu'à sa porte quand il se souvint du livre de la bibliothèque. Il s'arrêta, pris d'une panique soudaine. Il ne pouvait pas y retourner. Ils pourraient l'entendre cette fois --- ou Montgomery pourrait se lever pour boire un verre d'eau ou quelque chose.

Mais dans son horizon limité, le vol d'un livre de bibliothèque publique --- ou le fait de ne pas le rendre, ce qui revenait au même --- était, sinon un péché mortel, du moins en haut de la liste des crimes honteux. Il resta là, transpirant et y réfléchissant.

Puis il y retourna, tout le long chemin, contournant la planche qui grinçait et marchant tragiquement sur une autre qu'il n'avait pas mémorisée. Il se figea après l'avoir heurtée, mais apparemment cela n'avait pas alerté le couple dans la pièce voisine. Enfin il était penché sur le récepteur SV et tâtonnait l'étagère.

Montgomery, en fouillant les livres, avait changé leur disposition. L'un après l'autre, il dut les prendre et essayer de l'identifier au toucher, ouvrant chacun et cherchant les perforations sur la page de titre. C'était le quatrième qu'il manipula.

Il regagna sa chambre en se pressant lentement, terriblement anxieux mais n'osant pas bouger vite. Là enfin, il se mit à trembler et dut attendre que cela passe. Il n'essaya pas de fermer sa porte mais s'habilla dans le noir. Quelques instants plus tard, il se glissait par sa fenêtre, trouvait le tréteau du bout du pied et descendait silencieusement au sol. Ses chaussures étaient fourrées au-dessus des livres dans son sac à dos~; il décida de les y laisser jusqu'à ce qu'il soit bien loin de la maison, plutôt que de risquer le bruit qu'il pourrait faire avec ses pieds chaussés.

Il fit un large détour autour de la maison et regarda en arrière. La lumière de la chambre était encore allumée~; il commença à descendre vers la route quand il remarqua le monocycle de Montgomery.

Il s'arrêta. S'il continuait, il arriverait à la route que le bus empruntait. Qu'il tourne à droite ou à gauche, Montgomery aurait une chance sur deux de le rattraper avec le monocycle. N'ayant pas d'argent, il dépendait de ses propres jambes pour mettre de la distance derrière lui~; il ne pouvait pas prendre le bus.

Bah~! Montgomery n'essaierait pas de le ramener. Il dirait bon débarras et l'oublierait~!

Mais cette pensée le tracassait. Et si M'man l'y poussait~? Et si Montgomery n'oubliait pas une insulte et ferait tout pour \og se venger \fg{}~?

Il fit demi-tour, contournant toujours la maison de loin, et coupa à travers les pentes vers l'emprise de la C.S.\&E.
