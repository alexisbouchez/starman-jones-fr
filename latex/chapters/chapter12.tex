\chapter{Halcyon}

La nomination probatoire fut enregistrée plus tard ce même jour. Le Capitaine le fit venir, lui fit prêter serment, puis le félicita et l'appela \og Monsieur \fg{} Jones. La cérémonie était simple, sans autre spectateur qu'Hendrix et le secrétaire du Capitaine. Les banalités qui accompagnèrent le changement furent, pendant un moment, plus surprenantes pour Max que la promotion elle-même. Elles commencèrent immédiatement. \og Vous feriez mieux de prendre le reste de la journée pour vous installer, M. Jones, \fg{} dit le Capitaine, clignant vaguement des yeux. \og D'accord, Doc~? \fg{}

\og Certainement, monsieur. \fg{}

\og Bien. Bennett, voulez-vous demander à Dumont d'entrer~? \fg{}

Le Chef Steward des Passagers ne fut pas du tout surpris de trouver le récent aide-steward de troisième classe devenu officier du vaisseau. À la question du Capitaine il répondit~: \og Je comptais installer M. Jones dans la cabine B-014, monsieur. Est-ce satisfaisant~? \fg{}

\og Sans doute, sans doute. \fg{}

\og Je vais faire transporter ses bagages immédiatement. \fg{}

\og Bien. Allez donc avec Dumont, M. Jones. Non, attendez un instant. Il faut vous trouver une casquette. \fg{} Le Capitaine alla à son armoire, fouilla un peu. \og J'en avais une qui ferait l'affaire quelque part ici. \fg{}

Hendrix se tenait les mains derrière le dos. \og J'en ai apporté une, Capitaine. M. Jones et moi faisons la même taille de tête, je crois. \fg{}

\og Bien. Quoique peut-être que sa tête a un peu enflé ces dernières minutes. Hein~? \fg{}

Hendrix sourit férocement. \og Si c'est le cas, je la lui dégonflerai. \fg{} Il tendit la casquette à Max. Le large galon d'or et le soleil rayonnant de l'Astrogateur avaient été retirés~; à leur place, il y avait un galon étroit avec un minuscule soleil rayonnant entouré du cercle distinctif de l'apprenti. Max pensa que ce devait être d'anciens insignes gardés pour des raisons sentimentales par Hendrix lui-même. Il eut la gorge serrée en marmonnant ses remerciements, puis suivit Dumont hors de la cabine du Capitaine, trébuchant sur ses pieds. Quand ils atteignirent l'échelle, Dumont s'arrêta. \og Il n'est pas nécessaire de descendre au dortoir, monsieur. Si vous voulez me donner la combinaison de votre casier, nous nous occuperons de tout. \fg{}

\og Oh, bon sang, M. Dumont~! Je n'ai qu'une petite quantité d'affaires. Je peux les monter moi-même. \fg{}

Le visage de Dumont avait l'impassibilité d'un majordome. \og Si je puis me permettre une suggestion, monsieur, vous aimeriez peut-être voir votre cabine pendant que je fais s'occuper de cette affaire. \fg{}

Ce n'était pas une question~; Max l'interpréta correctement comme voulant dire~: \og \emph{Écoute, imbécile, je connais les règles et pas toi. Fais ce que je te dis avant de commettre une erreur terrible~!} \fg{} Max se laissa guider. Il n'est pas facile de faire le saut de membre d'équipage à officier tout en restant dans le même vaisseau. Dumont le savait, Max non. Que son intérêt fût paternel, ou simplement un goût pour le protocole correct -- ou les deux -- Dumont n'avait pas l'intention de permettre au tout nouveau jeune officier de descendre plus bas que le pont \og C \fg{} avant qu'il ait appris à porter sa nouvelle dignité avec grâce.

Alors Max chercha la cabine B-014. La couchette avait un vrai matelas en mousse et un dessus-de-lit. Il y avait un minuscule lavabo avec eau courante et un miroir. Il y avait une étagère à livres au-dessus de la couchette et une armoire pour ses uniformes. Il y avait même une petite tablette-bureau qui s'abaissait pour sa commodité. Il y avait un téléphone au mur, une sonnette par laquelle il pouvait appeler l'aide-steward de quart~! Il y avait une chaise mobile rien qu'à lui, une corbeille à papier, et -- oui~! -- un petit tapis sur le pont. Et mieux que tout, il y avait une porte avec une serrure. Le fait que la pièce entière fût à peu près aussi grande qu'une caisse de piano ne le gênait nullement.

Il était en train d'ouvrir des tiroirs et de fouiner partout quand Dumont revint. Dumont ne portait pas lui-même les maigres possessions de Max~; cette tâche était déléguée à un membre de son personnel des ponts supérieurs. L'aide-steward suivit Dumont à l'intérieur et dit~: \og Où dois-je mettre cela, monsieur~? \fg{}

Max réalisa avec un embarras soudain que l'homme qui le servait avait mangé en face de lui pendant des mois. \og Oh~! Bonjour, Jim. Pose ça sur la couchette. Merci beaucoup. \fg{}

\og Oui, monsieur. Et félicitations~! \fg{}

\og Euh, merci~! \fg{} Ils se serrèrent la main.

Dumont laissa cette cérémonie convenable durer un minimum de temps, puis dit~: \og Ce sera tout, Gregory. Tu peux retourner à l'office. \fg{} Il se tourna vers Max. \og Autre chose, monsieur~? \fg{}

\og Oh, non, tout est parfait. \fg{}

\og Puis-je suggérer que vous ne voudrez probablement pas coudre vous-même les insignes sur ces uniformes~? À moins que vous ne soyez plus habile avec une aiguille que moi, \fg{} ajouta Dumont avec juste le petit rire qu'il fallait.

\og Eh bien, je suppose que je pourrais. \fg{}

\og Mme Dumont est habile avec une aiguille, s'occupant des passagères comme elle le fait. Si je prenais celui-ci~? Il peut être prêt et repassé à temps pour le dîner. \fg{}

Max fut heureux de le laisser faire. Il fut soudain épouvanté par une idée terrifiante -- il allait devoir manger au Salon Bifrost~! Mais il y eut d'autres perturbations avant le dîner. Il était en train d'achever la petite tâche de ranger ses affaires quand on frappa à la porte, suivi immédiatement par quelqu'un qui entrait. Max se retrouva nez à nez avec M. Simes.

Simes regarda la casquette sur sa tête et rit. \og Enlevez ce truc avant de vous user les oreilles. \fg{}

Max ne le fit pas. Il dit~: \og Vous me vouliez, monsieur~? \fg{}

\og Oui. Juste assez longtemps, P'tit Malin, pour vous donner un mot de conseil. \fg{}

\og Oui~? \fg{}

Simes se tapota la poitrine. \og Juste ceci. Il n'y a qu'un seul assistant astrogateur dans ce vaisseau -- et c'est moi. Souvenez-vous de ça. Je le serai encore longtemps après que vous aurez été rétrogradé à balayer derrière les vaches. Ce qui est là où vous appartenez. \fg{}

Max sentit une rougeur monter le long de son cou et brûler ses joues. \og Pourquoi, \fg{} demanda-t-il, \og si vous pensez cela, n'avez-vous pas mis votre veto à ma nomination~? \fg{}

Simes rit de nouveau. \og Est-ce que j'ai l'air d'un imbécile~? Le Capitaine dit oui, l'Astrogateur dit oui -- je devrais me mouiller~? C'est plus facile d'attendre et de vous laisser vous mouiller vous-même -- ce que vous ferez. Je voulais juste vous faire savoir qu'un petit bout de galon doré ne signifie rien. Vous êtes toujours mon subalterne de loin. Ne l'oubliez pas. \fg{}

Max serra la mâchoire et ne répondit pas. Simes poursuivit~: \og Eh bien~? \fg{}

\og "Eh bien" quoi~? \fg{}

\og Je viens de vous donner un ordre. \fg{}

\og Oh. Bien reçu, M. Simes. Je ne l'oublierai pas. Je ne l'oublierai certainement pas. \fg{}

Simes le regarda d'un air perçant, dit~: \og Veillez-y, \fg{} et partit.

Max était encore face à sa porte, serrant les poings, quand Gregory frappa à la porte. \og Le dîner, monsieur. Cinq minutes. \fg{}

\bigskip

Max retarda autant qu'il put, souhaitant ardemment pouvoir se glisser jusqu'au pont Easy et prendre sa place habituelle dans le confort chaleureux, bruyant et détendu du mess de l'équipage. Il hésita dans l'embrasure de la porte du salon, paralysé par le trac. La belle salle flamboyait de lumière et paraissait inconnue~; il n'y était jamais entré sauf tôt le matin, pour changer le bac à sable situé au bout du passage de l'office -- moments où seules les veilleuses étaient allumées. Il arrivait à peine à temps~; certaines des dames étaient assises mais le Capitaine était encore debout. Max réalisa qu'il devrait être près de sa chaise, prêt à s'asseoir quand le Capitaine le ferait -- ou dès que les dames seraient assises, rectifia-t-il -- mais où devait-il aller~? Il était encore en train de s'agiter quand il entendit son nom crié.

\og Max~! \fg{} Ellie accourut et lui jeta les bras autour du cou. \og Max~! Je viens d'apprendre. Je trouve ça \emph{merveilleux}~! \fg{} Elle le regarda, les yeux brillants, puis l'embrassa sur les deux joues.

Max rougit jusqu'aux oreilles. Il avait l'impression que tous les regards étaient tournés vers lui -- et il avait raison. Pour ajouter à son embarras, Ellie portait une tenue de soirée formelle du grand style d'Hespérus, qui non seulement la faisait paraître plus âgée et beaucoup plus féminine, mais aussi choquait ses standards puritains de montagnard. Elle le lâcha, ce qui était bien mais le laissait en danger de s'effondrer sur les genoux. Elle commença à babiller quelque chose, Max ne savait pas quoi, quand le Chef Steward Dumont apparut à son coude. \og Le Capitaine attend, Mademoiselle, \fg{} dit-il fermement.

\og Le Capitaine peut bien attendre~! Oh, bon -- à plus tard, Max. \fg{} Elle se dirigea vers la table du Capitaine. Dumont toucha la manche de Max et murmura~: \og Par ici, monsieur. \fg{}

Sa place était en bout de la table du Chef Mécanicien. Max connaissait M. Compagnon de vue mais ne lui avait jamais parlé. Le Chef leva les yeux et dit~: \og Bonsoir, M. Jones. Content de vous avoir parmi nous. Mesdames et messieurs, notre nouvel officier d'astrogation, M. Jones. À votre droite, M. Jones, se trouve Mme Daigler. M. Daigler à sa droite, puis -- \fg{} et ainsi de suite autour de la table~: le Dr et Mme Weberbauer et leur fille Rebecca, M. et Mme Scott, un M. Arthur, le Senhor et la Senhora Vargas.

Mme Daigler trouvait cela \emph{adorable}, sa promotion. Et si \emph{agréable} d'avoir plus de jeunes gens à table. Elle était beaucoup plus âgée que Max mais assez jeune pour être belle et en avoir conscience. Elle portait plus de bijoux que Max n'en avait jamais vu et ses cheveux étaient laqués en une structure d'un pied de haut et parsemée de perles. Elle était aussi parfaitement finie et aussi coûteuse qu'une machine de précision et elle mettait Max mal à l'aise.

Mais il n'était pas encore aussi mal à l'aise qu'il pouvait l'être. Mme Daigler sortit un soupçon de mouchoir de son corsage, l'humecta et dit~: \og Ne bougez pas, M. Jones. \fg{} Elle lui frotta la joue. \og Tournez la tête. \fg{} Rougissant, Max obéit. \og Voilà, c'est mieux, \fg{} annonça Mme Daigler. \og Maman a arrangé ça. \fg{} Elle se détourna et dit~: \og Ne pensez-vous pas, M. Compagnon, que la science, avec toutes les choses merveilleuses qu'ils font de nos jours, pourrait découvrir un rouge à lèvres qui ne partirait pas~? \fg{}

\og Arrête, Maggie, \fg{} interrompit son mari. \og Ne faites pas attention, M. Jones. Elle a une veine de sadisme large comme elle. \fg{}

\og George, tu vas me le payer. Eh bien, Chef~? \fg{}

Le Chef Mécanicien tapota ses lèvres avec du lin immaculé. \og Je pense que ça a déjà dû être inventé, mais il n'y avait pas de marché. Les femmes aiment marquer les hommes, même temporairement. \fg{}

\og Oh, sottises~! \fg{}

\og C'est un monde de femmes, madame. \fg{}

Elle se tourna vers Max. \og Eldreth est une chérie, n'est-ce pas~? Je suppose que vous la connaissiez "côté terre"~? -- comme dit M. Compagnon. \fg{}

\og Non, madame. \fg{}

\og Alors comment~? Je veux dire, après tout, il n'y a pas beaucoup d'occasions. Ou si~? \fg{}

\og Maggie, arrête de l'embêter. Laisse cet homme manger son dîner. \fg{}

Mme Weberbauer de l'autre côté était aussi facile et maternelle que Mme Daigler était difficile. Sous sa présence apaisante, Max réussit à commencer à manger. Puis il remarqua que la façon dont il tenait sa fourchette n'était pas celle des autres, essaya de changer, fit un gâchis, prit conscience de ses ongles mal soignés, et voulut ramper sous la table. Il mangea environ trois cents calories, principalement du pain et du beurre.

À la fin du repas, Mme Daigler lui accorda de nouveau son attention, bien qu'elle s'adressât au Chef Mécanicien. \og M. Compagnon, n'est-il pas coutume de porter un toast à une promotion~? \fg{}

\og Si, \fg{} concéda le Chef. \og Mais il doit payer. C'est une obligation. \fg{}

Max se retrouva à signer une note présentée par Dumont. Le prix le fit cligner des yeux -- son premier voyage pouvait être un succès professionnel, mais jusqu'ici c'était un désastre financier. Du champagne, glacé dans un seau brillant, accompagnait la note et Dumont coupa les fils et tira le bouchon avec panache.

Le Chef Mécanicien se leva. \og Mesdames et messieurs -- je vous propose l'Astrogateur Jones. Qu'il ne déplace jamais une virgule~! \fg{}

\og Santé~! \fg{} -- \og Bravo~! \fg{} -- \og Un discours, un discours~! \fg{}

Max se mit debout en trébuchant et marmonna~: \og Merci. \fg{}

\bigskip

Son premier quart était à huit heures le lendemain matin. Il prit son petit-déjeuner seul et réfléchit avec bonheur qu'en tant qu'officier de quart, il mangerait généralement soit avant soit après les passagers. Il était dans la salle de contrôle une bonne vingtaine de minutes en avance. Kelly leva les yeux et dit~: \og Bonjour, monsieur. \fg{}

Max déglutit. \og Euh -- bonjour, Chef~! \fg{} Il surprit Smythe en train de sourire derrière l'ordinateur, détourna précipitamment les yeux. \og Du café frais, M. Jones. Vous en prendrez une tasse~? \fg{}

Max laissa Kelly lui servir~; pendant qu'ils buvaient, Kelly passa tranquillement en revue les détails -- programme d'accélération, position et vecteur, unités de puissance en service, visées prises, pas d'ordres spéciaux, etc. Noguchi releva Smythe, et peu avant l'heure, le Dr Hendrix apparut. \og Bonjour, monsieur. \fg{}

\og Bonjour, Docteur. \fg{}

\og 'jour. \fg{} Hendrix accepta du café, se tourna vers Max. \og Avez-vous relevé l'officier de quart~? \fg{}

\og Euh, non, monsieur. \fg{}

\og Alors faites-le. Il manque moins d'une minute avant huit heures. \fg{}

Max se tourna vers Kelly et salua d'une main tremblante. \og Je vous relève, monsieur. \fg{}

\og Très bien, monsieur. \fg{} Kelly descendit aussitôt.

Le Dr Hendrix s'assit, sortit un livre et commença à lire. Max réalisa avec un sentiment glacé qu'on l'avait poussé à l'eau, pour nager ou non. Il prit une profonde inspiration et alla vers Noguchi. \og Noggy, préparons les plaques pour les visées du milieu de quart. \fg{}

Noguchi jeta un coup d'œil au chronomètre. \og Comme vous dites, monsieur. \fg{}

\og Eh bien... je suppose que c'est tôt. Prenons quelques Doppler. \fg{}

\og Bien, monsieur. \fg{} Noguchi sortit de la selle où il paressait.

Max dit à voix basse~: \og Écoute, Noggy, tu n'as pas besoin de me dire "monsieur". \fg{}

Noguchi répondit tout aussi doucement. \og Kelly n'aimerait pas si je ne le faisais pas. Vaut mieux laisser courir. \fg{}

\og Oh. \fg{} Max fronça les sourcils. \og Noggy~? Comment le reste de la bande du Trou aux Soucis le prend~? \fg{}

Noguchi ne fit pas semblant de ne pas comprendre. Il répondit~: \og Ben, ils sont tous pour toi, si tu y arrives. \fg{}

\og Tu es sûr~? \fg{}

\og Certain. Du moment que tu n'essaies pas de jouer les gros bonnets comme -- eh bien, comme certains que je pourrais nommer. \fg{} Le calculateur ajouta~: \og Peut-être que Kovak n'applaudit pas exactement. Il a eu un quart à lui, tu sais -- pour la première fois. \fg{}

\og Il est fâché~? \fg{}

\og Pas exactement. Il ne pouvait pas s'attendre à le garder longtemps de toute façon, pas avec une transition qui approche. Il ne se donnera pas de mal pour te créer des ennuis, il sera correct. \fg{}

Max prit mentalement note de voir ce qu'il pourrait faire pour rallier Kovak à son côté. Les deux hommes se mirent au dopplerscope, prirent des lectures sur des étoiles en avant du vecteur, vérifièrent ce qu'ils trouvaient par spectrostellographe, et comparèrent les deux avec les plaques standard du coffre des cartes. Au début Max dut se rappeler qu'il était aux commandes~; puis il s'intéressa tellement aux détails méticuleux des mesures qu'il n'était plus gêné.

Enfin Noguchi lui toucha la manche. \og Bientôt dix heures, monsieur. Je ferais mieux de m'installer. \fg{}

\og Hein~? Bien sûr, vas-y. \fg{} Il se rappela de ne pas aider Noggy~; le cartographe a ses prérogatives aussi. Mais il vérifia l'installation comme Hendrix le faisait toujours, comme Simes le faisait rarement, et comme Kelly le faisait parfois, selon qui l'avait faite.

Après avoir obtenu les nouvelles données, Max programma le problème sur papier (ayant largement le temps), puis le dicta à Noguchi à l'ordinateur. Il consulta lui-même le livre, n'ayant pas d'\og homme des chiffres \fg{} disponible. Les chiffres étaient aussi clairs dans son souvenir que jamais, mais il obéit à l'injonction d'Hendrix de ne pas se fier à sa mémoire.

Le résultat l'inquiéta. Ils n'étaient pas \og dans le sillon \fg{}. Non que l'\emph{Asgard} fût très décalé, mais la discordance était mesurable. Il vérifia ce qu'il avait fait, puis fit Noguchi résoudre le problème à nouveau, utilisant une méthode de programmation différente. Le résultat fut le même. Soupirant, il calcula la correction et commença à la porter à Hendrix pour approbation.

Mais l'Astrogateur ne faisait toujours pas attention~; il était assis à la console, lisant un roman de la bibliothèque du vaisseau. Max prit sa décision. Il alla à la console et dit~: \og Excusez-moi, monsieur. J'ai besoin d'y accéder un moment. \fg{}

Hendrix se leva sans répondre et trouva un autre siège. Max s'assit et appela la salle des machines. \og Officier de contrôle. J'ai l'intention d'augmenter la poussée à onze heures. Tenez-vous prêt pour la vérification horaire. \fg{} Hendrix devait l'avoir entendu, pensa-t-il, mais l'Astrogateur ne donna aucun signe.

Max introduisit la correction, régla le chronomètre de contrôle pour exécuter ses souhaits à onze heures plus ou moins zéro. Peu avant midi, Simes se montra. Max avait déjà écrit son propre journal de bord, basé sur le journal de Noguchi, et l'avait signé \og M. Jones \fg{}. Il avait hésité, puis ajouté \og O.C. d/Q \fg{}.

Simes alla vers le Dr Hendrix, salua et dit~: \og Prêt à vous relever, monsieur. \fg{}

Hendrix prononça son premier mot depuis huit heures. \og C'est lui qui l'a. \fg{}

Simes parut déconcerté, puis alla vers Max. \og Prêt à vous relever. \fg{}

Max récita les données de situation pendant que Simes lisait le journal et le livre des ordres. Simes l'interrompit alors qu'il énumérait encore des données mineures du vaisseau. \og Okay, je vous relève. Sortez de ma salle de contrôle, Monsieur. \fg{}

Max sortit. Le Dr Hendrix était déjà descendu. Noguchi s'était attardé au pied de l'échelle. Il croisa le regard de Max, fit un cercle avec le pouce et l'index et hocha la tête. Max lui sourit, commença à poser une question~; il voulait savoir si cette discordance était un piège, laissée intentionnellement par Kelly. Puis il décida que ce ne serait pas convenable~; il demanderait à Kelly lui-même, ou le calculerait à partir des registres. \og Merci, Noggy. \fg{}

Ce quart s'avéra typique seulement en un point~: le Dr Hendrix continua à exiger que Max soit lui-même l'officier de quart. Il ne resta plus silencieux mais talonna Max sans répit, le faisant s'exercer heure après heure, l'obligeant à prendre des visées et poser des problèmes continuellement, comme si l'\emph{Asgard} était réellement proche de la transition.

Il ne permit pas à Max de programmer sur papier mais le força à faire comme si le temps était trop court et que les données devaient immédiatement entrer dans l'ordinateur, être traitées sur-le-champ. Max suait, les commandes à distance dans chaque poing et avec Hendrix lui-même agissant comme \og homme des chiffres \fg{}. L'Astrogateur continuait à le pousser vers plus de vitesse, de vitesse, et encore plus de vitesse -- jamais au sacrifice de la précision, car toute erreur était impardonnable. Mais l'objectif était toujours une plus grande vitesse.

Une fois Max protesta. \og Monsieur, si vous me laissiez le mettre directement dans la machine, je pourrais beaucoup réduire le temps. \fg{}

Hendrix répliqua sèchement~: \og Quand vous aurez votre propre salle de contrôle, vous pourrez faire ça, si vous le jugez sage. Pour l'instant vous le ferez \emph{à ma façon}. \fg{}

De temps en temps Kelly prenait le relais comme superviseur. Le Chef Calculateur était formel, utilisant des phrases comme \og Puis-je suggérer, monsieur -- \fg{} ou \og Je pense que je ferais comme ça, monsieur. \fg{} Mais une fois il éclata~: \og Bon sang, Max~! Ne refais jamais un coup aussi stupide~! \fg{} Puis il commença à amender ses remarques. Max sourit. \og S'il vous plaît, Chef. Pendant un instant vous m'avez fait me sentir chez moi. Merci. \fg{}

Kelly eut l'air penaud. \og Je suis fatigué, je suppose. Je pourrais bien faire avec une cigarette et du café. \fg{}

Pendant qu'ils se reposaient, Max remarqua que Lundy était hors de portée de voix et dit~: \og Chef~? Vous en savez plus que je n'en apprendrai jamais. Pourquoi \emph{vous} n'avez-vous pas visé astrogateur~? Vous n'avez jamais eu l'occasion~? \fg{}

Kelly eut soudain l'air sombre. \og J'en ai eu une, une fois, \fg{} dit-il raidement. \og Maintenant je connais mes limites. \fg{}

Max se tut, très embarrassé. Par la suite Kelly revint à l'appeler Max chaque fois qu'ils étaient seuls.

\bigskip

Max ne vit pas Sam pendant plus d'une semaine après avoir déménagé au pont Baker. Même alors la rencontre fut fortuite~; il tomba sur lui devant le bureau du Commissaire. \og Sam~! \fg{}

\og Bonjour, monsieur~! \fg{} Sam se redressa dans un salut impeccable avec un large sourire sur le visage.

\og Hein~? "Bonjour, monsieur" mon œil~! Comment ça va, Sam~? \fg{}

\og Vous n'allez pas me rendre mon salut~? En ma capacité officielle je peux vous signaler, vous savez. Le Capitaine est très, très à cheval sur l'étiquette du vaisseau. \fg{}

Max fit un bruit grossier. \og Tu peux garder ce salut jusqu'à ce que tu gèles, clown. \fg{}

Sam se détendit. \og Gamin, j'avais l'intention de monter te féliciter -- mais chaque fois je découvre que tu es de quart. Tu dois vivre dans le Trou aux Soucis. \fg{}

\og Presque. Écoute, je serai libre ce soir jusqu'à minuit. Si je descendais te voir~? \fg{}

Sam secoua la tête. \og Je serai occupé. \fg{}

\og Occupé comment~? Tu attends une évasion~? Ou une émeute, peut-être~? \fg{}

Sam répondit sobrement~: \og Gamin, ne te méprends pas -- mais reste dans ton coin du vaisseau et je resterai dans le mien. Non, non, tais-toi et écoute. Je suis aussi fier que si je t'avais inventé. Mais tu ne peux pas fraterniser dans les quartiers de l'équipage, même pas avec le Chef Maître d'Armes. Pas encore. \fg{}

\og Qui le saura~? Qui s'en souciera~? \fg{}

\og Tu sais fichtre bien que Giordano adorerait dire à Kuiper que tu ne sais pas te comporter comme un officier -- et la vieille dame Kuiper le transmettrait au Commissaire. Suis mon conseil. T'ai-je déjà joué un mauvais tour~? \fg{}

Max abandonna l'affaire, bien qu'il eût très envie de bavarder avec Sam. Il avait besoin de lui dire que son faux dossier avait été percé à jour et de le consulter sur les conséquences probables. Bien sûr, réfléchit-il en retournant à sa cabine, rien ne l'empêchait de mettre à exécution son intention originale de déserter avec Sam à Nova Terra -- sauf que c'était maintenant impossible à imaginer. Il était officier.

\bigskip

Ils approchaient de la transition du milieu~; la salle de contrôle passa au quart-et-quart. Mais le Dr Hendrix ne prit toujours pas le quart~; Simes et Jones alternaient. L'Astrogateur faisait chaque quart avec Max mais exigeait qu'il fasse le travail et porte lui-même la responsabilité. Max suait et apprit que les problèmes d'entraînement et l'étude de la théorie n'étaient rien comparés au fait que ça compte quand il n'avait aucun moyen ni aucun temps de vérifier. Il fallait avoir \emph{raison}, à chaque fois -- et il y avait toujours le doute.

Quand, pendant les dernières vingt-quatre heures, l'équipe du Trou aux Soucis passa au quart continu, Max pensa que le Dr Hendrix le mettrait de côté. Mais il n'en fit rien. Simes fut mis de côté, oui, mais Max prit le siège de responsabilité, avec Hendrix penché sur lui et observant tout ce qu'il faisait, mais sans interférer. \og Grand Dieu~! \fg{} pensa Max. \og Sûrement qu'il ne va pas me laisser faire cette transition~? Je ne suis pas prêt, pas encore. Je n'arriverai jamais à suivre. \fg{}

Mais les données arrivaient trop vite pour s'inquiéter davantage~; il devait continuer à les traiter, voir les réponses et prendre des décisions. Ce ne fut que vingt minutes avant la transition qu'Hendrix le poussa de côté sans un mot et prit la relève. Max était encore en train de se remettre quand ils percèrent dans un nouveau ciel.

\bigskip

La dernière approche-et-transition avant Halcyon ressembla beaucoup à la deuxième. Il y eut deux semaines de quarts faciles, dirigés par Simes, Jones et Kovak, avec Kelly et Hendrix qui se reposaient un peu tous les deux. Max aimait ça, de quart comme hors quart. De quart il continuait à s'exercer, essayant d'atteindre la vitesse inhumaine du Dr Hendrix. Hors quart il dormait et s'amusait. Le Salon Bifrost ne le terrifiait plus. Il jouait maintenant au trois-dés avec Ellie là-bas, avec Chipsie sur son épaule, donnant des conseils. Ellie avait depuis longtemps fait les yeux doux au Capitaine Blaine et l'avait convaincu qu'un animal de compagnie si bien élevé, si propre, et en particulier si bien éduqué (elle avait dressé le chiot-araignée à dire \og Bonjour, Capitaine \fg{} chaque fois qu'il voyait Blaine) -- à tous égards si civilisé ne devrait pas être forcé de vivre dans une cage. Max avait même appris à échanger de faibles réparties avec Mme Daigler, préparant ses remarques et attendant une occasion. Ellie menaçait de lui apprendre à danser, bien qu'il réussît à la faire patienter jusqu'à la reprise du quart-et-quart avant la transition qui rendit la chose impossible. De nouveau il se retrouva poussé dans le siège de responsabilité pour la dernière partie de l'approche. Cette fois le Dr Hendrix ne le déplaça que moins de dix minutes avant la percée.

\bigskip

Pendant la descente facile vers Halcyon, la détermination d'Ellie l'emporta. Max apprit à danser. Il découvrit que ça lui plaisait. Il avait un bon sens du rythme, n'oubliait pas ses instructions, et Ellie était une brassée parfumée et agréable. \og J'ai fait tout ce que je pouvais, \fg{} annonça-t-elle enfin. \og Tu es le meilleur danseur avec deux pieds gauches que j'aie jamais rencontré. \fg{} Elle exigea qu'il danse avec Rebecca Weberbauer et avec Mme Daigler. Mme Daigler n'était pas si mal après tout, tant qu'elle gardait la bouche fermée -- et Rebecca était mignonne. Il commença à se réjouir des plaisirs d'Halcyon, cela étant la raison invoquée par Ellie pour l'instruire~; il devait être réquisitionné comme son cavalier.

Une seule chose gâcha la dernière étape~; Sam avait des ennuis. Max ne l'apprit qu'après que les ennuis eurent éclaté. Il se leva tôt pour prendre son quart et trouva Sam en train de nettoyer les ponts dans les passages silencieux des quartiers des passagers. Il était en salopette et ne portait pas d'insigne. \og Sam~! \fg{}

Sam leva les yeux. \og Oh. Salut, gamin. Parle moins fort, tu vas réveiller les gens. \fg{}

\og Mais Sam, qu'est-ce que tu fais, bon sang~? \fg{}

\og Moi~? J'ai l'air de faire la manucure à ce pont. \fg{}

\og Mais pourquoi~? \fg{}

Sam s'appuya sur son balai. \og Eh bien, gamin, c'est comme ça. Le Capitaine et moi avons eu une différence d'opinion. C'est lui qui a gagné. \fg{}

\og Tu as été rétrogradé~? \fg{}

\og Ton intuition est éblouissante. \fg{}

\og Qu'est-ce qui s'est passé~? \fg{}

\og Max, moins tu en sais, mieux c'est. Ne t'en fais pas. \emph{Sic transit gloria mundi} -- le mardi est généralement pire. \fg{}

\og Mais -- Écoute, je dois aller manger et prendre mon quart. Je viendrai te voir plus tard. \fg{}

\og Non. \fg{}

Max apprit l'histoire par Noguchi. Sam, apparemment, avait installé un casino dans une réserve vide. Il aurait pu s'en tirer indéfiniment si c'était resté une affaire de cartes et de dés, avec un pourcentage pour la maison -- la \og maison \fg{} étant le Chef Maître d'Armes. Mais Sam avait ajouté une roulette et ça avait été sa perte~; Giordano avait fini par soupçonner que la roue avait moins d'élément de hasard que de coutume dans les maisons de jeu mieux tenues -- et avait exprimé son soupçon au Chef Commis Kuiper. À partir de là, les événements avaient suivi leur cours inévitable.

\og Quand a-t-il installé cette roulette~? \fg{}

\og Juste après avoir quitté la Planète de Garson. \fg{}

Max pensa avec malaise aux \og couvre-théières \fg{} qu'il avait aidé Sam à monter à bord là-bas. Noguchi poursuivit~: \og Euh, vous ne saviez pas, monsieur~? Je pensais que vous et lui étiez assez proches avant -- vous savez, avant que vous montiez de pont. \fg{}

Max évita de répondre et se plongea dans le journal de bord. Il le trouva sous la veille, ajouté par Bennett au journal de Simes. Sam était consigné au vaisseau pour le reste du voyage, toute mesure disciplinaire finale reportée jusqu'au retour sur Terre. Cette dernière partie semblait signifier que le Capitaine Blaine avait l'intention de donner à Sam une chance de montrer une bonne conduite avant de faire sa recommandation aux guildes -- le Capitaine était un gentil vieux bonhomme, vraiment. Mais \og consigné \fg{}~? Alors Sam n'aurait jamais la chance de fuir ce qu'il fuyait.

Il localisa Sam dès qu'il fut hors quart, le sortant de son dortoir et l'emmenant dans le couloir. Sam le regarda d'un air maussade. \og Je croyais t'avoir dit de ne pas me chercher~? \fg{}

\og Peu importe~! Sam, je suis inquiet pour toi. Ce truc de "consigné"... ça veut dire que tu n'auras pas la chance de -- \fg{}

\og \emph{Tais-toi~!} \fg{} C'était un murmure mais Max se tut. \og Maintenant écoute-moi, \fg{} poursuivit Sam, \og Oublie ça. J'ai mon magot et c'est le point important. \fg{}

\og Mais... \fg{}

\og Tu crois qu'ils peuvent sceller ce vaisseau assez hermétiquement pour me garder dedans quand je déciderai de partir~? Maintenant reste loin de moi. Tu es le chouchou du prof et je veux que ça reste comme ça. Je ne veux pas qu'on te fasse la leçon sur les mauvaises fréquentations, c'est-à-dire moi. \fg{}

\og Mais je veux aider, Sam. Je... \fg{}

\og Tu veux bien remonter au-dessus du pont "C" où est ta place~? \fg{}

Il ne revit pas Sam de cette étape~; bientôt il cessa de s'en inquiéter. Hendrix exigea qu'il calcule l'approche planétaire -- un jeu d'enfant comparé à une transition -- puis plaça Max à la barre quand ils se posèrent. C'était une responsabilité titulaire puisque c'était précalculé et fait en radar-automatique. Max était assis avec les commandes sous les mains, prêt à passer outre au pilote automatique -- et Hendrix se tenait derrière lui, prêt à passer outre à Max -- mais il n'y eut pas besoin~; l'\emph{Asgard} descendit selon la courbe tracée aussi facilement qu'en descendant des escaliers. Les faisceaux de poussée mordirent et Max rapporta~: \og Posés, monsieur, à l'heure prévue. \fg{}

\og Terminé. \fg{}

Max parla dans les haut-parleurs du vaisseau. \og Salle des machines, terminé. Terminé pour tous les postes spatiaux. Routine terrestre, deuxième section. \fg{}

Des quatre jours qu'ils passèrent là, il passa les trois premiers nominalement à superviser, et en fait à apprendre de, Kovak dans l'inspection et révision de routine de quatre-vingt-dix jours des instruments de la salle de contrôle. Ellie était fâchée contre lui, car elle avait d'autres projets. Mais le dernier jour il mit pied à terre avec elle, chaperonnés par M. et Mme Mendoza.

Ce fut des vacances merveilleuses. Comparé à la Terre, Halcyon est un endroit austère et Bonaparte n'est pas une grande ville. Néanmoins Halcyon est une planète de type terrestre avec une atmosphère respirable, et le groupe de l'\emph{Asgard} n'avait pas mis le pied dehors depuis Earthport, des mois de temps et des années-lumière impensables en arrière. La saison était post-aphélie, le milieu de l'été, Nu Pegasi brûlait chaud et brillant dans le ciel bleu.

M. Mendoza loua une voiture à chevaux et ils partirent dans une verte campagne vallonnée derrière quatre petits poneys d'Halcyon renifleurs. Là ils visitèrent un pueblo indigène, une grande structure en ruche de boue, conoïde sur conoïde, et achetèrent des souvenirs -- dont deux s'avérèrent porter la mention \og Fabriqué au Japon \fg{} discrètement imprimée dessus.

Leur cocher, Herr Eisenberg, servit d'interprète pour eux. L'indigène qui vendait les souvenirs n'arrêtait pas de tourner ses yeux, l'un après l'autre, vers Mme Mendoza. Il gazouilla quelques remarques au cocher, qui s'esclaffa.

\og Qu'est-ce qu'il dit~? \fg{} demanda-t-elle.

\og Il vous faisait des compliments. \fg{}

\og Ah~? Mais comment~? \fg{}

\og Eh bien... il dit que vous êtes faite pour un feu doux et qu'on n'a pas besoin d'assaisonnement~; vous cuisiriez bien. Et il le ferait, aussi, \fg{} ajouta le colon, \og si vous restiez ici après la tombée de la nuit. \fg{}

Mme Mendoza poussa un petit cri. \og Vous ne nous aviez pas dit qu'ils étaient \emph{cannibales}. Josie, ramène-moi~! \fg{}

Herr Eisenberg eut l'air horrifié. \og Cannibales~? Oh, non, madame~! Ils ne se mangent pas entre eux, ils nous mangent nous -- quand ils nous attrapent, c'est-à-dire. Mais il n'y a pas eu d'incident depuis vingt ans. \fg{}

\og Mais c'est pire~! \fg{}

\og Non, ça ne l'est pas, madame. Regardez de leur point de vue. Ils sont civilisés. Ce vieux bonhomme ne violerait jamais une de leurs lois. Mais pour eux nous sommes juste du bœuf de première qualité, malheureusement difficile à attraper. \fg{}

\og Ramène-nous tout de suite~! Mais il y en a des centaines, et nous ne sommes que cinq. \fg{}

\og Des milliers, madame. Mais vous êtes en sécurité tant que Gneeri brille. \fg{} Il fit un geste vers Nu Pegasi. \og C'est tabou de tuer de la viande pendant la journée. L'esprit reste pour hanter. \fg{}

Malgré ses assurances, le groupe repartit. Max remarqua qu'Eldreth n'avait pas eu peur. Lui-même s'était demandé ce qui avait empêché les indigènes de les ligoter jusqu'à la nuit.

Ils dînèrent au Joséphine, le meilleur (et unique) hôtel de Bonaparte. Mais il y avait un vrai orchestre de trois musiciens, une piste de danse, et de la nourriture qui était au moins un changement bienvenu des menus du Salon Bifrost. Beaucoup de passagers du vaisseau et plusieurs officiers étaient là~; ça faisait une joyeuse compagnie. Ellie fit danser Max entre chaque plat. Il rassembla même son courage pour inviter Mme Daigler à danser, après qu'elle fut venue le suggérer.

Pendant l'entracte, Eldreth le guida vers le balcon adjacent. Là elle leva les yeux vers lui. \og Laisse cette dévergondée de Daigler tranquille, tu m'entends~? \fg{}

\og Hein~? Je n'ai rien fait. \fg{}

Elle sourit soudain chaleureusement. \og Bien sûr que non, grand nigaud adorable. Mais Ellie doit prendre soin de toi. \fg{} Elle se retourna et s'appuya sur la balustrade.

La nuit précoce d'Halcyon était tombée, ses trois lunes se poursuivaient. Le ciel flamboyait de plus d'étoiles qu'on ne peut en voir dans le voisinage solitaire de la Terre. Max lui montra les constellations étranges et lui indiqua la direction de départ qu'ils prendraient demain pour atteindre la transition vers Nova Terra. Il avait appris quatre nouveaux ciels jusqu'ici, les connaissait aussi bien qu'il connaissait celui qui surplombait les Ozarks -- et il en apprendrait beaucoup d'autres. Il étudiait déjà, sur les cartes, d'autres ciels où ils seraient pendant ce voyage.

\og Oh, Max, n'est-ce pas \emph{magnifique}~! \fg{}

\og Ça c'est sûr. Tiens, il y a une étoile filante. Elles sont rares ici, très rares. \fg{}

\og Fais un vœu~! Fais un vœu vite~! \fg{}

\og D'accord. \fg{} Il souhaita s'en tirer facilement quand viendrait le règlement de comptes. Puis il décida que ce n'était pas bien~; il devrait souhaiter que le vieux Sam se sorte de son pétrin -- pas qu'il y croyait, de toute façon. Elle se tourna et lui fit face. \og Qu'est-ce que tu as souhaité~? \fg{}

\og Hein~? \fg{} Il fut soudain gêné. \og Oh, il ne faut pas le dire, ça gâche tout. \fg{}

\og D'accord. Mais je parie que tu auras ton vœu, \fg{} ajouta-t-elle doucement.

Il pensa un instant qu'il aurait pu l'embrasser, là, tout de suite, s'il avait bien joué ses cartes. Mais le moment passa et ils rentrèrent.

Le sentiment resta avec lui sur le trajet du retour, le rendit euphorique. C'était un bon vieux monde, même s'il y avait des passages difficiles. Le voilà pratiquement astrogateur junior à son premier voyage -- et ça ne faisait que quelques semaines qu'il empruntait les mules de McAllister pour travailler les champs et allait souvent pieds nus pour économiser les chaussures. Et pourtant le voilà en uniforme, assis à côté de la fille la mieux habillée de quatre planètes. Il caressa l'insigne sur sa poitrine.

Épouser Ellie n'était plus une idée aussi impossible maintenant qu'il était officier -- s'il décidait un jour de se marier. Peut-être que son vieux ne considérerait pas un officier -- et un astrogateur qui plus est -- comme complètement inéligible. Ellie n'était pas mal~; elle avait du cran et elle jouait une partie correcte de trois-dés -- la plupart des filles ne seraient même pas capables d'en apprendre les règles.

Il était encore dans une douce euphorie quand ils atteignirent le vaisseau et furent hissés à bord. Kelly l'accueillit au sas. \og M. Jones -- le Capitaine veut vous voir. \fg{}

\og Hein~? Oh. Bonne nuit, Ellie -- il faut que je file. \fg{} Il se hâta derrière Kelly. \og Qu'est-ce qui se passe~? \fg{}

\og Le Dr Hendrix est mort. \fg{}
