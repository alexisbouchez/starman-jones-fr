\chapter{Civilisation}

Ellie ne s'évanouit pas et ne devint pas hystérique. Après ce cri involontaire, sa remarque suivante fut simplement~: \og Max, je suis désolée. C'est ma faute. \fg{} Les mots étaient presque à son oreille, tant ils étaient étroitement liés ensemble par les cordes agrippantes. Il répondit~: \og Je vais nous libérer~! \fg{} et continua à forcer sur leurs liens.

\og Ne lutte pas, \fg{} dit-elle calmement, \og Ça ne fait que les resserrer. Il va falloir parlementer pour sortir de là. \fg{}

Ce qu'elle disait était vrai~; plus il forçait, plus les liens pythoniens les tenaient serrés. \og Non, \fg{} supplia Ellie. \og Tu aggraves les choses. Ça me fait mal. \fg{}

Max cessa. Le plus grand des centaures s'avança tranquillement et les examina. Son large visage simple était encore plus ridicule de près et ses grands yeux bruns avaient un air de doux étonnement. Le poulain approcha de l'autre côté et renifla avec curiosité, bêla d'une voix aiguë. L'adulte claironna comme un wapiti~; le poulain fit un écart de côté, puis rejoignit le troupeau au galop.

\og Du calme, \fg{} chuchota Ellie. \og Je pense qu'ils avaient peur qu'on fasse du mal au bébé. Peut-être qu'ils vont juste nous examiner et nous laisser partir. \fg{}

\og Peut-être. Mais j'aimerais pouvoir atteindre mon couteau. \fg{}

\og Je suis contente que tu ne puisses pas. Ça demande de la diplomatie. \fg{}

Le reste du troupeau s'approcha, tourna autour d'eux et les examina, tout en échangeant des appels qui combinaient clairons, hennissements, et quelque chose entre une toux et un reniflement. Max écouta. \og C'est un langage, \fg{} décida-t-il.

\og Bien sûr. Et comme j'aurais aimé l'avoir étudié chez Mademoiselle Mimsey. \fg{}

Le plus grand centaure se pencha sur eux, lissa leurs liens~; ils devinrent plus lâches mais les retenaient toujours. Max dit vivement~: \og Je pense qu'ils vont nous détacher. Prépare-toi à courir. \fg{}

\og Oui, chef. \fg{}

Un autre centaure plongea dans sa poche intégrée, en sortit une autre de ces choses semblables à des cordes. Il tomba sur ses genoux avant, fit claquer le bout de façon qu'il s'enroule autour de la cheville gauche de Max. Le bout sembla se souder en une boucle, entravant Max aussi efficacement qu'un nœud de chaise~; Ellie fut traitée de la même façon. Le plus grand centaure tapota alors leurs liens, qui tombèrent et se tortillèrent doucement sur le sol. Il les ramassa et les fourra dans sa poche.

Le centaure qui les avait entravés enroula les extrémités de leurs longes autour de son tronc vertical, elles fusionnèrent en une ceinture. Après un échange de claironnements aigres avec le chef, il tapota les laisses... qui alors s'étirèrent comme du caramel, devenant une bonne vingtaine de pieds de long et beaucoup plus minces.

Max passa son couteau à Ellie et dit~: \og Essaie de te libérer en coupant. Si tu y arrives, alors cours. Je les occuperai. \fg{}

\og Non, Max. \fg{}

\og Si~! Bon sang, arrête de faire ta gamine~! Tu as causé assez d'ennuis. \fg{}

\og Oui, Max. \fg{} Elle prit le couteau et essaya de scier l'étrange corde près de sa cheville. Les centaures ne firent aucune tentative pour l'arrêter, mais regardèrent avec le même air de doux étonnement. C'était comme s'ils n'avaient jamais vu de couteau, n'avaient aucune notion de ce que c'était. Bientôt elle abandonna. \og Pas moyen, Max. C'est comme essayer de trancher du duraplastique. \fg{}

\og Pourtant, je garde ce couteau comme un rasoir. Laisse-moi essayer. \fg{}

Il n'eut pas plus de chance. Il fut forcé de s'arrêter par le troupeau qui se mettait en marche -- marcher ou être traîné. Il réussit à fermer le couteau tout en sautillant sur un pied pour garder son équilibre. Le groupe avança à un pas lent pendant quelques foulées, puis le chef claironna et les centaures passèrent au trot, exactement comme l'ancienne cavalerie. Ellie trébucha aussitôt et fut traînée. Max s'assit, réussit à attraper son entrave et à s'y accrocher tout en criant~: \og Hé~! Stop~! \fg{}

Leur gardien s'arrêta et se retourna d'un air presque apologétique. Max dit~: \og Écoute, stupide. On ne peut pas suivre. On n'est pas des chevaux, \fg{} tout en aidant Ellie à se relever. \og Tu es blessée, petite~? \fg{}

\og Je crois pas. \fg{} Elle refoula ses larmes. \og Si je pouvais mettre la main sur ce crétin bouffeur de foin, c'est lui qui serait blessé -- et pas qu'un peu~! \fg{}

\og Tu t'es écorché la main. \fg{}

\og Ça ne va pas me tuer. Dis-lui juste de ralentir, tu veux~? \fg{}

Voyant qu'ils étaient debout, le monstre se remit immédiatement au trot. Ils retombèrent, avec Max essayant de les freiner. Cette fois le chef revint au trot depuis le gros du troupeau et consulta leur gardien. Max prit part à la discussion, compensant en véhémence ce qui lui manquait en efficacité sémantique. Peut-être fut-il efficace~; leur gardien ralentit à une marche rapide, laissant les autres aller devant. Un autre centaure resta en arrière et devint une arrière-garde. L'un des ballons animés, qui avait continué à planer au-dessus du troupeau, revint maintenant en arrière et resta au-dessus de Max et Ellie.

\bigskip

L'allure était tout juste supportable, entre une marche rapide et un petit trot. Le chemin traversait le fond plat et ouvert de la vallée à travers des herbes qui leur arrivaient aux genoux. L'herbe les sauva quelque peu, car le centaure qui les menait semblait considérer qu'une chute ou deux tous les quelques centaines de mètres représentait l'efficacité optimale. Il ne semblait jamais impatient et les laissait se relever, mais repartait toujours à une allure vive pour des humains. Max et Ellie cessèrent d'essayer de parler, leurs gorges étant brûlées à sec par leurs efforts haletants pour suivre.

Un minuscule ruisseau serpentait au fond de la vallée~; le centaure le franchit facilement d'un bond. Il fut nécessaire aux humains de patauger. Ellie fit une pause au milieu du cours d'eau, se pencha et commença à boire. Max protesta~: \og Ellie~! Ne bois pas ça -- tu ne sais pas si c'est sûr. \fg{}

\og J'espère que ça va m'empoisonner pour que je puisse m'allonger et mourir. Max, je ne peux pas aller beaucoup plus loin. \fg{}

\og Courage, petite. On va s'en sortir. J'ai gardé trace de notre chemin. \fg{} Il hésita, puis but aussi, étant terriblement assoiffé. Le centaure les laissa faire, puis les tira en avant.

La distance était aussi grande jusqu'au terrain montant et à la forêt de l'autre côté. Ils avaient cru être aussi fatigués qu'ils pouvaient l'être avant de commencer à monter~; ils se trompaient. Le centaure était agile comme une chèvre et semblait surpris qu'ils trouvent cela difficile. Finalement Ellie s'effondra et refusa de se relever~; le centaure revint et la remua rudement avec un sabot à trois orteils. Max le frappa des deux poings. Le centaure ne fit aucun mouvement pour riposter mais le regarda avec le même air d'étonnement stupide.

Leur garde arrière s'approcha et conversa avec lui, après quoi ils attendirent peut-être dix minutes. Max s'assit à côté d'Ellie et dit avec inquiétude~: \og Tu te sens mieux~? \fg{}

\og Ne parle pas. \fg{}

Bientôt la garde s'insinua entre eux et repoussa Max en lui marchant dessus, sur quoi l'autre centaure tira sur la laisse d'Ellie. Elle se contracta et elle fut forcée de se lever en hâte.

Les centaures les laissèrent se reposer deux fois après cela. Après un temps interminable, quand le soleil local descendait bas à l'ouest, ils débouchèrent sur un plateau plat, encore densément boisé. Ils continuèrent à travers les arbres sur une distance que le compte de pas de Max lui dit être de moins d'un mile mais qui semblait en faire dix, puis s'arrêtèrent.

Ils étaient dans une semi-clairière, un espace recouvert d'aiguilles tombées. Leur garde s'approcha de l'autre centaure et prit de lui le bout de la laisse de Max, le fit claquer autour de la base d'un arbre, auquel il s'accrocha. L'autre centaure fit de même avec la laisse d'Ellie à un autre arbre à environ douze mètres de là. Cela fait, ils les poussèrent rudement ensemble, tout en s'arrêtant pour caresser leurs liens jusqu'à ce qu'ils soient très étirés. Cela permit à Max et Ellie assez de mou pour qu'ils auraient pu se croiser. Cela ne sembla pas plaire aux centaures. L'un d'eux déplaça la laisse de Max plus loin dans les buissons environnants, le traînant avec. Cette fois, à la limite extrême permise par leurs liens, ils étaient à environ deux mètres l'un de l'autre.

\og Qu'est-ce qu'ils font~? \fg{} demanda Ellie.

\og On dirait qu'ils ne veulent pas qu'on unisse nos forces. \fg{}

Ayant terminé, les centaures partirent au trot. Ellie les regarda partir, commença à sangloter, puis pleura ouvertement, les larmes coulant sur son visage sale et laissant des traces. \og Arrête ça, \fg{} dit Max durement. \og Pleurnicher ne nous mènera nulle part. \fg{}

\og Je ne peux pas m'en empêcher, \fg{} brailla-t-elle. \og J'ai été courageuse toute la journée -- au moins j'ai essayé. Je... \fg{} Elle s'effondra face contre terre et se laissa aller.

En se couchant à plat ventre et en s'étirant, Max pouvait tout juste atteindre sa tête. Il tapota ses cheveux emmêlés. \og Du calme, petite, \fg{} dit-il doucement. \og Pleure un bon coup, si ça te fait du bien. \fg{}

\og Oh, Maxie~! Attachée... comme un chien. \fg{}

\og On va voir ça. \fg{} Il s'assit et examina sa longe. Quoi que fût cette chose semblable à une corde, ce n'était pas une corde. Elle avait une surface lisse et brillante qui lui rappelait plus un serpent, bien que la partie qui s'enroulait autour de sa cheville ne montrât aucun trait~; elle coulait simplement autour de sa cheville et fusionnait avec elle-même. Il souleva la boucle et détecta un faible battement. Il la caressa comme il avait vu les centaures faire et elle répondit par des pulsations fluides, mais elle ne rétrécit ni ne s'allongea, ni ne desserra sa prise.

\og Ellie, \fg{} annonça-t-il, \og Cette chose est \emph{vivante}. \fg{}

Elle leva un visage défait. \og Quelle chose~? \fg{}

\og Cette corde. \fg{}

\og Oh, ça~! Évidemment. \fg{}

\og Du moins, \fg{} poursuivit-il, \og si elle ne l'est pas, elle n'est pas vraiment morte. \fg{} Il essaya de nouveau son couteau, il n'y eut aucun effet. \og Je parie que si j'avais une allumette je pourrais la faire crier grâce. T'as un Everlite, Ellie~? \fg{}

\og Je ne fume pas. \fg{}

\og Moi non plus. Bon, peut-être que je peux faire du feu autrement. En frottant deux bâtons ensemble, ou quelque chose. \fg{}

\og Tu sais comment faire~? \fg{}

\og Non. \fg{} Il continua à caresser et tapoter la corde vivante, mais, bien qu'il obtînt toujours une réponse en pulsations, il ne semblait pas avoir le bon toucher~; le lien restait comme avant.

Il continuait cette tentative infructueuse quand il entendit son nom appelé. \og Max~! Ellie~! \fg{}

Ellie se redressa d'un coup. \og Chipsie~! Oh, Max, elle nous a suivis. Viens ici, chérie~! \fg{}

Le chiot-araignée était haut au-dessus d'eux dans un arbre. Elle regarda soigneusement autour d'elle, puis descendit en vitesse, faisant des trois derniers mètres un bond volant dans les bras d'Ellie. Elles se câlinèrent et firent des bruits doux, puis Ellie se redressa, les yeux brillants. \og Max, je me sens tellement mieux. \fg{}

\og Moi aussi. \fg{} Il ajouta~: \og Bien que je ne sache pas pourquoi. \fg{}

Le chiot-araignée annonça gravement~: \og Chipsie a suivi. \fg{}

Max se pencha et la caressa. \og Oui, Chipsie l'a fait. Bonne fille~! \fg{}

Ellie serra le chiot-araignée. \og Je ne me sens plus abandonnée maintenant, Max. Peut-être que tout va bien se passer. \fg{}

\og Écoute, Ellie, on n'est pas dans une si mauvaise situation. Peut-être que je trouverai la combinaison pour chatouiller ces cordes ou serpents ou quoi que ce soit pour qu'ils abandonnent. Si j'y arrive, on se faufilera en arrière cette nuit. \fg{}

\og Comment on trouverait notre chemin~? \fg{}

\og Ne t'inquiète pas. J'ai observé chaque pas du chemin, chaque changement de direction, chaque repère. \fg{}

\og Même dans le noir~? \fg{}

\og Plus facile dans le noir. Je connais ces étoiles -- je devrais bien. Mais supposons qu'on ne se libère pas~; on n'est toujours pas fichus. \fg{}

\og Hein~? Ça ne me dit rien de passer ma vie attachée à un arbre. \fg{}

\og Ça n'arrivera pas. Écoute -- je pense que ces choses sont juste curieuses de nous. Elles ne nous mangeront pas, c'est sûr -- elles vivent probablement d'herbe. Peut-être qu'elles s'ennuieront et nous libéreront. Mais sinon, ce sera dur pour elles. \fg{}

\og Hein~? Pourquoi~? \fg{}

\og À cause de M. Walther et George Daigler -- et Sam, Sam Anderson~; voilà pourquoi. Ils sont probablement en train de battre les buissons pour nous en ce moment même. On est à moins de quinze kilomètres du vaisseau -- huit en ligne droite. Ils nous trouveront. Alors si ces centaures ridicules veulent jouer les durs, ils apprendront ce que sont les armes modernes. Eux et leurs stupides cordes de lancer~! \fg{}

\og Il pourrait falloir longtemps pour nous trouver. Personne ne sait où nous sommes allés. \fg{}

\og Oui, \fg{} admit-il. \og Si j'avais une radio de poche. Ou un moyen de signaler. Ou même un moyen de faire du feu. Mais je n'en ai pas. \fg{}

\og Je n'y avais jamais pensé. Ça semblait juste être une promenade dans le parc. \fg{}

Max pensa sombrement qu'il avait essayé de l'avertir. Pourquoi, même les collines autour de chez lui n'étaient pas sûres si on ne gardait pas les yeux ouverts... on pouvait tomber sur un vieux lynx méchant, ou même un ours. Une personne comme Ellie n'avait jamais eu assez de coups durs pour lui mettre du plomb dans la cervelle, c'était ça son problème.

Bientôt il admit que lui-même n'avait pas cherché d'ennuis de la part de quelque chose d'aussi apparemment abruti et inoffensif que ces centaures. De toute façon, comme Sam dirait, pas la peine de pleurer sur le lait renversé quand le cheval est déjà volé.

\og Ellie. \fg{}

\og Hein~? \fg{}

\og Tu crois que Chipsie pourrait retrouver son chemin~? \fg{}

\og Ben, je ne sais pas. \fg{}

\og Si elle pouvait, on pourrait envoyer un message. \fg{}

Chipsie leva les yeux. \og Retour~? \fg{} demanda-t-elle. \og S'il vous plaît retour. Rentrer maison. \fg{}

Ellie fronça les sourcils. \og J'ai peur que Chipsie ne parle pas si bien. Elle hoquèterait probablement et deviendrait incohérente. \fg{}

\og Je ne veux pas dire ça. Je sais que Chipsie n'est pas un génie. Je... \fg{}

\og Chipsie est intelligente~! \fg{}

\og Bien sûr. Mais je veux envoyer un message écrit et une carte. \fg{} Il fouilla dans une poche, en sortit un stylet. \og Tu as du papier~? \fg{}

\og Je vais voir. \fg{} Elle trouva un papier plié dans une poche de salopette. \og Oh, zut~! Je devais apporter ça à M. Giordano. M. Hornsby va être tellement fâché contre moi. \fg{}

\og Qu'est-ce que c'est~? \fg{}

\og Une réquisition pour du fil numéro dix. \fg{}

\og Ça n'a plus d'importance maintenant. \fg{} Il prit le papier, ratura le mémo, le retourna et commença à dessiner, s'arrêtant pour consulter les images classées dans son esprit pour les distances, où se trouvait le soleil local, les contours, et d'autres détails. \og Max~? \fg{}

\og Silence, tu veux~? \fg{} Il continua à dessiner, puis ajouta~: \og URGENT -- au Premier Officier Walther~: Eldreth Coburn et moi-même capturés par des centaures. Soyez prudents et méfiez-vous de leurs cordes de lancer. Respectueusement, M. Jones. \fg{}

Il le tendit à Ellie. \og Ça devrait faire l'affaire. Y a-t-il un moyen de l'attacher à elle~? Je ne veux vraiment pas qu'elle le laisse tomber. \fg{}

\og Mmm... laisse-moi voir. Tourne-toi, Max. \fg{}

\og Pourquoi~? \fg{}

\og Ne fais pas de difficultés. Tourne-toi. \fg{}

Il le fit, peu après elle dit~: \og Voilà, c'est bon. \fg{} Il se retourna et elle lui tendit un ruban. \og Ça ira~? \fg{}

\og Super~! \fg{}

Ils réussirent à attacher le ruban, avec le mot plié et fermement attaché, autour de la taille de Mademoiselle Chips, l'ancrant à un membre du milieu... pas si facile car le chiot-araignée semblait croire que c'était un jeu et était chatouilleuse aussi.

\og Voilà~! Arrête de gigoter, Chipsie, et écoute. Ellie veut que tu rentres à la maison. \fg{}

\og Maison~? \fg{}

\og Oui, maison. Retourne au vaisseau. \fg{}

\og Ellie rentre maison~? \fg{}

\og Ellie ne peut pas rentrer. \fg{}

\og Non. \fg{}

\og Chérie, il \emph{faut} que tu y ailles. \fg{}

\og Non. \fg{}

\og Écoute, Chipsie. Tu trouves Maggie et tu lui dis qu'Ellie a dit de te donner des bonbons. Tu donnes ça à Maggie. \fg{} Elle tira sur le mot attaché. \og Bonbons~? \fg{}

\og Rentre à la maison. Trouve Maggie. Maggie te donnera des bonbons. \fg{}

\og Ellie rentre maison. \fg{}

\og S'il te plaît, Chipsie. \fg{}

\og Ellie, \fg{} dit Max avec urgence, \og quelque chose arrive. \fg{}

Eldreth leva les yeux, vit un centaure venir à travers les arbres. Elle pointa du doigt. \og Regarde, Chipsie~! Ils arrivent~! Ils vont attraper Chipsie~! Rentre~! Cours~! \fg{}

Le chiot-araignée couina de terreur et fila vers les arbres. Une fois sur une branche elle regarda en arrière et gémit. \og Rentre~! \fg{} cria Ellie. \og Trouve Maggie~! \fg{}

Mademoiselle Chips jeta un coup d'œil au centaure, puis disparut. Ils n'eurent pas le temps de s'inquiéter davantage, le centaure était presque arrivé. Il leur jeta un coup d'œil et passa~; c'était ce qui suivait le centaure qui attira leur attention.

Ellie réprima un cri. \og Max~! Ils ont attrapé tout le monde. \fg{}

\og Non, \fg{} corrigea-t-il sinistrement. \og Regarde encore. \fg{}

L'obscurité grandissante lui avait fait faire la même erreur~; il semblait que tout l'équipage du vaisseau trottait derrière le centaure en file indienne, cheville attachée à la cheville par des cordes vivantes. Mais seul le premier coup d'œil donnait une telle impression. Ces créatures étaient plus qu'humanoïdes -- mais de telles créatures dégradées n'avaient jamais navigué entre les étoiles. Elles traînaient les pieds rapidement comme des animaux bien dressés. Une ou deux regardèrent Ellie et Max en passant, mais leurs regards étaient bovins, sans curiosité. De petits enfants non attachés trottaient avec leurs mères, et une fois Max fut surpris de voir une petite tête ridée pointer hors d'une poche -- ces créatures-hommes étaient des marsupiaux aussi.

Max contrôla une envie de vomir et alors qu'ils disparaissaient de vue il se tourna vers Ellie. \og Mon Dieu~! \fg{}

\og Max, \fg{} dit Eldreth d'une voix rauque, \og tu crois qu'on est morts et qu'on est en enfer~? \fg{}

\og Hein~? Ne sois pas bête. Les choses sont assez mauvaises comme ça. \fg{}

\og Je suis sérieuse. C'était quelque chose tout droit sorti de l'Enfer de Dante. \fg{}

Max avalait avec malaise et n'était pas de bonne humeur. \og Écoute, tu peux faire semblant d'être morte si tu veux. Moi, je suis vivant et j'ai l'intention de le rester. Ces choses n'étaient pas des hommes. Ne te laisse pas abattre. \fg{}

\og Mais c'\emph{étaient} des hommes. Des hommes et des femmes et des enfants. \fg{}

\og Non, ce n'en étaient pas. Avoir notre forme ne fait pas d'eux des hommes. Être un homme c'est quelque chose d'entièrement différent. \fg{} Il fronça les sourcils. \og Peut-être que les centaures sont des "hommes". \fg{}

\og Oh, non -- \fg{}

\og N'en sois pas si sûre. Ils semblent diriger les choses dans ce pays. \fg{}

La discussion fut écourtée par une autre arrivée. Il faisait presque nuit et ils ne virent pas le centaure avant qu'il n'entre dans leur clairière. Il était suivi de trois des -- Max décida de les appeler \og hommes \fg{} bien qu'il en fût contrarié -- suivi de trois hommes. Ils n'étaient pas en laisse. Tous les trois portaient des fardeaux. Le centaure leur parla~; ils distribuèrent ce qu'ils portaient.

L'un d'eux posa un grand bol d'argile rempli d'eau dans l'espace séparant Max et Ellie. C'était le premier artefact qu'aucun humain avait vu sur Charité et n'indiquait pas un haut niveau de culture mécanique, étant grossièrement modelé et clairement pas tourné sur un tour de potier~; il contenait de l'eau, on ne pouvait en dire plus.

Un deuxième porteur déversa une double brassée de petits fruits à côté du bol. Deux d'entre eux éclaboussèrent dans le bol, il ne se donna pas la peine de les repêcher.

Max dut regarder à deux fois pour voir ce que le troisième esclave portait. On aurait dit qu'il avait trois grosses boules ovoïdes suspendues par des cordes dans chaque main~; un second regard montra que c'étaient des animaux de la taille d'opossums qu'il portait par la queue. Il fit le tour de la clairière, s'arrêtant tous les quelques pieds et soulevant un de ses fardeaux sur une branche basse. Quand il eut fini ils étaient entourés de six petites créatures, chacune pendue par sa queue.

Le centaure suivit l'esclave, Max le vit caresser chaque animal et presser un point sur son cou. Dans chaque cas le corps entier du petit animal s'illumina, commença à briller comme une luciole d'une douce lumière argentée. La clairière en était doucement illuminée -- assez bien, pensa Max, pour lire en gros caractères.

L'un des ballons farfadets vint naviguer silencieusement entre les arbres et s'ancra à un point dix mètres au-dessus d'eux~; il sembla s'installer pour la nuit.

Le centaure vint vers Max et le poussa avec un sabot, reniflant d'un air interrogateur. Max écouta attentivement, puis répéta le son. Le centaure répondit et de nouveau Max l'imita. Cet échange inutile continua pendant quelques phrases, puis le centaure abandonna et partit, son train trottant derrière lui.

Ellie frissonna. \og Ouf~! \fg{} s'exclama-t-elle, \og Je suis contente qu'ils soient partis. Je peux supporter les centaures, un peu, mais ces hommes... beurk~! \fg{}

Il partageait son dégoût~; ils avaient l'air moins humains de près, ayant des lignes de cheveux qui commençaient là où leurs sourcils auraient dû être. Ils avaient le crâne si plat que leurs oreilles dépassaient au-dessus de leurs crânes. Mais ce n'était pas cela qui avait impressionné Max. Quand le centaure lui avait parlé Max avait eu son premier bon regard dans la bouche d'un centaure. Ces dents n'étaient pas faites pour mâcher du grain, elles ressemblaient plus aux dents d'un tigre -- ou d'un requin. Il décida de ne pas mentionner cela.

\og Dis, ce n'était pas le même qui menait le troupeau qui nous a capturés~? \fg{}

\og Comment je le saurais~? Ils se ressemblent tous. \fg{}

\og Mais non, pas plus que deux chevaux ne se ressemblent. \fg{}

\og Les chevaux se ressemblent tous. \fg{}

\og Mais... \fg{} Il s'arrêta, déconcerté par un point de vue citadin où la communication échouait. \og Je pense que c'était le même. \fg{}

\og Je ne vois pas que ça importe. \fg{}

\og Ça pourrait. J'essaie d'apprendre leur langue. \fg{}

\og Je t'ai entendu avaler tes amygdales. Comment tu fais ça~? \fg{}

\og Oh, tu te souviens juste de ce à quoi un son ressemble, puis tu le fais. \fg{} Il rejeta la tête en arrière et fit un son très plaintif.

\og C'était quoi \emph{ça}~? \fg{}

\og Un porcelet coincé dans une clôture. Un petit porcelet du nom d'Abner que j'ai eu une fois. \fg{}

\og Ça a l'air tragique. \fg{}

\og Ça l'était, jusqu'à ce que je l'aide à se libérer. Ellie, je pense qu'ils nous ont installés pour la nuit. \fg{} Il fit un geste vers le bol et les fruits à côté. \og Comme nourrir les cochons. \fg{}

\og Ne dis pas ça comme ça. Service en chambre. Service en chambre et service de femme de chambre et lumières. De la nourriture et de la boisson. \fg{} Elle ramassa un des fruits. Il avait à peu près la taille et la forme d'un concombre. \og Tu crois que c'est mangeable~? \fg{}

\og Je ne pense pas que tu devrais essayer. Ellie, ce serait malin de ne rien manger ni boire jusqu'à ce qu'on soit secourus. \fg{}

\og Mais on pourrait peut-être avoir faim mais on ne peut certainement pas se passer d'eau. On meurt de soif en un jour ou deux. \fg{}

\og Mais on sera peut-être secourus avant le matin. \fg{}

\og Peut-être. \fg{} Elle pela le fruit. \og Ça sent bon. Un peu comme une banane. \fg{}

Il en pela un et le renifla. \og Plus comme une papaye. \fg{}

\og Alors~? \fg{}

\og Mmm -- Écoute, je vais en manger un. Si ça ne m'a pas rendu malade dans une demi-heure, alors tu pourras en essayer un. \fg{}

\og Oui, chef. \fg{} Elle mordit dans celui qu'elle tenait. \og Attention aux pépins. \fg{}

\og Ellie, tu es une délinquante juvénile. \fg{}

Elle plissa le nez et sourit. \og Tu dis les choses les plus douces~! J'essaie de l'être. \fg{}

Max mordit dans le sien. Pas mal -- pas autant de goût qu'une papaye, mais pas mal. Quelques minutes plus tard il disait~: \og Peut-être qu'on devrait en garder pour le petit-déjeuner~? \fg{}

\og D'accord. Je suis pleine de toute façon. \fg{} Ellie se pencha et but. Sans paroles ils avaient tous deux conclu que le repas écœurant les obligeait à risquer l'eau. \og Voilà, je me sens mieux. Au moins on mourra confortablement. Max~? Tu crois qu'on ose dormir~? Je suis morte. \fg{}

\og Je pense qu'ils en ont fini avec nous pour la nuit. Tu dors, je vais veiller. \fg{}

\og Non, ce n'est pas juste. Honnêtement, à quoi ça servirait de monter la garde~? On ne peut pas s'enfuir. \fg{}

\og Bon... tiens, prends mon couteau. Tu peux dormir avec dans la main. \fg{}

\og D'accord. \fg{} Elle tendit la main au-dessus du bol et l'accepta. \og Bonne nuit, Max. Je vais compter les moutons. \fg{}

\og Bonne nuit. \fg{} Il s'étendit, se décala et enleva une pomme de pin de ses côtes, puis essaya de se détendre. La fatigue et un estomac plein aidaient, la connaissance de leur situation difficile gênait -- et ce farfadet qui pendait là-haut. Peut-être qu'\emph{il} montait la garde -- mais pas pour leur bénéfice.

\og Max~? Tu dors~? \fg{}

\og Non, Ellie. \fg{}

\og Tiens ma main~? J'ai \emph{peur}. \fg{}

\og Je ne peux pas l'atteindre. \fg{}

\og Si, tu peux. Tourne-toi de l'autre côté. \fg{}

Il le fit, et trouva qu'il pouvait tendre le bras au-dessus de sa tête au-delà du bol d'eau et serrer sa main. \og Merci, Max. Bonne nuit encore. \fg{}

Il resta allongé sur le dos et fixa à travers les arbres. Malgré la demi-lumière donnée par les animaux luminifères il pouvait voir les étoiles et les nombreuses traînées de météores zébrant le ciel. Pour éviter de penser il commença à les compter. Bientôt elles commencèrent à exploser dans sa tête et il était endormi.

\bigskip

La lumière du soleil local à travers les arbres le réveilla. Il leva la tête. \og Je me demandais combien de temps tu allais dormir, \fg{} annonça Eldreth. \og Regarde qui est là. \fg{}

Il s'assit, grimaçant à chaque mouvement, et se retourna. Mademoiselle Chips était assise sur le ventre d'Ellie et pelait un des fruits semblables à des papayes. \og 'lut, Maxie. \fg{}

\og Bonjour, Chipsie. \fg{} Il vit que le mot était toujours attaché à elle. \og Méchante fille~! \fg{}

Mademoiselle Chips se tourna vers Ellie pour être réconfortée. Des larmes commencèrent à couler. \og Non, non, \fg{} corrigea Ellie. \og Bonne fille. Elle a promis d'aller trouver Maggie dès qu'elle aura fini son petit-déjeuner. N'est-ce pas, chérie~? \fg{}

\og Aller trouver Maggie, \fg{} approuva le chiot-araignée.

\og Ne la blâme pas, Max. Les chiots-araignées ne sont pas nocturnes chez eux. Elle a juste attendu qu'on se calme, puis est revenue. Elle ne pouvait pas s'en empêcher. Je l'ai trouvée endormie dans mon bras. \fg{}

Le chiot-araignée finit de manger, puis but délicatement dans le bol. Max décida que ça n'avait pas d'importance, considérant qui l'avait probablement utilisé avant eux. Cette pensée il la réprima vite.

\og Trouver Maggie, \fg{} annonça Mademoiselle Chips.

\og Oui, chérie. Va directement au vaisseau aussi vite que tu peux et trouve Maggie. Dépêche-toi. \fg{}

\og Trouver Maggie. Vite dépêcher. 'lut, Maxie. \fg{} Le chiot-araignée monta dans les arbres et fila dans la bonne direction.

\og Tu crois qu'elle y arrivera~? \fg{} demanda Max.

\og Je le pense. Après tout, ses ancêtres ont trouvé leur chemin à travers les forêts et tout ça pendant beaucoup de générations. Elle sait que c'est important~; on a eu une longue conversation. \fg{}

\og Tu crois vraiment qu'elle comprend autant~? \fg{}

\og Elle comprend qu'il faut me faire plaisir et c'est assez. Max, tu crois qu'ils peuvent nous atteindre aujourd'hui~? Je ne veux pas passer une autre nuit ici. \fg{}

\og Moi non plus. Si Chipsie peut aller plus vite que nous... \fg{}

\og Oh, elle peut. \fg{}

\og Alors peut-être -- s'ils partent vite. \fg{}

\og J'espère. Prêt pour le petit-déjeuner~? \fg{}

\og Chipsie a laissé quelque chose~? \fg{}

\og Trois chacun. J'ai pris les miens. Tiens. \fg{}

\og Tu es sûre que tu ne mens pas~? Il n'y en avait que cinq quand on s'est endormis. \fg{}

Elle eut l'air penaud et lui permit de partager le fruit en trop. Pendant qu'ils mangeaient il remarqua un changement. \og Hé, qu'est-ce qui est arrivé aux lucioles géantes~? \fg{}

\og Oh. Une de ces horribles créatures est venue à l'aube et les a emportées. J'allais crier mais il ne s'est pas approché de moi, alors je t'ai laissé dormir. \fg{}

\og Merci. Je vois que notre chaperon est avec nous. \fg{} Le farfadet pendait toujours dans les cimes des arbres.

\og Oui, et il y a eu des zieuters tout autour de nous ce matin aussi. \fg{}

\og Tu en as vu un~? \fg{}

\og Bien sûr que non. \fg{} Elle se leva, s'étira et grimaça. \og Maintenant voyons quelles belles surprises cette belle journée va nous apporter. \fg{} Elle fit une grimace. \og Le programme que je choisirais serait de rester assise ici et de ne poser les yeux sur rien jusqu'à ce que George Daigler se montre avec une douzaine d'hommes armés. Je l'embrasserais. Je les embrasserais tous. \fg{}

\og Moi aussi. \fg{}

Jusqu'à bien après midi le programme choisi par Eldreth prévalut, rien ne se passa. Ils entendirent de temps en temps le claironnement et le reniflement des centaures mais n'en virent aucun. Ils parlèrent de façon décousue, ayant déjà épuisé à la fois les espoirs et les craintes, et somnolaient au soleil, quand ils se rendirent soudain compte qu'un centaure entrait dans la clairière.

Max était sûr que c'était le chef du troupeau, ou du moins que c'était celui qui les avait nourris et abreuvés. La créature ne perdit pas de temps, faisant clairement comprendre avec des coups de pied et des coups de sabot qu'ils devaient se laisser attacher en laisse pour le voyage. Jamais ils ne furent libérés des cordes vivantes.

Max pensa à attaquer le centaure, peut-être sauter sur son dos et lui trancher la gorge. Mais il semblait très improbable qu'il puisse le faire assez silencieusement~; un reniflement pourrait faire fondre le troupeau sur eux. En plus il ne connaissait aucun moyen de se libérer de leurs liens même s'il tuait le centaure. Mieux valait attendre -- surtout avec un messager parti chercher de l'aide.

Ils furent conduits, tombant et traînés occasionnellement, le long de la route prise par le groupe d'esclaves. Il devint apparent qu'ils entraient dans une grande colonie de centaures. Le sentier s'ouvrait sur une route sinueuse bien entretenue avec des centaures allant dans les deux sens et bifurquant sur des routes latérales. Il n'y avait pas de bâtiments, aucune des marques extérieures d'une race civilisée -- mais il y avait un air d'organisation, de coutume, de stabilité. De petits centaures gambadaient, gênaient le passage, et étaient renvoyés sur le côté. Il y avait des activités de différentes sortes des deux côtés de la route et des esclaves humains grotesques étaient presque aussi nombreux que les centaures, portant des fardeaux, travaillant de façons inexpliquées -- certains avec des liens de cordes vivantes, certains autorisés à courir librement.

Ils ne pouvaient pas voir grand-chose à cause de l'allure inconfortable qu'on les forçait à maintenir. Une fois Max nota une activité de son côté de la route qu'il souhaita mieux voir. Il ne la mentionna pas à Ellie, non seulement parce que parler était difficile mais parce qu'il ne voulait pas l'inquiéter -- mais ça ressemblait à une boucherie en plein air pour lui. Les carcasses pendues n'étaient pas des centaures.

Ils s'arrêtèrent enfin dans une très grande clairière, bien remplie de centaures. Leur maître tapota les lignes qui les liaient et leur fit ainsi raccourcir jusqu'à ce qu'ils soient amenés près de ses flancs. Il prit alors sa place dans une file de centaures.

Un grand centaure grisonnant et vraisemblablement âgé tenait audience d'un côté de la \og place \fg{}. Il se tenait avec une dignité tranquille pendant que des centaures individuels ou des groupes venaient successivement devant lui. Max regarda avec un intérêt si grand qu'il en oublia presque sa peur. Chaque cas était la cause de beaucoup de discussion, puis le chef des centaures faisait une seule remarque et le cas était clos. Les adversaires partaient tranquillement. La conclusion était inévitable que la loi ou la coutume était administrée, avec le grand centaure comme arbitre.

Il n'y avait aucune des parodies d'hommes dans la clairière mais il y avait sous les pieds d'étranges animaux qui ressemblaient à des cochons aplatis. Leurs pattes étaient si courtes qu'elles ressemblaient plus à des chenilles de tracteur. Ils étaient surtout bouche et dents et groins renifleurs, et quoi qu'ils rencontrent, si ce n'était pas le sabot d'un centaure, ils le dévoraient. Max comprit en les regardant comment la zone, bien que densément habitée, était maintenue si propre~; ces charognards étaient des balayeurs de rue animés.

Leur maître progressa graduellement vers la tête de la file. Le dernier cas avant le leur concernait le seul centaure qu'ils avaient vu qui ne semblait pas en santé vibrante. Il était vieux et maigre, son pelage était terne et ses os perçaient piteusement à travers sa peau. Un œil était aveugle, d'un blanc opaque~; l'autre était enflammé et suintait un épais ichor. Le juge, maire, ou chef de troupeau suprême discuta de son cas avec deux jeunes centaures en bonne santé qui semblaient le soigner presque comme des infirmiers. Puis le centaure chef quitta sa position d'honneur et marcha autour du malade, l'inspectant de tous les côtés. Puis il lui parla.

Le vieux malade répondit faiblement, un seul mot reniflé.

Le chef centaure parla de nouveau, obtint ce qui sembla à Max être la même réponse.

Le chef recula dans sa position précédente, émit un curieux cri hennissant. De tous les côtés les charognards trapus convergèrent vers l'endroit. Ils formèrent un cercle autour du malade et de ses accompagnateurs, des dizaines d'entre eux, reniflant et grognant.

Le chef claironna une fois~; un accompagnateur plongea dans sa poche et en sortit une créature enroulée en boule, le centaure la caressa et elle se déroula. Pour Max elle ressemblait désagréablement à une anguille.

L'accompagnateur la tendit vers le centaure malade. Il ne fit aucun mouvement pour l'arrêter, mais attendit, regardant avec son unique œil valide. La tête de la chose mince fut soudain touchée au cou du centaure malade~; il sursauta dans la convulsion caractéristique d'un choc électrique et s'effondra.

Le chef centaure renifla une fois -- et les charognards avancèrent en dandinant avec une vitesse surprenante, grouillant sur le corps et le dissimulant. Quand ils reculèrent, reniflant toujours, il n'y avait même plus d'os.

Max appela doucement~: \og Du calme, Ellie~! Ressaisis-toi, petite. \fg{}

Elle répondit faiblement~: \og Ça va. \fg{}
