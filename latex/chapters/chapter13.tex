\chapter{Transition}

Max interrogea Kelly tandis qu'ils se hâtaient vers la cabine du Capitaine. \og Je ne sais pas. Je ne sais vraiment pas, Max. \fg{} Kelly semblait au bord des larmes. \og Je l'ai vu avant le dîner -- il est venu au Trou pour vérifier ce que toi et Kovak avez fait. Il avait l'air d'aller bien. Mais le Commissaire l'a trouvé mort dans sa couchette, au milieu de la soirée. \fg{} Il ajouta avec inquiétude~: \og Je ne sais pas ce qui va se passer maintenant. \fg{}

\og Qu'est-ce que tu veux dire~? \fg{}

\og Eh bien... si j'étais capitaine, je resterais ici et j'enverrais chercher un remplaçant. Mais je ne sais pas. \fg{}

Pour la première fois Max réalisa que ce changement ferait de M. Simes l'astrogateur. \og Combien de temps faudrait-il pour avoir un remplaçant~? \fg{}

\og Calcule. Le \emph{Dragon} est à environ trois mois derrière nous~; il ramasserait notre courrier. Un an environ. \fg{}

Dans les contradictions du voyage interstellaire, les vaisseaux eux-mêmes étaient le moyen de communication le plus rapide~; un message radio (si une chose aussi stupide avait été tentée) aurait mis plus de deux siècles pour atteindre la Terre, autant pour une réponse.

Max trouva la cabine du Capitaine ouverte et bondée d'officiers, tous debout, ne disant rien, et l'air solennel~; il se glissa à l'intérieur sans s'annoncer et essaya de se faire discret. Kelly n'entra pas. Le Capitaine Blaine était assis à son bureau, la tête baissée. Plusieurs retardataires, membres de la joyeuse compagnie du Joséphine, arrivèrent après Max~; le Premier Officier Walther les compta des yeux, puis dit doucement à Blaine~: \og Tous les officiers du vaisseau présents, monsieur. \fg{}

Le Capitaine Blaine leva la tête et Max fut choqué de voir combien il avait l'air vieux. \og Messieurs, \fg{} dit-il d'une voix basse, \og vous connaissez la triste nouvelle. Le Dr Hendrix a été trouvé mort dans sa chambre ce soir. Crise cardiaque. Le Médecin me dit qu'il est passé environ deux heures avant d'être trouvé -- et que sa mort a probablement été presque indolore. \fg{} Sa voix se brisa, puis il continua. \og Le Frère Hendrix sera placé sur sa dernière orbite deux heures après notre décollage demain. C'est ainsi qu'il l'aurait voulu, la Galaxie était son foyer. Il s'est donné sans compter pour que les hommes voyagent en sécurité parmi les étoiles. \fg{}

Il fit une pause si longue que Max crut que le vieil homme avait oublié que d'autres étaient présents. Mais quand il reprit, sa voix était presque vive. \og C'est tout, messieurs. Les astrogateurs voudront bien rester. \fg{} Max n'était pas sûr de compter comme astrogateur mais l'usage du pluriel le décida. Le Premier Officier Walther commença à partir~; Blaine le rappela. Quand les quatre furent seuls, le Capitaine dit~: \og M. Simes, vous prendrez les fonctions de chef de département immédiatement. M., euh... \fg{}~; ses yeux se posèrent sur Max.

\og Jones, monsieur. \fg{}

\og M. Jones assumera vos fonctions de routine, bien sûr. Cette tragédie vous laisse à court de personnel~; pour le reste de ce voyage je tiendrai un quart régulier. \fg{}

Simes intervint. \og Ce n'est pas nécessaire, Capitaine. Nous nous débrouillerons. \fg{}

\og Peut-être. Mais ce sont mes souhaits. \fg{}

\og Bien, monsieur. \fg{}

\og Préparez le décollage à l'heure prévue. Des questions~? \fg{}

\og Non, monsieur. \fg{}

\og Bonne nuit, messieurs. Dutch, reste un moment, s'il te plaît~? \fg{}

Dehors, devant la porte, Simes commença à s'éloigner~; Max l'arrêta. \og M. Simes~? \fg{}

\og Hein~? Oui~? \fg{}

\og Des instructions pour moi, monsieur~? \fg{}

Simes le toisa. \og Vous tenez votre quart, Monsieur. Je m'occuperai de tout le reste. \fg{}

Le lendemain matin Max trouva un brassard de crêpe sur son bureau et un avis du Premier Officier que le deuil durerait une semaine. L'\emph{Asgard} décolla à l'heure, avec le Capitaine assis tranquillement dans son fauteuil, avec Simes à la console de contrôle. Max se tenait près du Capitaine, sans rien à faire. À part l'absence d'Hendrix tout était routine -- sauf que Kelly était d'assez mauvaise humeur.

Simes, Max l'admit, géra la manœuvre habilement -- mais c'était précalculé, n'importe qui aurait pu le faire~; bon sang, Ellie aurait pu être assise là. Ou Chipsie.

Max eut le premier quart. Simes le quitta après lui avoir enjoint de ne pas dévier du programme sans l'appeler d'abord. Une heure plus tard Kovak releva Max temporairement et Max se hâta vers le sas des passagers. Il y avait cinq porteurs honoraires, le Capitaine, M. Walther, Simes, Max et Kelly. Derrière eux, encombrant les passages, se trouvaient les officiers et la plupart de l'équipage. Max ne vit aucun passager.

La porte intérieure du sas fut ouverte~; deux aides-stewards portèrent le corps et le placèrent contre la porte extérieure. Max fut soulagé de voir qu'il avait été enveloppé dans un linceul le couvrant complètement. Ils fermèrent la porte intérieure et se retirèrent. Le Capitaine se tenait face à la porte, avec Simes et le Premier Officier montant la garde d'un côté de la porte et, de l'autre côté leur faisant face, Max et Kelly.

Le Capitaine lança un seul mot par-dessus son épaule~: \og Pression~! \fg{}

Derrière se tenait Bennett portant un téléphone portable~; il transmit le mot à la salle des machines. Le manomètre au-dessus de la porte du sas indiquait une atmosphère~; maintenant il commença à monter. Le Capitaine sortit un petit livre de sa poche et commença à lire le service des morts.

Sentant qu'il ne pourrait pas supporter d'écouter, Max regarda le manomètre. Régulièrement il grimpait. Max réfléchit que le vaisseau avait déjà dépassé la vitesse de libération pour le système de Nu Pegasi avant qu'il n'ait été relevé~; le corps prendrait une orbite ouverte. La jauge atteignit dix atmosphères~; le Capitaine Blaine ferma son livre.

\og Prévenez les passagers, \fg{} dit-il à Bennett.

Bientôt les haut-parleurs annoncèrent~: \og Tout l'équipage~! Tous les passagers~! Le vaisseau sera en chute libre pendant trente secondes. Ancrez-vous et ne changez pas de position. \fg{}

Max tendit le bras derrière lui, trouva l'une des nombreuses poignées toujours présentes autour d'un sas et tira vers le bas pour que sa prise maintienne ses pieds en contact avec le pont. Une sirène d'avertissement hurla -- puis soudain il fut en apesanteur quand la poussée du vaisseau et le champ de gravité artificielle anormale furent tous deux coupés.

Il entendit le Capitaine dire d'une voix forte et ferme~: \og "Cendres aux cendres, poussière à la poussière." Que le corps soit jeté au loin. \fg{}

Le manomètre tomba soudain à zéro et le Dr Hendrix fut lancé dans l'espace, là pour errer parmi les étoiles pour l'éternité. Max sentit le poids revenir quand la salle des machines les ramena à la normale du vaisseau. Le manomètre montra une pression qui remontait graduellement. Les gens se détournèrent et partirent, leurs voix murmurant bas. Max monta et prit le quart.

\bigskip

Le matin suivant Simes emménagea dans la cabine du Dr Hendrix. Il y eut des problèmes avec le Premier Officier Walther à ce sujet -- Max n'entendit que des rapports de troisième main -- mais le Capitaine donna raison à Simes~; il resta dans les quartiers de l'Astrogateur.

Le Trou aux Soucis s'installa dans une routine pas très différente de ce qui s'était passé avant, sauf que la personnalité de Simes imprégnait tout. Il n'y avait jamais eu de liste de quarts affichée auparavant~; Kelly avait toujours assigné les membres d'équipage et le Docteur avait simplement informé oralement les officiers de quart de ses souhaits. Maintenant une liste dactylographiée apparut~:

\begin{center}
\begin{tabular}{l}
PREMIER QUART \\
Randolph Simes, Astrogateur \\
\\
DEUXIÈME QUART \\
Capitaine Blaine \\
(M. Jones, apprenti intérimaire, en instruction) \\
\\
TROISIÈME QUART \\
Kelly, Ch. Calc. \\
\\
\emph{(signé)} Randolph Simes, Astrogateur
\end{tabular}
\end{center}

En dessous se trouvait une liste de quatre quarts pour les membres d'équipage, également signée par Simes.

Max la regarda et haussa les épaules. Il était évident que Simes lui en voulait, bien qu'il ne pût comprendre pourquoi. Il était également évident que Simes n'avait pas l'intention de le laisser faire de l'astrogation et que les chances de Max d'être accepté à temps comme frère à part entière avaient maintenant, avec la mort du Dr Hendrix, sombré à zéro. À moins, bien sûr, que le Capitaine Blaine ne passe outre à Simes et ne force un rapport favorable, ce qui était extrêmement improbable.

Max recommença à penser à partir avec Sam à Nova Terra. Eh bien, en attendant il tiendrait ses quarts et essaierait de rester hors d'ennuis. Voilà tout.

Il n'y avait qu'une seule transition à faire entre Halcyon et Nova Terra, un saut de quatre-vingt-dix-sept années-lumière trois semaines après Halcyon à une poussée de dix-sept gravités -- la poussée dépendait toujours de la distance entre l'étoile et le portail, puisque le but était d'y arriver juste en dessous de la vitesse de la lumière. Le Trou aux Soucis resta sur un quart sur trois pour les officiers et un sur quatre pour les membres d'équipage pendant les deux premières semaines.

Le Capitaine Blaine se montrait à chaque quart mais semblait tout à fait disposé à laisser Max accomplir les légères tâches de cette portion de l'étape. Il donnait peu d'instruction -- quand il le faisait, il avait tendance à s'égarer dans des anecdotes, amusantes mais pas utiles. Max essaya de continuer son propre entraînement, effectuant le calcul de routine du milieu de quart comme si c'était l'affaire frénétique que ça aurait été près de la transition. Le Capitaine Blaine le regarda, puis dit doucement~: \og Ne vous mettez pas dans un tel état, fiston. Programmez toujours sur papier quand c'est possible -- toujours. Et prenez le temps de vérifier. Se presser cause des erreurs. \fg{}

Max ne dit rien, pensant au Dr Hendrix, mais obéit aux ordres. À la fin de son premier quart sous le Capitaine, Max signa le journal comme d'habitude. Quand Simes vint prendre le quart quatre heures plus tard, Max fut tiré du lit et sommé de se présenter à la salle de contrôle.

Simes pointa le journal du doigt. \og C'est quoi l'idée, Monsieur~? \fg{}

\og De quoi, monsieur~? \fg{}

\og De signer le journal. Vous n'étiez pas officier de quart. \fg{}

\og Eh bien, monsieur, le Capitaine semblait s'y attendre. J'ai signé beaucoup de journaux et il les a toujours approuvés par le passé. \fg{}

\og Hmm -- je parlerai au Capitaine. Descendez. \fg{}

À la fin de son quart suivant, n'ayant reçu aucune instruction, Max prépara le journal et l'apporta au Capitaine. \og Monsieur~? Voulez-vous signer ceci~? Ou dois-je~? \fg{}

\og Hein~? \fg{} Blaine le regarda. \og Oh, je suppose que je ferais mieux. Laissez toujours un chef de département faire les choses à sa façon si possible. Souvenez-vous de ça quand vous serez capitaine, fiston. \fg{} Il le signa.

Cela régla la question jusqu'à ce que le Capitaine prenne l'habitude de ne pas être là, d'abord pour de courtes périodes, puis pour plus longtemps. Vint le moment où il était absent à la fin du quart~; Max téléphona à M. Simes. \og Monsieur, le Capitaine n'est pas là. Que voulez-vous que je fasse~? \fg{}

\og Et alors~? C'est son privilège de quitter la salle de contrôle. \fg{}

\og Mais Kelly est prêt à prendre la relève et le journal n'est pas signé. Dois-je le signer~? Ou dois-je lui téléphoner~? \fg{}

\og Lui téléphoner~? Bon sang de bonsoir, non~! Vous êtes fou~? \fg{}

\og Quels sont vos ordres, monsieur~? \fg{}

Simes resta silencieux, puis répondit~: \og Imprimez son nom, puis signez en dessous "Par ordre" -- et à l'avenir utilisez votre tête. \fg{}

Ils passèrent au quart-et-quart pour la dernière semaine. Max continua sous le Capitaine~; Kelly assistait Simes. Une fois le changement effectué, Blaine devint méticuleux à propos d'être présent dans la salle de contrôle et, quand Max commença à faire le premier calcul, le poussa doucement de côté. \og Je ferais mieux de prendre la relève, petit. Nous approchons maintenant. \fg{}

Alors Max l'assista -- et prit horriblement conscience que le Capitaine n'était plus l'homme qu'il avait dû être autrefois. Sa connaissance de la théorie était solide et il connaissait tous les raccourcis -- mais son esprit avait tendance à vagabonder. Deux fois en un seul calcul Max dut lui rappeler diplomatiquement des détails. Pourtant le Vieux semblait inconscient de cela, était tout à fait enjoué.

Cela continua ainsi. Max commença à prier pour que le Capitaine laisse le nouvel Astrogateur faire la transition lui-même -- bien qu'il méprisât Simes. Il voulait discuter de ses inquiétudes avec Kelly -- il n'y avait personne d'autre avec qui cela aurait été possible -- mais Kelly était au quart opposé avec Simes. Il n'y avait rien à faire sinon s'inquiéter.

Quand le dernier jour arriva, il découvrit que le Capitaine Blaine n'avait l'intention ni de faire passer le vaisseau lui-même ni de laisser Simes le faire~; il avait son propre système. Quand ils furent tous dans le Trou aux Soucis, le Capitaine dit~: \og Je veux vous montrer à tous une astuce qui enlève la tension de l'astrogation. Sans vouloir manquer de respect à notre cher frère, le Dr Hendrix, bien qu'il fût un grand astrogateur, le meilleur qui soit -- néanmoins il travaillait trop dur. Maintenant voici une méthode que m'a enseignée mon propre maître. Kelly, si vous voulez bien faire sortir les commandes à distance, s'il vous plaît. \fg{}

Il les fit s'asseoir en demi-cercle, lui-même, Simes et Max, autour de la selle de l'ordinateur, avec Kelly dans la selle. Chacun d'eux était armé de formulaires de programmation et le Capitaine Blaine tenait les interrupteurs de commande à distance sur ses genoux.

\og Maintenant l'idée est que chacun de nous travaille une visée à tour de rôle, d'abord moi, puis M. Simes, puis M. Jones. De cette façon nous maintenons le flux de données sans tension. Très bien, les gars, commencez. Postes de transition pour tout le monde. \fg{}

Ils firent une répétition à blanc, puis le Capitaine se leva. \og Appelez-moi, M. Simes, deux heures avant la transition. Je crois que vous et M. Jones trouverez que cette méthode vous donne assez de repos entre-temps. \fg{}

\og Oui, monsieur. Mais Capitaine -- puis-je faire une suggestion~? \fg{}

\og Hein~? Certainement, monsieur. \fg{}

\og C'est un système excellent, mais je suggère que Kelly soit mis dans le groupe d'astrogation à la place de Jones. Jones n'est pas expérimenté. Nous pouvons mettre Kovak dans la selle et Lundy au livre. \fg{}

Blaine secoua la tête. \og Non. La précision est tout, monsieur, donc nous devons avoir notre meilleur opérateur à l'ordinateur. Quant à M. Jones, c'est ainsi qu'il doit acquérir de l'expérience -- s'il perd ses moyens, vous et moi pouvons toujours le remplacer. \fg{} Il commença à partir, puis ajouta~: \og Mais Kovak peut alterner avec Kelly jusqu'à mon retour. Il ne faut pas que quelqu'un se fatigue, c'est comme ça qu'on fait des erreurs. \fg{}

\og Bien, monsieur. \fg{}

Simes ne dit rien de plus à Max. Ils commencèrent à traiter les visées, alternativement, utilisant la programmation écrite sur des formulaires imprimés. Les visées arrivaient selon un horaire de vingt minutes, donnant à chacun d'eux quarante minutes pour un problème s'il le désirait. Max commença à penser que la méthode du Capitaine avait ses mérites. Certainement le Dr Hendrix s'était tué à la tâche -- les vaisseaux ne s'usaient pas mais les hommes oui.

Il avait largement le temps de travailler non seulement ses propres problèmes, mais aussi ceux de Simes. Les données sortaient oralement et rien n'empêchait Max de programmer les visées de Simes dans sa tête et de vérifier ce qui entrait dans l'ordinateur. Pour autant qu'il pût voir, Simes s'en sortait bien -- bien que bien sûr il n'y eût pas de vraie tension, pas encore.

Ils mangèrent des sandwichs et burent du café où ils étaient assis, ne quittant leurs places que pour cinq minutes environ à la fois. Le Capitaine Blaine se montra vingt minutes en avance. Il sourit et dit gaiement~: \og Tout le monde heureux et détendu~? Maintenant nous passons aux choses sérieuses. J'ai juste le temps pour une tasse de café. \fg{}

Quelques minutes plus tard il s'assit et prit les interrupteurs de contrôle des mains de Simes. Les visées arrivaient maintenant selon un horaire de dix minutes, encore amplement de temps. Max continua à toutes les traiter, les siennes sur papier et les autres dans sa tête. Il avait toujours fini à temps pour attraper les données de la visée suivante, les programmer mentalement et vérifier les traductions pendant que Lundy feuilletait le livre. Cela lui donnait une image continue de la précision avec laquelle ils étaient dans le sillon, de combien de tâtonnements ils devaient faire en approchant leur cible invisible.

Il lui semblait que Simes avait tendance à sur-corriger et que le Capitaine sous-corrigeait avec un certain optimisme, mais aucun des deux n'était assez loin pour mettre le vaisseau en danger. Peut-être se trompait-il au sujet du Capitaine -- le Vieux semblait se stabiliser quand ça comptait. Ses propres corrections, il fut content de voir, le Capitaine les appliquait sans question.

Après plus d'une heure, avec la transition à quarante-cinq minutes, le Capitaine Blaine leva les yeux et dit~: \og Très bien, les gars, nous approchons. Envoyez-les-nous aussi vite que vous pouvez maintenant. \fg{}

Smythe et Kovak, avec Noguchi et Bennett qui couraient pour eux, passèrent à la vitesse supérieure~; les données se déversèrent en un flot continu. Max continua à travailler chaque visée, programmant les siennes dans sa tête et dictant des chiffres plus vite qu'il ne les écrivait. Il remarqua que Simes transpirait, effaçant parfois et recommençant. Mais les chiffres que Simes dictait correspondaient à ce que Max pensait qu'ils devraient être, d'après sa propre programmation mentale. Le Capitaine Blaine semblait détendu, bien qu'il n'eût pas matériellement accéléré et utilisait parfois encore l'ordinateur quand Max était prêt à y verser sa visée.

À un moment Simes parla trop vite, avalant ses chiffres. Lundy dit promptement~: \og Répétez, monsieur~! \fg{}

\og Bon sang~! Débouchez-vous les oreilles~! \fg{} Mais Simes répéta. Le Capitaine leva les yeux, puis se pencha de nouveau sur son propre problème.

Dès que l'ordinateur fut libre, le Capitaine Blaine dicta ses propres chiffres à Lundy. Max avait déjà posé la visée du Capitaine dans son esprit, écoutait inconsciemment tout en observant Simes. Une sonnette d'alarme retentit dans son esprit.

\og Capitaine~! Je ne vous confirme pas~! \fg{}

Le Capitaine Blaine s'arrêta. \og Hein~? \fg{}

\og Ce programme est faux, monsieur. \fg{}

Le Capitaine ne sembla pas en colère. Il tendit simplement sa planchette de programmation à Simes. \og Vérifiez-moi, monsieur. \fg{}

Simes jeta un rapide coup d'œil aux chiffres. \og Je vous confirme, monsieur~! \fg{}

Blaine dit~: \og Laissez tomber, Jones. M. Simes et moi finirons. \fg{}

\og Mais -- \fg{}

\og Laissez tomber, Monsieur~! \fg{}

Max sortit du cercle, bouillant intérieurement. La vérification de Simes du programme du Capitaine n'avait rien signifié, à moins que Simes n'ait écouté et retenu (comme Max l'avait fait) les données au fur et à mesure qu'elles arrivaient. Le Capitaine avait transposé un huit et un trois aux cinquième et sixième décimales -- le programme semblerait correct à moins qu'on ne connaisse les bons chiffres. Si Simes s'était même donné la peine de le vérifier, ajouta-t-il amèrement.

Mais Max ne pouvait s'empêcher de noter et de traiter les données dans son esprit. La prochaine visée de Simes devrait attraper l'erreur du Capitaine~; sa correction devrait la réparer. Ce serait une grosse correction, Max le savait~; voyageant juste sous la vitesse de la lumière, le vaisseau avalait un million de miles en moins de six secondes.

Max put voir Simes hésiter quand les lumières de sa prochaine visée s'allumèrent sur l'ordinateur et que Lundy les traduisit. Ma parole, l'homme avait l'air effrayé~! La correction demandée pousserait le vaisseau extrêmement près de la vitesse critique -- Simes hésita, puis ordonna moins de la moitié de la quantité que Max croyait nécessaire.

Blaine l'appliqua et continua avec son problème suivant. Quand la réponse sortit, l'erreur, multipliée par le temps et la vitesse impensable, était plus flagrante que jamais. Le Capitaine jeta à Simes un regard d'étonnement, puis fit promptement une correction. Max ne put dire ce qu'elle était, puisqu'elle fut faite sans mots au moyen de l'interrupteur sur ses genoux.

Simes se lécha les lèvres sèches. \og Capitaine~? \fg{}

\og Le temps juste pour une dernière visée, \fg{} répondit Blaine. \og Je la prendrai moi-même, M. Simes. \fg{}

Les données lui furent transmises, il commença à poser son problème sur le formulaire. Max le vit effacer, puis lever les yeux~; Max suivit son regard. Le préréglage sur le chronomètre au-dessus de l'ordinateur montrait les secondes qui s'écoulaient.

\og Tenez-vous prêts~! \fg{} annonça Blaine.

Max leva les yeux. Les étoiles faisaient ce rapprochement rampant qui marquait les derniers instants avant la transition. Le Capitaine Blaine devait avoir appuyé sur le second interrupteur, celui qui les ferait basculer, pendant que Max regardait, car les étoiles clignotèrent soudain et furent remplacées instantanément par un autre firmament étoilé, d'apparence normale.

Le Capitaine se renversa en arrière, leva les yeux. \og Eh bien, \fg{} dit-il joyeusement, \og il semble que nous ayons réussi encore une fois. \fg{} Il se leva et se dirigea vers l'écoutille, disant par-dessus son épaule~: \og Appelez-moi quand vous nous aurez mis dans le sillon, M. Simes. \fg{} Il disparut par l'écoutille.

Max leva de nouveau les yeux, essayant de se rappeler des cartes qu'il avait étudiées quelle partie de ce nouveau ciel ils avaient devant eux. Kelly regardait aussi en l'air.

\og Oui, nous sommes passés, \fg{} l'entendit marmonner Max. \og Mais \emph{où}~? \fg{}

Simes aussi avait regardé le ciel. Maintenant il se retourna avec colère. \og Qu'est-ce que vous voulez dire~? \fg{}

\og Ce que j'ai dit, \fg{} insista Kelly. \og Ce n'est pas un ciel que j'aie jamais vu avant. \fg{}

\og Absurdités, mon vieux~! Vous ne vous êtes simplement pas orienté. Tout le monde sait qu'un morceau de ciel peut sembler étrange quand on le regarde pour la première fois. Sortez les cartes plates pour cette zone~; nous trouverons nos repères assez vite. \fg{}

\og Elles sont sorties, monsieur. Noguchi. \fg{}

Il ne fallut que quelques minutes pour convaincre tous les autres dans la salle de contrôle que Kelly avait raison, un peu plus longtemps pour convaincre même Simes. Il finit par lever les yeux des cartes avec un visage blanc verdâtre.

\og Pas un mot à quiconque, \fg{} dit-il. \og C'est un ordre -- et je casserai tout homme qui bavarde. Kelly, prenez le quart. \fg{}

\og Bien, monsieur. \fg{}

\og Je serai dans la cabine du Capitaine. \fg{}

Il descendit pour dire à Blaine que l'\emph{Asgard} avait émergé dans un espace inconnu -- était perdu.
