\chapter{Earthport}

Le chemin de ferme passait sous l'autoroute de fret~; Max ressortit de l'autre côté et se dirigea vers le sud en longeant l'autoroute. L'itinéraire était jalonné de panneaux \og PROPRIÉTÉ PRIVÉE \fg{} mais le sentier était bien usé. L'autoroute s'élargissait pour faire place à une bande de décélération. Au bout de sa surface lisse, à un kilomètre et demi de là, Max pouvait voir le restaurant que Sam avait mentionné.

Il escalada la clôture entourant le restaurant et le parking et se dirigea vers les emplacements de stationnement où une douzaine des grands navires terrestres étaient alignés. L'un d'eux vibrait, prêt au départ, son fond plat à quelques centimètres du sol métallique. Max s'approcha de l'avant et leva les yeux vers le compartiment du conducteur. La porte était ouverte et il pouvait voir le chauffeur devant son tableau de bord.

Max appela~: \og Hé, monsieur~! \fg{}

Le chauffeur passa la tête. \og Qu'est-ce qui te gratte~? \fg{}

\og Est-ce que je pourrais avoir un transport vers le sud~? \fg{}

\og Dégage, gamin. \fg{} La porte claqua.

Aucun des autres camions n'était soulevé du sol~; leurs compartiments de contrôle étaient vides. Max allait se détourner quand un autre géant glissa sur la bande de freinage, atteignit le parking, rampa lentement dans un emplacement et se posa au sol. Il envisagea d'approcher son conducteur, mais décida d'attendre que l'homme ait mangé.

Il retourna vers le bâtiment du restaurant et regardait par la porte, observant des hommes affamés engloutir de la nourriture tandis que sa bouche salivait, quand il entendit une voix agréable derrière lui.

\og Excusez-moi, mais vous bloquez la porte. \fg{}

Max bondit de côté. \og Oh~! Désolé. \fg{}

\og Allez-y. Vous étiez là avant. \fg{}

L'homme qui parlait avait une dizaine d'années de plus que Max. Il était couvert de taches de rousseur et avait un sourire en coin. Max vit sur sa casquette l'insigne de la Guilde des Camionneurs.

\og Entrez \fg{}, répéta l'homme, \og avant de vous faire piétiner par la ruée. \fg{}

Max s'était dit qu'il pourrait tomber sur Sam à l'intérieur --- et après tout, ils ne pouvaient pas lui faire payer juste pour entrer, s'il ne \textit{mangeait} rien vraiment. Sous-jacente était l'idée de demander à travailler pour un repas, si le gérant avait l'air sympathique. L'insistance du rouquin fit pencher la balance~; il suivit son nez vers la source des odeurs célestes qui sortaient par la porte.

Le restaurant était bondé~; il y avait une seule table libre, pour deux. L'homme se glissa dans une chaise et dit~: \og Asseyez-vous. \fg{} Quand Max hésita, il ajouta~: \og Allez, posez vos fesses. Je n'aime pas manger seul. \fg{}

Max sentait les yeux du gérant sur lui~; il s'assit. Une serveuse leur tendit à chacun un menu et le camionneur la détailla avec appréciation. Quand elle partit, il dit~: \og Ce boui-boui avait autrefois un service automatique --- et il a fait faillite. La clientèle est partie au \textit{Tivoli}, cent trente kilomètres plus bas sur la route. Alors le nouveau propriétaire a jeté les machines et engagé des filles, et les affaires ont repris. Rien ne fait paraître la nourriture meilleure que d'avoir une jolie fille qui vous la sert. Pas vrai~? \fg{}

\og Euh, j'imagine. Sûr. \fg{} Max n'avait pas entendu ce qui s'était dit. Il avait rarement été dans un restaurant et seulement au comptoir du Carrefour de Clyde. Les prix qu'il lisait l'effrayaient~; il voulait se glisser sous la table.

Son compagnon le regarda. \og Quel est le problème, l'ami~? \fg{}

\og Problème~? Euh, rien. \fg{}

\og Tu es fauché~? \fg{}

L'expression misérable de Max lui répondit. \og Bah, j'y suis passé moi aussi. Détends-toi. \fg{} L'homme fit signe à la serveuse. \og Viens ici, ma jolie. Mon associé et moi, on prendra chacun un steak du petit-déjeuner avec un œuf au plat posé dessus et des accompagnements. Je veux cet œuf à peine mort. S'il est trop cuit, je le clouerai au mur comme avertissement aux autres. Compris~? \fg{}

\og Je doute que vous arriviez à y faire passer un clou \fg{}, rétorqua-t-elle et elle s'éloigna, se balançant doucement.

Le camionneur la suivit des yeux jusqu'à ce qu'elle disparaisse dans la cuisine. \og Tu vois ce que je veux dire~? Comment les machines peuvent-elles rivaliser~? \fg{}

Le steak était bon et l'œuf n'était pas figé. Le camionneur dit à Max de l'appeler \og Red \fg{} et Max donna son nom en échange. Max chassait le dernier du jaune avec un morceau de toast et se demandait si c'était le moment d'aborder le sujet du transport quand Red se pencha en avant et parla doucement.

\og Max --- tu as quelque chose qui te pousse~? Libre de prendre un boulot~? \fg{}

\og Quoi~? Eh bien, peut-être. C'est quoi~? \fg{}

\og Ça te dérangerait de faire un petit voyage vers le sud-ouest~? \fg{}

\og Le sud-ouest~? En fait, j'allais justement par là. \fg{}

\og Bien. Voilà le deal. Le patron dit qu'on doit avoir deux camionneurs par engin --- ou alors faire une pause de huit heures après huit heures de conduite. Je ne peux pas~; j'ai un délai à respecter sous peine de pénalité --- et mon coéquipier a fait faux bond. L'abruti s'est fait prendre à picoler et j'ai dû le mettre au frais. Maintenant j'ai un point de contrôle à passer à deux cents kilomètres d'ici. Ils me feront attendre si je ne peux pas montrer un autre chauffeur. \fg{}

\og Zut alors~! Mais je ne sais pas conduire, Red. Je suis vraiment désolé. \fg{}

Red fit un geste avec sa tasse. \og Tu n'auras pas à le faire. Tu seras toujours le chauffeur de repos. Je ne confierais pas ma petite \textit{Molly Malone} à quelqu'un qui ne connaît pas ses manières. Je me tiendrai éveillé avec des pilules stimulantes et je rattraperai le sommeil à Earthport. \fg{}

\og Tu vas jusqu'à \textit{Earthport}~? \fg{}

\og Exact. \fg{}

\og C'est d'accord~! \fg{}

\og Okay, voilà le topo. Chaque fois qu'on passe un point de contrôle, tu es dans la couchette, endormi. Tu m'aides à charger et décharger --- j'ai un chargement partiel et un ramassage à Oklahoma City --- et je te nourris. D'accord~? \fg{}

\og D'accord~! \fg{}

\og Alors allons-y. Je veux démarrer avant que ces autres conducteurs se mettent en route. On ne sait jamais, il pourrait y avoir un inspecteur. \fg{} Red jeta un billet et n'attendit pas la monnaie.

La \textit{Molly Malone} faisait soixante mètres de long et était profilée de telle sorte qu'elle avait une portance négative en croisière. Max s'en rendit compte en observant les instruments~; quand elle se mit à vibrer et à s'élever, le cadran marqué GARDE AU SOL indiquait vingt-cinq centimètres, mais à mesure qu'ils prenaient de la vitesse sur la bande d'accélération, il diminua à quinze.

\og La répulsion fonctionne selon une loi en cube inverse \fg{}, expliqua Red. \og Plus le vent nous pousse vers le bas, plus fort la route nous repousse vers le haut. Ça nous empêche de sauter par-dessus l'horizon. Plus on va vite, plus on est stable. \fg{}

\og Et si tu allais si vite que la pression du vent forçait le fond jusqu'à la route~? Tu pourrais t'arrêter assez tôt pour éviter de tout casser~? \fg{}

\og Réfléchis. Plus on s'aplatit, plus fort on est repoussé vers le haut --- cube inverse, j'ai dit. \fg{}

\og Oh. \fg{} Max sortit la règle à calcul de son oncle. \og Si elle supporte juste son propre poids à vingt-cinq centimètres de garde, alors à dix centimètres la répulsion serait de vingt-sept fois son poids et à trois centimètres ce serait sept cent vingt-neuf, et à moins d'un centimètre\ldots \fg{}

\og N'y pense même pas. À vitesse maximale, je n'arrive pas à la descendre en dessous de douze centimètres. \fg{}

\og Mais qu'est-ce qui la fait avancer~? \fg{}

\og C'est une relation de phase. Le champ rampe vers l'avant et Molly essaie de le rattraper --- sauf qu'elle n'y arrive pas. Ne me demande pas la théorie, je pousse juste les boutons. \fg{} Red alluma une cigarette et se renversa en arrière, une main sur la barre. \og Tu ferais mieux d'aller dans la couchette, gamin. Point de contrôle dans soixante-cinq kilomètres. \fg{}

La couchette était une tablette transversale derrière le compartiment de contrôle. Max y grimpa et s'enroula dans une couverture. Red lui tendit une casquette. \og Enfonce-la sur tes yeux. Laisse le bouton visible. \fg{}

Le bouton était un écusson de camionneur. Max fit ce qu'on lui disait. Bientôt il entendit le bruit du vent passer d'un rugissement sourd à un soupir, puis s'arrêter. Le camion se posa sur la chaussée et la porte s'ouvrit. Il resta immobile, incapable de voir ce qui se passait. Une voix étrangère dit~: \og Depuis combien de temps tu conduis~? \fg{}

\og Depuis le petit-déjeuner chez Tony. \fg{}

\og Ah oui~? Comment tes yeux sont-ils si injectés de sang~? \fg{}

\og C'est la vie dissolue que je mène. Tu veux voir ma langue~? \fg{}

L'inspecteur ignora cela et dit à la place~: \og Ton coéquipier n'a pas signé son quart. \fg{}

\og Si tu le dis. Tu veux que je réveille cet abruti~? \fg{}

\og Hmm\ldots laisse tomber. Signe pour lui. Dis-lui d'être plus prudent. \fg{}

\og D'accord. \fg{}

La \textit{Molly Malone} démarra et prit de la vitesse. Max redescendit. \og J'ai cru qu'on était fichus quand il a demandé ma signature. \fg{}

\og C'était fait exprès \fg{}, dit Red avec mépris. \og Il faut leur donner quelque chose sur quoi grogner, ou ils vont chercher autre chose. \fg{}

Max aimait le camion. La vitesse formidable si près du sol l'exaltait~; il décida que s'il ne pouvait pas être astronaute, cette vie ne serait pas mal --- il se renseignerait sur le montant des frais de candidature et commencerait à économiser.

Il aimait la façon décontractée dont Red repérait sur la chaussée la ligne de vitesse correspondant à celle de la \textit{Molly} puis engageait le gros engin dans un virage. C'était généralement la ligne la plus extérieure, avec la \textit{Molly} sur le côté et l'horizon incliné à un angle fou.

Près d'Oklahoma City, ils passèrent sous les guides de la C.S.\&E. juste au moment où un train passait --- le \textit{Razor}, selon les calculs de Max.

\og J'avais l'habitude de conduire ces trucs \fg{}, remarqua Red en levant les yeux.

\og \textit{Vraiment}~? \fg{}

\og Ouaip. Mais ils ont commencé à m'inquiéter. Je détestais ça chaque fois que je faisais un saut et sentais le poids se dérober sous moi. Puis j'ai eu l'idée que le train avait une volonté propre et attendait juste de dévier au lieu d'entrer dans l'anneau suivant. Ce genre de chose n'est pas bon. Alors j'ai trouvé un camionneur qui voulait s'élever dans la société et j'ai payé l'amende aux deux guildes pour qu'on puisse échanger. Je ne l'ai jamais regretté. Trois cents kilomètres à l'heure quand on est près du sol, c'est suffisant. \fg{}

\og Euh, et les vaisseaux spatiaux~? \fg{}

\og Ça c'est autre chose. De la place là-haut. Dis, gamin, pendant que tu es à Earthport, tu devrais jeter un coup d'œil aux gros bébés. Ils valent le détour. \fg{}

Le livre de la bibliothèque lui brûlait le sac à dos~; à Oklahoma City, il remarqua une boîte postale au dépôt de fret et, sur une impulsion, y déposa le livre. Après l'avoir envoyé, il eut un pincement d'inquiétude à l'idée qu'il aurait pu donner un indice sur sa position qui remonterait jusqu'à Montgomery, mais il réprima l'inquiétude --- le livre \textit{devait} être rendu. Le vagabondage aux yeux de la loi ne l'avait pas inquiété, ni l'intrusion, ni l'usurpation de l'identité d'un camionneur licencié --- mais voler un livre était un péché.

Max dormait dans la couchette quand ils arrivèrent. Red le secoua. \og Terminus, gamin. \fg{}

Max s'assit en bâillant. \og On est où~? \fg{}

\og Earthport. Remuons-nous et déchargeons cette demoiselle. \fg{}

Il était deux heures après le lever du soleil et la chaleur du désert montait quand ils eurent fini de vider la \textit{Molly}. Red lui offrit un dernier repas.

Red termina le premier, paya, puis posa un billet près de l'assiette de Max. \og Merci, gamin. C'est pour la chance. Salut. \fg{} Il était parti avant que Max ait refermé la bouche. Il n'avait jamais appris le nom de son ami, ne connaissait même pas son numéro d'écusson.

Earthport était de loin le plus grand lieu habité que Max eût jamais vu et tout le déroutait --- les foules pressées et égocentrées, les bâtiments énormes, les trottoirs roulants à la place des rues, le bruit, le soleil du désert qui tapait, la platitude --- mais il n'y avait rien qu'on puisse appeler une colline plus près que l'horizon~!

Il vit son premier extra-terrestre, un natif d'Epsilon Gémini~V de deux mètres quarante, sortant d'une boutique avec un paquet sous ses bras gauches --- aussi décontracté, pensa Max, qu'un fermier faisant ses courses de la semaine au Carrefour.

Max le fixa. Il savait ce que c'était d'après les images et les émissions de SV, mais en voir un était autre chose. Ses yeux multiples, comme une couronne de raisins jaunes autour de la tête, lui donnaient une apparence grotesquement sans visage. Max tourna la tête pour le suivre.

La créature s'approcha d'un policier, tapota le dessus de sa casquette et dit~: \og Egscusez-moi, monschieur, mais bouvez-vous m'indiquer le Tesert Balms Athletic Club~? \fg{}

Max ne put dire d'où sortait le son.

Max finit par remarquer qu'il semblait être le seul à fixer, alors il marcha lentement, tout en jetant des coups d'œil par-dessus son épaule --- ce qui eut pour résultat qu'il bouscula un inconnu.

\og Oh, excusez-moi~! \fg{} balbutia Max.

L'inconnu le regarda. \og Du calme, cousin. Tu es dans la grande ville maintenant. \fg{}

Après cela, il essaya de faire attention.

Il avait eu l'intention de chercher immédiatement le Siège de la Guilde du Chapitre Mère des Astrogateurs dans le faible espoir que même sans ses livres et sa carte d'identité, il puisse encore s'identifier et découvrir que l'oncle Chet avait pourvu à son avenir. Mais il y avait tant à voir qu'il traîna.

Il se retrouva bientôt devant l'Imperial House, l'hôtel qui garantissait de fournir n'importe quelle combinaison de pression, température, éclairage, atmosphère, pseudo-gravitation et régime alimentaire préférée par n'importe quelle race connue de créatures intelligentes. Il resta là à espérer voir quelques-uns des clients, mais le seul qui sortit pendant qu'il était là fut roulé dehors dans un réservoir de transport pressurisé et il ne put voir à l'intérieur.

Il remarqua le garde policier à la porte qui l'observait et commença à partir --- puis décida de demander son chemin, raisonnant que si c'était acceptable pour un Géminien de questionner un policier, ça devait certainement l'être pour un être humain. Il se retrouva à citer l'extra-terrestre.

\og Excusez-moi, monsieur, mais pourriez-vous m'indiquer le Siège de la Guilde des Astrogateurs~? \fg{}

L'agent l'examina. \og Au bout de l'Avenue des Planètes, juste avant d'arriver au port. \fg{}

\og Euh, dans quelle direction\ldots \fg{}

\og Nouveau en ville~? \fg{}

\og Ouais. Oui, monsieur. \fg{}

\og Où loges-tu~? \fg{}

\og Loger~? Eh bien, nulle part encore. Je viens d'arriver. Je\ldots \fg{}

\og Qu'est-ce que tu as à faire au Siège des Astrogateurs~? \fg{}

\og C'est à cause de mon oncle \fg{}, répondit Max misérablement.

\og Ton oncle~? \fg{}

\og Il\ldots il est astrogateur. \fg{} Il croisa mentalement les doigts sur le temps du verbe.

Le policier l'inspecta de nouveau. \og Prends ce trottoir roulant jusqu'au prochain croisement, change et va vers l'ouest. Grand bâtiment avec le soleil éclatant de la guilde au-dessus de la porte --- tu ne peux pas le manquer. Reste en dehors des zones interdites. \fg{}

Max partit sans attendre de découvrir comment il était censé reconnaître une zone interdite.

Le Siège de la Guilde s'avéra effectivement facile à trouver~; le trottoir roulant vers l'ouest plongeait sous terre et quand il refit surface à son terminus, Max fut déposé juste devant. Mais il n'avait pas d'yeux pour le bâtiment. À l'ouest, là où l'avenue et les bâtiments se terminaient, c'était le terrain d'aviation et dessus des vaisseaux spatiaux, s'étendant sur des kilomètres --- de petites flèches militaires rapides, des navettes lunaires trapues, des vaisseaux ailés qui desservaient les stations satellites, des cargos robots, sans grâce mais puissants.

Mais directement devant le portail, à moins d'un kilomètre, se trouvait un grand vaisseau qu'il reconnut immédiatement, le vaisseau stellaire \textit{Asgard}.

Il connaissait son histoire~; l'oncle Chet y avait servi. Cent ans plus tôt, il avait été construit dans l'espace comme vaisseau fusée spatial~; il s'appelait alors le \textit{Prince of Wales}. Les années passèrent, ses tubes furent arrachés et une torche à conversion de masse y fut allumée~; il devint l'\textit{Einstein}. D'autres années passèrent~; pendant près de vingt ans, il orbita vide autour de Luna, une épave sans vie et obsolète. Maintenant, à la place de la torche, il avait des propulseurs Horst-Conrad qui s'accrochaient au tissu même de l'espace~; grâce à eux, il pouvait désormais toucher la Terre Mère. Pour commémorer sa renaissance, il avait été rebaptisé \textit{Asgard}, demeure céleste des dieux.

Son corps massif en forme de poire était posé sur sa partie la plus petite, stabilisé par un échafaudage invisible de faisceaux de poussée. Max savait où ils devaient être, car il y avait un cercle de barrières autour de lui pour empêcher les imprudents de s'aventurer dans les zones mortelles.

Il pressa son nez contre le portail du terrain et essaya d'en voir davantage, jusqu'à ce qu'une voix crie~: \og Éloigne-toi de là, toi~! Tu ne vois pas ce panneau~? \fg{}

Max leva les yeux. Au-dessus de sa tête il y avait un panneau~: ZONE INTERDITE.

À contrecœur, il s'éloigna et retourna vers le Siège de la Guilde.
