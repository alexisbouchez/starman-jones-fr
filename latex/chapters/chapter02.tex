\chapter{Le bon Samaritain}

Il aurait aimé avoir une lampe, mais son absence ne le gênait pas trop. Il connaissait ce pays, chaque pente, presque chaque arbre. Il resta en altitude, longeant le flanc de la colline, jusqu'à atteindre l'anneau de sortie où les trains franchissaient le gouffre, et là il déboucha sur la route utilisée par les équipes de maintenance de la ligne circulaire. Il s'assit et enfila ses chaussures. La route de maintenance n'était guère plus qu'une piste taillée à travers les arbres~; elle convenait aux chenilles de tracteur mais pas aux roues. Mais elle descendait à travers le gouffre et remontait jusqu'à l'endroit où la ligne circulaire disparaissait dans le tunnel traversant la crête opposée. Il la suivit, avançant bon train avec la démarche souple et relâchée du montagnard né.

Soixante-dix minutes plus tard, il avait traversé le gouffre et passait sous l'anneau d'entrée. Il continua jusqu'à être près de l'anneau qui marquait l'entrée noire du tunnel. Il s'arrêta à ce qu'il jugea être une distance sûre et évalua ses chances.

La crête était haute, sinon les anneaux auraient été construits dans une tranchée plutôt qu'un tunnel. Il y avait souvent chassé et savait qu'il faudrait deux heures pour l'escalader --- en plein jour. Mais la route de maintenance traversait la colline en ligne droite, sous les anneaux. S'il la suivait, il pourrait passer en dix ou quinze minutes.

Max n'avait jamais traversé la crête. Légalement, c'était une intrusion --- non que cela le dérangeât~; il était déjà en situation d'intrusion en ce moment. De temps en temps, un cochon ou un animal sauvage s'aventurait dans le tunnel et s'y trouvait piégé quand un train le traversait à toute allure. Ils mouraient, instantanément et sans une égratignure. Une fois, Max avait repéré la carcasse d'un renard juste à l'intérieur du tunnel et s'était faufilé pour la récupérer. Elle ne portait aucune marque, mais quand il l'avait dépouillé, il avait découvert que c'était une masse de petites hémorragies. Plusieurs années auparavant, un homme avait été pris à l'intérieur~; l'équipe de maintenance avait sorti le corps.

Le tunnel était plus large que les anneaux mais pas plus que nécessaire pour permettre au projectile de chevaucher sa propre onde de choc réfléchie. Tout être vivant dans le tunnel ne pouvait éviter l'onde~; ce coup de tonnerre insoutenable, douloureux à distance, était si chargé d'énergie qu'il signifiait une mort instantanée de près.

Mais Max ne voulait pas escalader la crête~; il passa en revue dans sa tête le programme des trains du soir. Le \textit{Tomahawk} était celui qu'il avait regardé au coucher du soleil~; le \textit{Javelin}, il l'avait entendu pendant qu'il se cachait dans la grange. L'\textit{Assegai} avait dû passer il y a un bon moment bien qu'il ne se souvînt pas de l'avoir entendu~; il ne restait que le \textit{Cleaver} de minuit.

Il regarda alors le ciel. Vénus s'était couchée, bien sûr, mais il fut surpris de voir Mars encore à l'ouest. La Lune ne s'était pas levée. Voyons --- la pleine lune était mercredi dernier. Sûrement\ldots

Le résultat qu'il obtint lui parut faux, alors il se vérifia en prenant une visée soigneuse de Véga à l'œil nu et la compara avec ce que lui indiquait la Grande Ourse. Puis il siffla doucement --- malgré tout ce qui s'était passé, il n'était que dix heures, à cinq minutes près~; les étoiles ne pouvaient pas se tromper. Auquel cas l'\textit{Assegai} n'était pas attendu avant trois quarts d'heure.

Sauf la faible chance d'un train spécial, il avait largement le temps. Il s'enfonça dans le tunnel.

Il n'avait pas fait cinquante mètres qu'il commença à le regretter et à paniquer un peu~; il faisait noir comme dans un cercueil scellé. Mais la progression était bien plus facile car le tube était revêtu pour permettre des réflexions lisses de l'onde de choc. Il était en route depuis plusieurs minutes, sentant chaque pas mais se pressant, quand ses yeux, s'adaptant à l'obscurité complète, distinguèrent un faible cercle gris au loin. Il se mit à trottiner puis à courir à perdre haleine à mesure que sa peur du lieu s'accumulait.

Il atteignit l'extrémité opposée avec la gorge brûlée et desséchée et le cœur au bord de l'éclatement~; là, il dévala la pente sans se soucier du soudain durcissement de son chemin en quittant le tunnel pour retrouver la piste de maintenance. Il ne ralentit que lorsqu'il se trouva sous des pylônes si hauts que l'anneau au-dessus paraissait petit. Là, il s'arrêta et lutta pour reprendre son souffle.

Il fut projeté en avant et renversé.

Il se releva péniblement, finit par se rappeler où il était et réalisa qu'il avait perdu connaissance. Il avait du sang sur une joue et ses mains et ses coudes étaient à vif. Ce ne fut qu'en remarquant cela qu'il comprit ce qui s'était passé~; un train était passé juste au-dessus de lui. Il n'avait pas été assez près pour tuer, mais assez près pour le projeter au sol.

Ce ne pouvait pas être l'\textit{Assegai}~; il regarda de nouveau les étoiles et le confirma. Non, ce devait être un train spécial --- et il était sorti du tunnel environ une minute avant lui.

Il se mit à trembler et il lui fallut plusieurs minutes pour se ressaisir, après quoi il repartit sur la route de maintenance aussi vite que son corps meurtri le lui permettait.

Bientôt il prit conscience d'un fait étrange~; la nuit était silencieuse. Mais la nuit n'est jamais silencieuse. Ses oreilles, accordées depuis l'enfance aux sons et aux signes de ses collines, auraient dû percevoir un motif sans fin de petits bruits nocturnes --- le vent dans les feuilles, les courses furtives de ses petits cousins, les rainettes, les appels d'insectes, les hiboux.

Par une logique brutale, il conclut correctement qu'il n'entendait plus --- \og sourd comme un pot \fg{} --- l'onde de choc l'avait rendu sourd. Mais il n'y avait rien à y faire, alors il continua~; il ne lui vint pas à l'esprit de rentrer chez lui.

Au fond de ce vallon, où les pylônes faisaient près de quatre-vingt-dix mètres de haut, la route de maintenance croisait un chemin de ferme. Il tourna vers le bas de la colline, ayant accompli son premier objectif d'entrer dans un territoire où Montgomery serait moins susceptible de le chercher. Il était maintenant dans un autre bassin versant~; bien que toujours à quelques kilomètres seulement de chez lui, néanmoins en traversant la crête, il s'était mis dans un autre voisinage. Il continua à descendre pendant quelques heures. La route n'était guère plus qu'un chemin de charrette mais elle était plus facile que la route de maintenance. Quelque part en contrebas, quand les collines céderaient la place à la vallée où vivaient les \og étrangers \fg{}, il trouverait l'autoroute de fret qui longeait la ligne circulaire sur la route d'Earthport --- Earthport étant sa destination bien qu'il n'eût que des plans flous sur ce qu'il ferait en y arrivant.

La Lune était maintenant derrière lui et il avançait bien. Un lapin bondit sur la route devant lui, se dressa et le fixa, puis détala. En le voyant, il regretta de ne pas avoir emporté sa carabine à écureuils. Certes, elle était usée et ne valait pas grand-chose, et dernièrement il était devenu de plus en plus difficile d'acheter les projectiles de cette petite arme obsolète --- mais un lapin au pot en ce moment serait vraiment bienvenu, vraiment bienvenu~!

Il réalisa qu'il était non seulement épuisé mais terriblement affamé. Il avait à peine touché à son dîner et il semblait qu'il allait petit-déjeuner de l'air du temps.

Peu après, son attention fut détournée de la faim par un bourdonnement dans ses oreilles, un bourdonnement qui s'aggravait de façon inquiétante. Il secoua la tête et se frappa les oreilles mais cela n'aida pas~; il dut se résoudre à l'ignorer. Après encore un kilomètre environ, il remarqua soudain qu'il pouvait s'entendre marcher. Il s'arrêta net, puis frappa dans ses mains. Il pouvait les entendre claquer, perçant à travers le bourdonnement fantôme.

Le cœur plus léger, il reprit sa route. Enfin il déboucha sur un épaulement qui dominait la large vallée. Au clair de lune, il pouvait distinguer le tracé de l'autoroute de fret menant vers le sud-ouest et pouvait détecter, lui semblait-il, ses lignes de guidage fluorescentes. Il pressa le pas.

Il approchait de l'autoroute et pouvait entendre le vrombissement des camions qui passaient quand il repéra une lumière devant lui. Il s'en approcha prudemment, détermina que ce n'était ni un véhicule ni une ferme. En s'approchant encore, il vit que c'était un petit feu de camp, visible depuis la colline mais masqué de l'autoroute par un éperon calcaire. Un homme était accroupi au-dessus, remuant le contenu d'une boîte posée sur des pierres au-dessus du feu.

Max se rapprocha furtivement jusqu'à dominer le campement de vagabonds. Il sentit une bouffée du ragoût et sa bouche se mit à saliver. Tiraillé entre la faim et la méfiance viscérale du montagnard envers les \og étrangers \fg{}, il resta immobile et observa.

Bientôt l'homme retira la boîte du feu et appela~: \og Allons, ne reste pas caché~! Descends. \fg{}

Max fut trop surpris pour répondre. L'homme ajouta~: \og Descends dans la lumière. Je ne vais pas te l'apporter. \fg{}

Max se leva et descendit en traînant les pieds dans le cercle de lumière du feu. L'homme leva les yeux. \og Salut. Tire-toi une chaise. \fg{}

\og Salut. \fg{} Max s'assit de l'autre côté du feu, face au vagabond. Il était encore moins bien habillé que Max et avait besoin d'un rasage. Néanmoins, il portait ses haillons avec un air désinvolte et se comportait avec l'assurance d'un moineau.

L'homme continua à remuer le mélange dans la boîte puis en préleva un échantillon à la cuillère, souffla dessus et le goûta. \og Parfait \fg{}, annonça-t-il. \og Ragoût de quatre jours, juste à point. Trouve-toi une assiette. \fg{} Il se leva et fouilla dans un tas de boîtes plus petites derrière lui, en sélectionna une.

Max hésita, puis fit de même, optant pour une qui avait autrefois contenu du café et ne semblait pas avoir été utilisée depuis. Son hôte le servit généreusement de ragoût, puis lui tendit une cuillère.

Max la regarda. \og Si tu ne fais pas confiance au dernier qui l'a utilisée \fg{}, dit l'homme raisonnablement, \og passe-la au feu, puis essuie-la. Moi, je ne m'inquiète pas. Si un microbe me mord, il meurt dans d'atroces souffrances. \fg{}

Max suivit le conseil, tenant la cuillère dans les flammes jusqu'à ce que le manche devienne trop chaud, puis l'essuya sur sa chemise. Le ragoût était bon et sa faim le rendait sublime. La sauce était épaisse, il y avait des légumes et de la viande non identifiée. Max ne se soucia pas du pedigree des ingrédients~; il se contenta d'apprécier.

Au bout d'un moment, son hôte demanda~: \og Encore~? \fg{}

\og Hein~? Bien sûr. Merci~! \fg{}

La deuxième boîte de ragoût le rassasia et répandit dans ses tissus une douce chaleur de bien-être. Il s'étira paresseusement, savourant sa fatigue.

\og Tu te sens mieux~? \fg{} demanda l'homme.

\og Oh oui. Merci. \fg{}

\og Au fait, tu peux m'appeler Sam. \fg{}

\og Oh, moi c'est Max. \fg{}

\og Enchanté, Max. \fg{}

Max attendit avant de soulever un point qui le tracassait. \og Euh, Sam~? Comment tu as su que j'étais là~? Tu m'as entendu~? \fg{}

Sam sourit. \og Non. Mais tu te découpais sur le ciel. Ne refais jamais ça, petit, ou ce sera peut-être la dernière chose que tu feras. \fg{}

Max se retourna et regarda l'endroit où il s'était tapi. En effet, Sam avait raison. Ça alors~!

Sam ajouta~: \og T'as fait du chemin~? \fg{}

\og Hein~? Ouais, un bon bout. \fg{}

\og Tu vas loin~? \fg{}

\og Euh, assez loin, j'imagine. \fg{}

Sam attendit, puis dit~: \og Tu crois que ta famille va te chercher~? \fg{}

\og Hein~? Comment tu le sais~? \fg{}

\og Que tu t'es enfui de chez toi~? Eh bien, c'est le cas, non~? \fg{}

\og Ouais. Ouais, j'imagine que oui. \fg{}

\og Tu avais l'air crevé quand tu t'es traîné jusqu'ici. Peut-être qu'il n'est pas trop tard pour tuer l'oie avant que tes ponts ne soient brûlés. Réfléchis-y, petit. C'est dur sur la route. Je sais. \fg{}

\og Y retourner~? Je n'y retournerai jamais~! \fg{}

\og À ce point-là~? \fg{}

Max fixa le feu. Il avait terriblement besoin de mettre de l'ordre dans ses pensées, même si cela signifiait raconter ses affaires privées à un étranger --- et cet homme à la voix douce était facile à qui parler.

\og Écoute, Sam, tu as déjà eu une belle-mère~? \fg{}

\og Hein~? Je ne me souviens pas d'en avoir jamais eu. Le Centre de Développement pour Enfants de l'État du New Jersey Central me bordait le soir. \fg{}

\og Oh. \fg{} Max déversa son histoire avec une question compatissante occasionnelle de Sam pour en démêler la confusion.

\og Alors je me suis tiré \fg{}, conclut-il. \og Il n'y avait rien d'autre à faire. N'est-ce pas~? \fg{}

Sam pinça les lèvres. \og Je suppose que non. Ce double beau-père que tu as --- il a l'air d'une souris qui étudie pour devenir un rat. Tu es bien débarrassé de lui. \fg{}

\og Tu ne crois pas qu'ils vont essayer de me retrouver et de me ramener, n'est-ce pas~? \fg{}

Sam s'interrompit pour mettre un morceau de bois sur le feu. \og Je n'en suis pas si sûr. \fg{}

\og Hein~? Pourquoi~? Je ne lui sers à rien. Il ne m'aime pas. Et M'man s'en fichera, pas vraiment. Elle pleurnichera peut-être un peu, mais elle ne lèvera pas le petit doigt. \fg{}

\og Eh bien, il y a la ferme. \fg{}

\og La ferme~? Je m'en fiche, maintenant que Papa n'est plus là. Franchement, elle ne vaut pas grand-chose. On se casse le dos à essayer de faire une récolte. Si la Loi sur la Conservation Alimentaire n'avait pas interdit aux propriétaires de laisser les terres agricoles en friche, Papa aurait arrêté de cultiver depuis longtemps. Il faudrait quelque chose comme cette expropriation gouvernementale pour permettre de trouver quelqu'un pour la reprendre. \fg{}

\og C'est ce que je veux dire. Ce type a fait vendre la ferme à ta mère. Or, mes connaissances juridiques ne valent peut-être pas grand-chose, mais il me semble que cet argent devrait te revenir. \fg{}

\og Quoi~? Oh, je me fiche de l'argent. Je veux juste m'éloigner d'eux. \fg{}

\og Ne parle pas comme ça de l'argent~; les autorités vont t'enfermer pour blasphème. Mais peu importe ce que tu ressens, car je pense que le citoyen Montgomery va vouloir te voir très, très fort. \fg{}

\og Pourquoi~? \fg{}

\og Ton père a laissé un testament~? \fg{}

\og Non. Pourquoi~? Il n'avait rien à léguer à part la ferme. \fg{}

\og Je ne connais pas les subtilités des lois de ton État, mais il est certain qu'au moins la moitié de cette ferme t'appartient. Possiblement, ta belle-mère n'a qu'un usufruit sur sa moitié, avec retour à toi quand elle mourra. Mais c'est une certitude qu'elle ne peut pas faire un acte de vente valide sans ta signature. Vers l'heure où le tribunal de ton comté ouvrira demain matin, les acheteurs vont s'en rendre compte. Alors ils vont rappliquer ventre à terre, à la recherche d'elle --- et de toi. Et dix minutes plus tard, ce Montgomery se mettra à te chercher, s'il ne l'a pas déjà fait. \fg{}

\og Oh là là~! S'ils me trouvent, est-ce qu'ils peuvent me forcer à y retourner~? \fg{}

\og Ne te laisse pas trouver. Tu as pris un bon départ. \fg{}

Max ramassa son sac à dos. \og Je ferais mieux de me remettre en route. Merci beaucoup, Sam. Peut-être que je pourrai t'aider un jour. \fg{}

\og Assieds-toi. \fg{}

\og Écoute, je ferais mieux de m'éloigner autant que possible. \fg{}

\og Petit, tu es épuisé et ton jugement s'en ressent. Quelle distance tu peux parcourir cette nuit, dans l'état où tu es~? Demain matin, à la première heure, on descendra à l'autoroute, on la suivra sur un ou deux kilomètres jusqu'au restaurant de routiers au sud d'ici et on interceptera les camionneurs quand ils sortiront du petit-déjeuner, de bonne humeur. On leur soutirera un transport et tu iras plus loin en dix minutes que tu ne pourrais faire de toute la nuit. \fg{}

Max dut admettre qu'il était fatigué, épuisé même, et Sam connaissait certainement mieux ces ficelles que lui.

Sam ajouta~: \og T'as une couverture dans ton baluchon~? \fg{}

\og Non. Juste une chemise\ldots et des livres. \fg{}

\og Des livres, hein~? Je lis pas mal moi-même, quand j'en ai l'occasion. Je peux les voir~? \fg{}

Avec une certaine réticence, Max les sortit. Sam les tint près du feu et les examina. \og Eh bien, je suis un Martien à trois yeux~! Petit, tu sais ce que tu as là~? \fg{}

\og Bien sûr. \fg{}

\og Mais tu ne devrais pas avoir ça. Tu n'es pas membre de la Guilde des Astrogateurs. \fg{}

\og Non, mais mon oncle l'était. Il était du premier voyage vers Bêta Hydrae \fg{}, ajouta-t-il fièrement.

\og Sans blague~! \fg{}

\og Aussi sûr que les impôts. \fg{}

\og Mais tu n'as jamais été dans l'espace toi-même~? Non, bien sûr que non. \fg{}

\og Mais je vais y aller~! \fg{} Max admit quelque chose qu'il n'avait jamais dit à personne, son ambition d'imiter son oncle et de partir vers les étoiles.

Sam écouta pensivement. Quand Max s'arrêta, il dit lentement~: \og Alors tu veux devenir astrogateur~? \fg{}

\og Absolument. \fg{}

Sam se gratta le nez. \og Écoute, petit, je ne veux pas te décourager, mais tu sais comment va le monde. Devenir astrogateur est presque aussi difficile que d'entrer dans la Guilde des Plombiers. La soupe est claire ces temps-ci et il n'y en a pas assez pour tout le monde. La guilde ne va pas t'accueillir juste parce que tu as envie d'être apprenti. L'adhésion est héréditaire, comme dans toutes les autres guildes bien payées. \fg{}

\og Mais mon oncle était membre. \fg{}

\og Ton oncle n'est pas ton père. \fg{}

\og Non, mais un membre qui n'a pas de fils peut désigner quelqu'un d'autre. L'oncle Chet me l'a expliqué. Il m'a toujours dit qu'il allait enregistrer ma nomination. \fg{}

\og Et l'a-t-il fait~? \fg{}

Max resta silencieux. À l'époque de la mort de son oncle, il avait été trop jeune pour savoir comment vérifier. Quand son père avait suivi son oncle, les événements s'étaient enchaînés --- il n'avait jamais vérifié, préférant inconsciemment entretenir le rêve plutôt que de le mettre à l'épreuve.

\og Je ne sais pas \fg{}, dit-il enfin. \og Je vais au Chapitre Mère à Earthport pour le savoir. \fg{}

\og Hmm --- je te souhaite bonne chance, petit. \fg{} Il fixa le feu, avec tristesse sembla-t-il à Max. \og Bon, je vais piquer un roupillon, et tu ferais bien de faire pareil. Si tu as froid, tu trouveras des trucs là-bas sous ce surplomb rocheux --- de la toile de jute et des matériaux d'emballage et autres. Ça te tiendra chaud, si ça ne te dérange pas de risquer une puce ou deux. \fg{}

Max rampa dans le trou sombre indiqué, trouva une demi-grotte dans le calcaire. À tâtons, il localisa la litière primitive. Il s'était attendu à rester éveillé, mais il s'endormit avant que Sam n'eût fini de couvrir le feu.

Il fut réveillé par le soleil qui flamboyait à l'extérieur. Il sortit en rampant, se leva et s'étira pour chasser la raideur de ses membres. D'après le soleil, il jugea qu'il devait être environ sept heures du matin.

Sam n'était pas en vue. Il regarda autour de lui et appela, pas trop fort, supposant que Sam était descendu au ruisseau pour boire et se laver à l'eau froide. Max retourna dans l'abri et sortit son sac à dos, avec l'intention de changer de chaussettes.

Les livres de son oncle avaient disparu.

Il y avait un mot sur sa chemise de rechange~: \og Cher Max \fg{}, disait-il, \og il reste du ragoût dans la boîte. Tu peux le réchauffer pour le petit-déjeuner. À plus --- Sam. P.S. Désolé. \fg{}

Une fouille plus poussée révéla que sa carte d'identité manquait aussi, mais Sam ne s'était pas donné la peine de prendre ses autres maigres possessions. Max ne toucha pas au ragoût mais se mit en route sur la route, l'esprit empli de pensées amères.
