\chapter{\og ...plus de cent ans... \fg{}}

L'\emph{Asgard} atterrit sur Charité le lendemain. Eldreth imposa son choix par le processus statistique de désigner la planète par ce nom, supposant que c'était officiel, et le répétant fréquemment.

Quand l'ordre passa que l'atterrissage commencerait à midi, heure du vaisseau, Max alla à la salle de contrôle et supposa simplement que c'était son droit d'être présent. Simes le regarda d'un air maussade mais ne dit rien -- pour une raison évidente~: le Capitaine Blaine était présent.

Max fut choqué par son apparence. Le Capitaine semblait avoir vieilli de dix à quinze ans depuis la mauvaise transition. À la place de son expression joyeuse habituelle, il y en avait une que Max eut du mal à définir -- jusqu'à ce qu'il se rappelle l'avoir vue sur des chevaux, sur des chevaux trop vieux pour travailler mais travaillant encore -- tête baissée, yeux ternes, muets et résignés face à un sort à la fois inévitable et insupportable. La peau du vieil homme pendait, comme s'il n'avait pas mangé depuis des jours ou des semaines. Il semblait à peine intéressé par ce qui se passait autour de lui.

Il ne parla qu'une fois pendant la manœuvre. Juste avant que le chronomètre n'indique midi, Simes se redressa de la console et regarda son capitaine. Blaine leva la tête et dit dans un murmure rauque~: \og Descendez-la, Monsieur. \fg{}

Un vaisseau militaire impérial atterrissant sur un endroit inconnu guiderait normalement d'abord un robot-balise radar, puis se dirigerait sur la balise. Mais l'\emph{Asgard} était un paquebot marchand~; il ne s'attendait à atterrir nulle part sauf dans des ports équipés de faisceaux et de balises et d'autres aides. Par conséquent l'atterrissage fut fait à l'aveugle en radar-automatique précalculé et était prévu pour une vallée ouverte sélectionnée par photographie. La planète était densément boisée dans la plupart des zones, le choix était limité.

Simes présentait l'image du pilote alerte, les mains en position sur les commandes, les yeux sur l'écran radar montrant la vue en dessous d'eux, tandis que devant lui étaient rangées des photographies de comparaison, radar et visuelles. La descente fut sans incident~; le ciel noir étoilé céda la place à un violet profond, puis au bleu. Il n'y eut même pas de secousse quand le vaisseau toucha le sol, car sa gravité privée à l'intérieur de son champ horstien les empêchait de sentir l'accélération imposée. Max sut qu'ils étaient posés quand il vit Simes enclencher les faisceaux de poussée pour maintenir le vaisseau droit.

Simes dit dans le microphone~: \og Salle des machines, démarrez les auxiliaires et coupez. Tout l'équipage, routine terrestre, première section. \fg{} Il se tourna vers Blaine. \og Posés, Capitaine. \fg{}

Les lèvres de Blaine formèrent les mots~: \og Très bien, monsieur. \fg{} Il se leva et traîna les pieds vers l'écoutille. Quand il fut parti, Simes ordonna~: \og Lundy, prends le quart de garde. Vous autres, dégagez la salle de contrôle. \fg{}

Max descendit avec Kelly. Quand ils atteignirent le pont \og A \fg{}, Max dit à contrecœur~: \og C'était un atterrissage habile, je dois l'admettre. \fg{}

\og Merci, \fg{} dit Kelly.

Max le regarda. \og Alors c'est toi qui l'as calculé~? \fg{}

\og J'ai pas dit ça. J'ai juste dit "Merci". \fg{}

\og Ah~? Eh bien, de rien. \fg{}

Max sentit son poids pulser et soudain il fut un peu plus léger. \og Ils ont coupé le champ. Maintenant on est vraiment posés. \fg{} Il était sur le point d'inviter Kelly dans sa chambre pour l'inévitable café quand les haut-parleurs du vaisseau retentirent~:

\og Tout l'équipage~! Tous les passagers~! Présentez-vous au Salon Bifrost pour une annonce importante. Ceux qui sont de quart ont ordre d'écouter par téléphone. \fg{}

\og Qu'est-ce qui se passe~? \fg{} demanda Max.

\og Pourquoi se demander~? On va voir. \fg{}

Le salon était bondé de passagers et d'équipage. Le Premier Officier Walther se tenait près de la table du Capitaine, comptant la foule des yeux. Max le vit parler à Bennett, qui hocha la tête et se hâta de partir. Le grand hublot était de l'autre côté du salon par rapport à Max~; il se dressa sur la pointe des pieds et essaya de voir dehors. Tout ce qu'il put voir fut des sommets de collines et un ciel bleu.

Il y eut une diminution du murmure des voix~; Max se retourna pour voir Bennett précéder le Capitaine Blaine à travers la foule. Le Capitaine alla à sa table et s'assit~; le Premier Officier lui jeta un coup d'œil, puis se racla bruyamment la gorge.

\og Silence, s'il vous plaît. \fg{} Il poursuivit~: \og Je vous ai réunis parce que le Capitaine Blaine a quelque chose à vous dire. \fg{} Il s'arrêta et recula respectueusement.

Le Capitaine Blaine se leva lentement, regarda autour de lui avec incertitude. Max le vit redresser ses épaules maigres et lever la tête. \og Messieurs, \fg{} dit-il, sa voix soudain ferme et forte. \og Mes invités et amis -- \fg{} continua-t-il, sa voix faiblissant. Il y eut un silence dans le salon, Max pouvait entendre la respiration laborieuse du Capitaine. Il reprit le contrôle de lui-même et continua~: \og Je vous ai amenés... je vous ai amenés aussi loin que je le pouvais... \fg{} Sa voix s'éteignit. Il les regarda pendant un long moment, la bouche tremblante. Il semblait impossible qu'il continue. La foule commença à s'agiter.

Mais il continua et ils se calmèrent immédiatement. \og J'ai autre chose à dire, \fg{} commença-t-il, puis fit une pause. Cette pause fut plus longue, quand il la rompit sa voix était un murmure. \og Je suis désolé. Que Dieu vous garde tous. \fg{} Il se retourna et se dirigea vers la porte. Bennett se glissa rapidement devant lui. Max pouvait l'entendre dire doucement et fermement~: \og Laissez passer, s'il vous plaît. Place pour le Capitaine. \fg{}

Personne ne dit rien jusqu'à ce qu'il soit parti, mais une passagère au coude de Max sanglotait doucement. La voix nette et claire de M. Walther retentit. \og Ne partez pas~! J'ai des annonces supplémentaires à faire. \fg{} Son attitude ignora ce qu'ils venaient tous de voir. \og Le moment est venu de résumer notre situation actuelle. Comme vous pouvez le voir, cette planète ressemble beaucoup à notre Terre Mère. Des tests doivent être faits pour s'assurer que l'atmosphère est respirable, et ainsi de suite~; le Médecin et le Chef Mécanicien les font en ce moment. Mais il semble probable que cette nouvelle planète s'avérera éminemment convenable pour les êtres humains, probablement même plus accueillante que la Terre.

\og Jusqu'ici, nous n'avons vu aucune indication de vie civilisée. Dans l'ensemble, cela semble une bonne chose. Maintenant quant à nos ressources -- L'\emph{Asgard} transporte une variété d'animaux domestiques, ils seront utiles et devraient être conservés comme cheptel reproducteur. Nous avons une variété encore plus large de plantes utiles, à la fois dans les jardins hydroponiques du vaisseau et transportées comme graines. Nous avons un approvisionnement limité mais adéquat d'outils. Plus important que tout, la bibliothèque du vaisseau contient un échantillon représentatif de notre culture. Tout aussi important, nous avons nous-mêmes nos compétences et traditions... \fg{}

\og M. Walther~! \fg{}

\og Oui, M. Hornsby~? \fg{}

\og Essayez-vous de nous dire que vous nous larguez ici~? \fg{}

Walther le regarda froidement. \og Non. Personne n'est "largué" comme vous dites. Vous pouvez rester dans le vaisseau et vous serez traité comme un invité aussi longtemps que l'\emph{Asgard} -- ou vous-même -- sera en vie. Ou jusqu'à ce que le vaisseau atteigne la destination sur votre billet. S'il y arrive. Non, j'ai essayé de discuter raisonnablement d'un secret de Polichinelle~; ce vaisseau est perdu. \fg{}

Un soupir silencieux parcourut la salle. Tous le savaient, mais jusqu'à présent cela n'avait pas été admis officiellement. L'annonce brutale d'un officier responsable résonna comme la sentence d'un tribunal.

\og Permettez-moi d'exposer la position légale, \fg{} poursuivit M. Walther. \og Pendant que ce vaisseau était dans l'espace, vous passagers étiez soumis à l'autorité du Capitaine, telle que définie par la loi, et par lui vous étiez soumis à moi et aux autres officiers du vaisseau. Maintenant nous avons atterri. Vous pouvez partir librement... ou vous pouvez rester. Légalement c'est une escale non prévue~; si le vaisseau quitte un jour cet endroit vous pouvez y retourner et continuer comme passagers. C'est ma responsabilité envers vous et elle sera assumée. Mais je vous dis franchement qu'à présent je n'ai aucun espoir à offrir que nous quitterons jamais cet endroit -- c'est pourquoi j'ai parlé de colonisation. Nous sommes perdus. \fg{}

Au fond de la salle une femme commença à crier hystériquement, avec des sons incohérents de~: \og ...la maison~! Je veux rentrer~! Emmenez-moi... \fg{}

La voix de Walther coupa à travers le brouhaha. \og Dumont~! Flannigan~! Emmenez-la. Conduisez-la au Médecin. \fg{} Il continua comme si rien ne s'était passé. \og Le vaisseau et l'équipage du vaisseau donneront toute l'assistance possible, compatible avec ma responsabilité légale de maintenir le vaisseau en service, pour aider quiconque parmi vous souhaite coloniser. Personnellement je pense... \fg{}

Une voix maussade coupa~: \og Pourquoi parler de "loi"~? Il n'y a pas de loi ici~! \fg{}

Walther n'éleva même pas la voix. \og Mais il y en a une. Tant que ce vaisseau est en service, il y a une loi, peu importe combien d'années-lumière il peut être de son port d'attache. De plus, bien que je n'aie aucune autorité sur ceux qui choisissent de quitter le vaisseau, je vous conseille fortement de faire de votre premier acte sur terre une assemblée municipale, d'élire des officiers, et de fonder un gouvernement constitutionnel. Je doute que vous puissiez survivre autrement. \fg{}

\og M. Walther. \fg{}

\og Oui, M. Daigler~? \fg{}

\og Ce n'est évidemment pas le moment des récriminations... \fg{}

\og Évidemment~! \fg{}

Daigler sourit ironiquement. \og Alors je ne m'y livrerai pas, bien que je puisse en penser. Mais il se trouve que je connais quelque chose professionnellement sur l'économie de la colonisation. \fg{}

\og Bien~! Nous utiliserons vos connaissances. \fg{}

\og Vous voulez bien me laisser finir~? Un principe premier pour maintenir une colonie hors de contact avec sa base d'approvisionnement est de la rendre assez grande. C'est une question statistique, une colonie trop petite peut être submergée par un revers mineur. C'est comme aller à une partie de dés avec trop peu d'argent~: trois mauvais lancers et vous êtes coulé. En regardant autour de moi, il est évident que nous avons beaucoup moins que le minimum optimal. En fait -- \fg{}

\og C'est ce que nous avons, M. Daigler. \fg{}

\og Je le vois. Je ne suis pas un rêveur. Ce que je veux savoir, c'est si nous pouvons compter sur l'équipage aussi~? \fg{}

M. Walther secoua la tête. \og Ce vaisseau ne sera pas désarmé tant qu'il y aura des hommes capables de le manœuvrer. Il y a toujours de l'espoir, aussi petit soit-il, que nous puissions trouver un chemin pour rentrer. Il est même possible qu'un vaisseau de reconnaissance impérial nous découvre. Je suis désolé -- non. \fg{}

\og Ce n'est pas tout à fait ce que j'ai demandé. J'avais deux coups d'avance sur vous, je me doutais que vous ne laisseriez pas l'équipage coloniser. Mais pouvons-nous compter sur leur aide~? Nous semblons avoir environ six femmes, à peu près, qui vont probablement aider à perpétuer la race. Cela signifie que la prochaine génération de notre nouvelle nation va être beaucoup plus petite. Une telle colonie vacillerait et mourrait, par probabilité statistique -- à moins que chacun d'entre nous ne travaille dix heures par jour pour le reste de sa vie, juste pour donner à nos enfants une meilleure chance de s'en sortir. Ça me va, si nous faisons tous un effort total. Mais il faudra toute la main-d'œuvre que nous avons pour s'assurer que des jeunes gens qui ne sont même pas encore nés s'en sortent dans trente ans. L'équipage aidera-t-il~? \fg{}

M. Walther dit calmement~: \og Je pense que vous pouvez compter dessus. \fg{}

\og Ça me suffit. \fg{}

Un petit homme au visage rouge dont Max n'avait jamais appris le nom interrompit. \og Ça me suffit, mon œil~! Je vais poursuivre la compagnie, je vais poursuivre les officiers du vaisseau individuellement. Je vais le crier sur les... \fg{}

Max vit Sam se faufiler à travers la foule jusqu'au côté de l'homme, le trouble cessa brusquement. \og Emmenez-le au Médecin, \fg{} dit M. Walther avec lassitude. \og Il peut nous poursuivre demain. La réunion est levée. \fg{}

Max se dirigea vers sa chambre. Eldreth le rattrapa. \og Max~! Je veux te parler. \fg{}

\og D'accord. \fg{} Il retourna vers le salon.

\og Non, je veux parler en privé. Allons dans ta chambre. \fg{}

\og Hein~? Mme Dumont piquerait une crise, puis elle le dirait à M. Walther. \fg{}

\og Au diable tout ça~! Ces règles stupides sont mortes. Tu n'as pas écouté à la réunion~? \fg{}

\og C'est toi qui n'as pas écouté. \fg{} Il la prit fermement par le bras, la tourna vers la salle publique. Ils tombèrent sur M. et Mme Daigler venant en sens inverse. Daigler dit~: \og Max~? Vous êtes occupé~? \fg{}

\og Oui, \fg{} répondit Eldreth.

\og Non, \fg{} dit Max.

\og Hmm... vous feriez mieux de voter tous les deux. J'aimerais poser quelques questions à Max. Je n'ai pas d'objection à ce que tu sois avec nous, Eldreth, si tu veux pardonner l'intrusion. \fg{}

Elle haussa les épaules. \og Oh, tant pis, peut-être que tu arriveras à le manœuvrer. Moi je n'y arrive pas. \fg{}

Ils allèrent à la cabine des Daigler, plus grande et plus luxueuse que celle de Max et possédant deux fauteuils. Les deux femmes se perchèrent sur le lit, les hommes prirent les fauteuils. Daigler commença~: \og Max, vous m'impressionnez comme un homme qui préfère donner une réponse directe. Il y a des choses que je veux savoir que je n'ai pas eu envie de demander là-bas. Peut-être pouvez-vous me dire. \fg{}

\og Je le ferai si je peux. \fg{}

\og Bien. J'ai essayé de demander à M. Simes, tout ce que j'obtiens c'est un rebuffade mielleusement polie. Je n'ai pas pu voir le Capitaine -- après aujourd'hui je vois que ça n'aurait servi à rien de toute façon. Maintenant, pouvez-vous me dire, sans les mathématiques, quelle chance nous avons de rentrer~? Est-ce une sur trois, ou une sur mille -- ou quoi~? \fg{}

\og Euh, je ne pourrais pas répondre de cette façon. \fg{}

\og Répondez à votre façon. \fg{}

\og Eh bien, mettez-le comme ça. Bien que nous ne sachions pas où nous sommes, nous savons positivement où nous ne sommes pas. Nous ne sommes pas à moins de, oh, disons cent années-lumière de toute partie explorée de la Galaxie. \fg{}

\og Comment savez-vous~? Il me semble que c'est un espace assez grand à explorer dans les semaines depuis que nous avons déraillé. \fg{}

\og C'est sûr. C'est un globe de douze cents billions de miles d'épaisseur. Mais nous n'avons pas eu à l'explorer, pas exactement. \fg{}

\og Alors comment~? \fg{}

\og Eh bien, monsieur, nous avons examiné les spectres de toutes les étoiles de première magnitude visibles -- et beaucoup d'autres. Aucune d'elles n'est dans nos catalogues. Certaines sont des géantes qui seraient de première magnitude n'importe où à moins de cent années-lumière d'elles -- elles seraient certainement dans les catalogues si un vaisseau de reconnaissance avait jamais été aussi proche d'elles. Donc nous sommes absolument certains que nous sommes très, très loin de tout endroit où les hommes ont jamais été auparavant. En fait, j'ai parlé trop prudemment. Faites-en un globe deux fois plus épais, huit fois plus gros, et vous seriez encore largement du côté conservateur. Nous sommes \emph{vraiment} perdus. \fg{}

\og Mmm... je suis content de ne pas avoir posé ces questions au salon. Y a-t-il une possibilité que nous sachions un jour où nous sommes~? \fg{}

\og Oh, bien sûr~! Il reste des milliers d'étoiles à examiner. Le Chef Kelly en photographie probablement une en ce moment même. \fg{}

\og Eh bien, alors, quelles sont les chances que nous finissions par nous retrouver~? \fg{}

\og Oh, je dirais qu'elles sont excellentes -- dans un an ou deux au maximum. Sinon à partir d'étoiles individuelles, alors à partir d'amas globulaires. Vous réalisez que la Galaxie fait cent mille années-lumière de diamètre, plus ou moins, et que nous ne pouvons voir que des étoiles qui sont assez proches. Mais les amas globulaires font aussi de bons repères. \fg{}

Max ajouta la réserve mentale, \emph{si nous ne sommes pas dans la mauvaise galaxie}. Il ne semblait pas utile de les accabler avec cette possibilité décourageante.

Daigler se détendit et sortit un cigare. \og C'est le dernier de ma marque, mais je risque de le fumer maintenant. Eh bien, Maggie, je suppose que tu n'auras pas besoin d'apprendre à faire du savon avec des cendres de bois et des graisses de porc après tout. Que ce soit un an ou cinq, nous pouvons attendre et rentrer à la maison. \fg{}

\og Je suis contente. \fg{} Elle tapota sa coiffure ornée avec des mains douces et magnifiquement manucurées. \og Je ne suis guère du genre pour ça. \fg{}

\og Mais vous ne comprenez pas~! \fg{}

\og Hein~? Qu'est-ce que c'est, Max~? \fg{}

\og Je n'ai pas dit que nous pouvions rentrer. J'ai juste dit que je pensais qu'il était assez certain que nous saurions où nous sommes. \fg{}

\og Quelle est la différence~? On le découvre, puis on rentre. \fg{}

\og Non, parce que nous ne \emph{pouvons pas} être à moins de cent années-lumière de l'espace exploré. \fg{}

\og Je ne vois pas le problème. Ce vaisseau peut faire cent années-lumière en une fraction de seconde. Quel a été le plus long saut que nous ayons fait cette croisière~? Presque cinq cents années-lumière, non~? \fg{}

\og Oui, mais -- \fg{} Max se tourna vers Eldreth. \og Tu comprends~? N'est-ce pas~? \fg{}

\og Eh bien, peut-être. Cette histoire d'écharpe pliée que tu m'as montrée~? \fg{}

\og Oui, oui. M. Daigler, bien sûr l'\emph{Asgard} peut transiter cinq cents années-lumière instantanément -- ou n'importe quelle autre distance. Mais \emph{seulement} à des congruences calculées et cartographiées. Nous n'en connaissons aucune à moins de cent années-lumière, au moins... et nous n'en connaîtrons aucune même si nous découvrons où nous sommes parce que nous savons où nous ne sommes \emph{pas}. Vous me suivez~? Cela signifie que le vaisseau devrait voyager à vitesse maximale pendant quelque chose comme plus de cent ans et peut-être beaucoup plus longtemps, juste pour la première étape du voyage. \fg{}

M. Daigler fixa pensivement la cendre de son cigare, puis sortit un canif et coupa le bout qui brûlait. \og Je garde le reste. Eh bien, Maggie, tu ferais mieux d'étudier cette histoire de savon fait maison. Merci, Max. Mon père était fermier, je peux apprendre. \fg{}

Max dit impulsivement~: \og Je vous aiderai, monsieur. \fg{}

\og Oh oui, vous nous avez dit que vous étiez fermier autrefois, n'est-ce pas~? Vous devriez vous en sortir. \fg{} Ses yeux se tournèrent vers Eldreth. \og Vous savez ce que je ferais, si j'étais à votre place, les enfants~? Je ferais le Capitaine vous marier tout de suite. Ensuite vous seriez prêts à affronter la vie coloniale. \fg{}

Max rougit jusqu'au col et ne regarda pas Ellie. \og J'ai peur de ne pas pouvoir. Je suis membre de l'équipage, je ne suis pas éligible pour coloniser. \fg{}

M. Daigler le regarda avec curiosité. \og Quel dévouement au devoir. Eh bien, sans doute Ellie peut-elle choisir parmi les passagers célibataires. \fg{}

Eldreth lissa sa jupe pudiquement. \og Sans doute. \fg{}

\og Viens, Maggie. Tu viens, Eldreth~? \fg{}
